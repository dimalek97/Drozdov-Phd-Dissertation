\newpage
\chapter{Фрактальные $k$-кубы, являющиеся дендритами}

В этой главе мы рассмотрим фрактальные $k$-кубы и опишем последовательность действий, по которым можно определить, является ли данный фрактальный куб дендритом.


\section{Пересечения копий фрактального $k$-куба}


\subsection{Фрактальные $k$-кубы}

\begin{definition}\label{dfn:FQ} 
[Olsen L. (1998) \cite{Olsen1998}; Lau K., Luo J.J.,Rao H. (2013) \cite{LLR2013}]
Пусть  $D=\{d_1,\ldots,d_r\},\; d_i\in\{0,1,\ldots,n-1\}^k$, где $n\ge 2$, а $1<\#D<n^k$.\\
{\em Фрактальным $k$-кубом порядка $n$ с множеством единиц $D$} называют компактное множество $K\IN\rr^k$, удовлетворяющее 
\begin{equation}\label{fqeq}
K=\dfrac{K+D}{n}.
\end{equation}
\end{definition}

\begin{figure}[h!]
    \centering
    \qquad
    \includegraphics[width=0.4\textwidth]{FQD.png}
    \hfill
    \includegraphics[width=0.4\textwidth]{FQK.png}
    \qquad
    \caption{Множество $D+P$ (слева) и фрактальный куб $nK=K+D$ (справа)}
    \label{fig:FQ}
\end{figure}

В случаях, когда $k=2$ и $k=3$, мы называем $K$ \emph{фрактальным квадратом} и \emph{фрактальным кубом} сответственно.

Уравнение \eqref{fqeq} можно использовать в его эквивалентной форме $nK=K+D$.
Само же множество $K$ содержится в единичном $k$-мерном кубе $P=[0,1]^k$, поскольку $P$ удовлетворяет уравнению $nP\NI P+D$.

Рассмотрим также эквивалентное определение фрактального $k$-куба $K$ порядка $n$ с множеством единиц $D$.
Возьмем единичный $k$-куб $P$ и разобьем его на $n^k$ равных $k$-кубов с ребром $1/n$.
Нижний ближайшая к нулю вершина каждого маленького $k$-куба $P_i$ имеет координату вида $\dfrac{d_i}{n}$, где $d_i\in\{0,1,\ldots,n-1\}^k$.
Из этого разбиения выберем те $k$-кубы $P_i$, для которых $d_i\in D$ (см. пример на рисуноке \ref{fig:FQ}).
Гомотетия, переводящая $P$ в $P_i$, имеет вид $S_i(x)=\dfrac{x+d_i}{n}$.
Тогда $K=\bigcup\limits_{d_i\in D}\frac{d_i+K}{n}$.

Уравнение \eqref{fqeq} определяет систему $\eS$ гомотетий $S_i(x)=\dfrac{x+d_i}{n}$, где $d_i\in D$, а оператор Хатчинсона $T_\eS$ системы $\eS$ определяется уравнением $T_\eS(A)=\dfrac{D+A}{n}=\bigcup\limits_{d_i\in D}\dfrac{d_i+A}{n}$.

\begin{definition}\label{refin}
Для любого $m\in \nn$, {\em $m$-е измельчение} системы $\eS$ --- это система $\eS^m=\{S_\bi, \bi=i_1i_2\ldots i_m\in I^m\}$, где $S_\bi(x)=\dfrac{x+d_\bi}{n^m}$ и $d_\bi=n^{m-1}d_{i_1}+n^{m-2}d_{i_2}+\ldots+d_{i_m}$. 
\end{definition}

Системе $\eS^m$ соответствует аттрактор $K$ как фрактальный $k$-куб порядка $n^m$ с множеством единиц $D^m=n^{m-1}D+n^{m-2}D+\ldots+D$ и копиями $\dfrac{K+d_\bi}{n^m}$.
Оператор Хатчинсона $T_\eS^m$ этой системы $\eS^m$ определяется уравнением $T_\eS^m(A)=\dfrac{D^m+A}{n^m}$.

Каждой бесконечной строке $\bi=i_1i_2\ldots\in I^\8$ соответствует единственная точка $x=\pi(\bi)$, где $\pi(\bi)=\sum\limits_{m=1}^\8 \dfrac{d_{i_m}}{n^m}$.

\begin{remark}% \label{rmk:fsd}
Далее, если не указано иное, говоря {\em <<фрактальный $k$-куб $K$>>}, мы будем иметь в виду {\em <<фрактальный $k$-куб $K$ порядка $n$ с множеством единиц $D$>>}. 
\end{remark}


\subsection{Грани $K_\al$ и пересечения граней $F_\al$ фрактального $k$-куба}

Единичный $k$-куб $P$ является фрактальным $k$-кубом порядка $n$ с множеством единиц $D=\{0,1,\ldots,n-1\}^k$, поскольку $P=\dfrac{P+\{0,1,\ldots,n-1\}^k}{n}.$
Значит, любой фрактальный $k$-куб $K$ содержится в единичном $k$-кубе $P$.

Если фрактальный $k$-куб $K$ является аттрактором системы $\eS$, то из включения $K\IN P$ следует $S_i(K)\IN S_i(P)$ для любого $S_i\in\eS$.
Более того, для любых $S_i, S_j\in\eS$ малые $k$-кубы $S_i(P)$ и $S_j(P)$ могут пересекаться друг с другом только по своим граням.
Прообразы этих граней в $S_i(P)$ и $S_j(P)$ относительно соответственных отображений $S_i$ и $S_j$ являются парой противоположных граней в $P$.
Введём обозначения для граней $k$-куба $P$ и пересечений этих граней с фрактальным $k$-кубом $K$.

Рассмотрим множество векторов $A=\{-1,0,1\}^k$.
Между множеством $A$ и множеством граней единичного $k$-куба $P=[0,1]^k$ существует взаимно-однозначное соответствие.

\begin{definition}\label{dfn:Pa}
Каждому вектору $\bma\in A=\{-1,0,1\}^k$ соответствует грань $P_\bma:=P\cap(P+\bma)$ единичного $k$-куба $P$.
\end{definition}

Симметрия противоположных граней $k$-куба $P$ выражается равенством 
$$P_\bma=P\cap(P+\bma)=((P-\bma)\cap P)+\bma=P_{-\bma}+\bma.$$

Для $\bma=(\al_1,\ldots,\al_k)\in A$ введём обозначение $|\bma|=\al_1+\ldots+\al_k$.
Тогда вектору $\bma\in A$, при $|\bma|=l$, соответствует $(k-l)$-мерная грань $P_\bma$ $k$-куьа $P$.

\begin{figure}[h!]
 \centering
 \includegraphics[width=0.45\textwidth]{faces_p.pdf}
 \hfill
 \includegraphics[width=0.45\textwidth]{faces_k.pdf}
 \caption{Множества $P_\bma$ и $D_\bma.$}
 \label{fig:faces}
\end{figure}

Введём отношение порядка $\sqsubseteq$ на $A$, равносильное отношению включения $\supseteq$ граней $k$-куба $P$:

\begin{definition}\label{Aorder}
Пусть $\bma=(\al_1,\al_2,\ldots,\al_k)$, $\bmb=(\be_1,\be_2,\ldots,\be_k)\in A$.
Будем говорить, что $\bma\sqsubseteq\bmb$, если из $\al_i\neq 0$ следует $\be_i=\al_i$.
Если при этом $\bma\neq\bmb$, то $\bma\sqsubset\bmb$.
\end{definition}

Очевидно, что $\bma\sqsubseteq\bmb$ тогда и только тогда, когда $P_\bma\supseteq P_\bmb$. 
Максимальными элементами из $A$ по отношению $\sqsubseteq$ являются векторы $(\pm 1,\ldots,\pm 1)$, которым соответствуют вершины $k$-куба $P$.
Минимальным элементом из $A$ по отношению $\sqsubseteq$ является $0$, которому соответствует единичный $k$-куб $P=P_{0}$.

\begin{definition}\label{def:K_alpha}
Гранью $K_\bma$ фрактального $k$-куба $K$ называется множество $K\cap P_\bma$.
\end{definition}

\begin{figure}[h!]
\centering
\includegraphics[width=0.45\textwidth]{images/presentation/qK.png}
 \hfill
 \includegraphics[width=0.45\textwidth]{images/presentation/qK_a.png}
  \caption{Фрактальный куб $K$ и его грань $K_\bma$}
 \label{fig:qK_a}
\end{figure}

\begin{proposition}\label{prop:Ka}
Грань $K_\bma$ фрактального $k$-куба $K$ удовлетворяет уравнению $n K_\bma=K_\bma+D_\bma$, где $D_\bma=D\cap(n-1)P_\bma$.
\end{proposition}

\begin{proof}
Заметим, что $n(K\cap P_\bma)=(K+D)\cap nP_\bma.$ 
Если $d\in D$ и \linebreak $d\notin(n-1)P_\bma$, то множество $(d+K)\cap nP_\bma$ пусто.
Поэтому мы будем рассматривать только те $d\in D$, для которых $d\in D\cap (n-1)P_\bma$.\\ 
Тогда $(d+K)\cap nP_\bma=d+K_\bma$, следовательно $n K_\bma={K_\bma+D_\bma}$.
\end{proof}

\begin{definition}\label{def:F_alpha}
Пусть даны фрактальные $k$-кубы $K^1$ и $K^2$.
Символом $F_\bma=F_\bma(K^1,K^2)$ обозначим пересечение $K^1_\bma\cap(K^2_{-\bma}+\bma)$ пары граней фрактальных $k$-кубов $K^1$ и $K^2$.
\end{definition}

Далее в тексте, если не указано иное, будем считать, что $K=K^1=K^2$.
Тогда $F_\bma$ есть пересечение $K_\bma\cap(K_{-\bma}+\bma)$ пары противоположных граней фрактального $k$-куба $K$.
В этом случае справедлива следующее предложение.

\begin{proposition}\label{fbma}
Пусть $K$ --- фрактальный $k$-куб.
\begin{itemize}[nolistsep]
\item[1.] Для любого $\bma\in A$ верно равенство $K\cap(K+\bma)=F_\bma=F_{-\bma}+\bma $;
\item[2.] Если $\bi,\bj \in I^k$, $\bi\neq \bj$ и $K_\bi\cap K_\bj\neq\0$, то $d_\bi-d_\bj=\bma$ для некоторого $\bma\in A$, и при этом 
$$K_\bi\cap K_\bj=\dfrac{d_\bi+F_\bma}{n^k}.$$
\end{itemize}
\end{proposition}

\begin{proof}
1. Поскольку $P\cap (P+\bma)=P_\bma=P_{-\bma}+\bma$, то мы имеем цепочку равенств 
$$K\cap(K+\bma)= K_\bma\cap (K_{-\bma}+\bma)=F_\bma=F_{-\bma}+\bma.$$

2. Заметим, что $S_\bi^{-1}(K_\bi\cap K_\bj)=K\cap S_\bi^{-1}(K_\bj)=K\cap (K+d_\bj-d_\bi)$. 
Так как $d_\bj-d_\bi\in\zz^k$, последнее пересечение может быть непусто только в том случае, если $d_\bj-d_\bi$ равно некоторому $\bma\in A$. 
В этом случае $K\cap(K+d_\bj-d_\bi)=F_\bma$.
Значит $K_\bi\cap K_\bj= S_\bi(F_\bma)= \dfrac{d_\bi+F_\bma}{n^k}$. 
\end{proof}

Из предложения \ref{fbma} следует, что для любого фрактального $k$-куба существует $\dfrac{3^k-1}{2}$ способов пересечения смежных копий одинакового размера (поскольку $F_\bma$ и $F_{-\bma}$ соответсвуют одному и тому же способу смежности копий). 
Любое из этих пересечений является образом множества из системы $\{F_\bma\ :\ \bma\in A\mmm\{0\}\}$ относительно некоторого отображения $S_\bi$. 

Пусть $K^1$ и $K^2$ --- фрактальные $k$-кубы порядка $n$ с множествоами единиц $D^1$ и $D^2$, и $F_{\bma}=K^1\cap(K^2+\bma)$.
Тогда при $|\bma|=k$ множество $F_{\bma}=\left(\frac{\al_1+1}{2},\ldots,\frac{\al_k+1}{2}\right)$, если $(n-1)\cdot\left(\frac{\al_1+1}{2},\ldots,\frac{\al_k+1}{2}\right)\in D^1$ и $(n-1)\cdot\left(\frac{-\al_1+1}{2},\ldots,\frac{-\al_k+1}{2}\right)\in D^2$, в противном случае $F_{\bma}=\0$. 
Иными словами, множества $F_{(\pm1,\ldots,\pm1)}$ соответствуют пересечениям противоположных вершин $k$-куба $P$, а потому являются одноточечными или пустыми.

Уравнения, задающие множества $F_\bma$, получаются из следующей теоремы.

\begin{theorem}\label{IFC}
Пусть $K^1$ и $K^2$ --- фрактальные $k$-кубы порядка $n$ с множествоами единиц $D^1$ и $D^2$.
Семейство множеств $\{F_\bma:=K^1_{\bma}\cap(K^2_{-\bma}+\bma)\ :\ \bma\in A\mmm\{0\}$ удовлетворяет системе уравнений
\begin{equation}\label{sideint}
\Sigma=\left\{
F_\bma=\bigcup\limits_{\bmb\sqsupseteq\bma} \dfrac{F_\bmb+G_{\bma\bmb}}{n}\ :\ \bma\in A\right\},
\end{equation}
где $G_{\bma\bmb}=D^1_\bma\cap(D^2_{-\bma}+n\bma-\bmb)$.
\end{theorem}

Множество $G_{\bma\bma}=D_\bma\cap(D_{-\bma}+(n-1)\bma)$ мы для удобства обозначим как $G_{\bma}$.

\begin{proof}
Поскольку $P\cap (P+\bma)=P_\bma=P_{-\bma}+\bma$, то справедлива цепочка равенств
\begin{multline}\label{line}
nF_\bma=nK^1_\bma\cap (nK^2_{-\bma}+n\bma)=nK^1\cap(nK^2+n\bma)=\\
=(K^1+D^1)\cap(K^2+D^2+n\bma)=
\bigcup\limits_{d_1\in D^1,\ d_2\in D^2}(K^1+d_1)\cap (K^2+d_2+n\bma).
\end{multline}
Из соотношения $(K^1+d_1)\cap (K^2+d_2+n\bma)\neq\0$ следует, что $d_2-d_1+n\bma=\bmb$ для некоторого $\bmb\in A$.

Рассматривая $i$-ю координату в последнем равенстве, мы получаем
$$d_{2i}-d_{1i}+n\al_i=\be_i.$$ 
Из соотношений $\al_i,\be_i\in\{-1,0,1\}$ и $|d_{2i}-d_{1i}|\le n-1$ следует, что: \medskip
\begin{itemize}[nolistsep]
 \item[1.] если $\al_i=1$, то $\be_i>0$, а значит $\be_i=\al_i$ и $d_{2i}-d_{1i}=n-1$;
 \item[2.] если $\al_i=-1$ , то $\be_i<0$, поэтому $\be_i=\al_i$ и $d_{2i}-d_{1i}=1-n$;
 \item[3.] если $\al_i=0$ то $d_{2i}-d_{1i}=\be_i\in\{-1,0,1\}$.
\end{itemize} 
\medskip

Равенство $d_1=d_2+n\bma-\bmb$ показывает, что $d_1$ принадлежит множеству $D^1\cap(D^2+n\bma-\bmb)$, которое мы обозначим через $G_{\bma\bmb}$.
Каждое $d_1\in G_{\bma\bmb}$ задаёт единственное $d_2$ и для них справедливы равенства 
$$(K^1+d_1)\cap (K^2+d_2+n\bma)= (K^1\cap(K^2+\bmb))+d_1= F_\bmb+d_1,$$ 
из которых следует, что $nF_\bma= \bigcup\limits_{\bmb\sqsupseteq\bma} {F_\bmb+G_{\bma\bmb}}$.
\end{proof}

%\begin{theorem}\label{IFC}
%Семейство $\{F_\bma, \bma\in A_k\}$ пересечений $F_\bma =K_{1}\cap (K_{2}+\bma)$ удовлетворяет системе $\Sa$ уравнеий
% 
%\begin{equation}\label{perall}
%F_\bma=\bigcup\limits_{\bmb\sqsupseteq{\bma}}T_{\bma\bmb}(F_\bmb),\qquad \bma\in A_k,
%\end{equation}
% 
%где для каждого $\bmb\sqsupseteq\bma$, 
%\begin{equation}\label{Gab}
%T_{\bma\bmb}(F_\bmb)=\frac{1}{n}(F_\bmb+G_{\bma\bmb})\mbox{\quad  \text{и} \quad}
%  G_{\bma\bmb}=D_1\cap(D_2+n\bma-\bmb)
%\end{equation}
%\end{theorem}

%\begin{proof}
%Представим $F_\bma$ как $K_1\cap (K_2+\bma)=\dfrac{1}{n}\bigl((K_1+D_1)\cap (K_2+D_2+n\bma)\bigr).$ \\
%
%Пусть $d_1\in D_1$ и $d_2\in D_2$, тогда пересечение $(K_{1}+d_1)\cap (K_{2}+d_2+n\bma)$ непусто если $(P+d_1)\cap (P+d_2+n\bma)\neq\0$, что означает, что вектор $\bmb=d_2-d_1+n\bma$ лежит в множестве $A$. 
%Поскольку для любого номера координаты $i=1,\ldots  ,k$ мы имеем $|(d_2-d_1)_i|\le n-1$, это возможно только если $\bmb\sqsupseteq \bma$.\\
%
%Если $\bmb=\bma$, то $d_1=d_2+(n-1)\bma$, следовательно $d_1\in D_1\cap(D_2+(n-1)\bma)= G_\bma$.
%
%Если $\bmb\sqsupset\bma$, то $(K_{1}+d_1)\cap (K_{2}+d_2+n\bma)=(K_1\cap (K_2+\bmb))+d_1$, и $d_1\in D_1\cap(D_2+n\bma-\bmb)$. Множество $D_1\cap(D_2+n\bma-\bmb)$ мы обозначим как  $G_{\bma\bmb}$. 
%
%Заметим, что  $G_{\bma\bma}=D_1\cap(D_2+n\bma-\bma)$, а значит $G_{\bma\bma}=G_{\bma}$.
%
%В результате мы получаем  $F_\bma=\frac{1}{n}\bigcup\limits_{\bmb\sqsupseteq{\bma}}(F_\bmb+G_{\bma\bmb})=
%\frac{1}{n}(F_\bma+G_\bma)\cup\bigcup\limits_{\bmb\sqsupset{\bma}}\frac{1}{n}(F_\bmb+G_{\bma\bmb})$.
%\end{proof}

Отношения между множествами $F_\bma$, установленные в Теореме \ref{IFC} приводят к {\em структурному графу $\Ga_\Sa$} системы $\Sa$. 
%С помощью этой системы и графа можно найти важные свойства множеств $F_\bma$.

\begin{definition}
Структурный граф $\Ga_\Sa$ является ориентированным графом, в котором вершинам соответсвуют множества $F_\bma\in\Sigma$ и для каждого $\bma\sqsubseteq\bmb$ существует ребро $(F_\bma,F_\bmb)$, направленное из $F_\bma$ в $F_\bmb$ и соответствующее множеству единиц $G_{\bma\bmb}$.
\end{definition}

\begin{figure}[h!]
    \centering
    \includegraphics[width=.6\textwidth]{structure_grapg_full.pdf}
    \caption{Структурный граф $\Ga_\Sa$ в общем случае для пересечения двух фрактальных квадратов}
\end{figure}

В общем случае граф $\Ga_\Sa$ будет содержать $3^k$ вершин и $5^k$ ребер, при этом $3^k$ ребер являются петлями.\\

Структурный граф $\Ga_\Sa$ можно упростить без потери информации об отношениях между множествами $F_\bma$.
Для этого требуется удалить из графа вершины, которым соответсвуют пустые $F_\bma$, а также удалить те рёбра $(F_\bma,F_\bmb)$ при $\bmb\sqsupseteq\bma$, для которых $G_{\bma\bmb}=\0$ или $F_\bmb=\0$.
%После такого упрощения структурный граф для системы $\Sa$, определенной в теореме \ref{IFC}, имеет множество вершин $V_\Sa=\{F_\bma: \bma\in A, F_\bma\neq\0\}$ и множество ребер $E_\Sa=\{(F_\bma, F_\bmb): \bma\sqsupseteqeq\bmb, G_{\bma\bmb}\neq\0, F_\bmb\neq\0\}$.
После такого упрощения структурный граф $\Ga_\Sa$ может быть несвязен.
В таком случае интерес представляет та связная компонента графа, которая содержит вершину $F_0$.\\

Множество $F_\bma$ пусто, если $G_\bma=\0$ и для любого $\bmb\sqsupset\bma$ множество $F_\bmb+G_{\bma\bmb}=\0$.
Это позволяет вывести условие того, что $F_\bma$ пусто.

\begin{lemma}
Множество $F_\bma=\0$ тогда и только тогда, когда для любого $\bmb\sqsupseteq\bma$ и для любой конечной последовательности\\ $\bma=\bma_0\sqsubseteq\bma_1\sqsubseteq\ldots \bma_{p-1}\sqsubseteq\bma_p=\bmb$ произведение 
$\#G_{\bma_0\bma_1}\#G_{\bma_1\bma_2}\ldots  \#G_{\bma_{p-1}\bma_p}\#G_{\bmb}$ равно нулю. 
\qed
\end{lemma} 

\begin{definition}
Мы говорим, что в графе $\Ga_\Sa$  есть {\em направленный путь от $F_\bma$ до $F_\bmb$}, если 
\begin{itemize}[nolistsep]
\item[1.] $\bma\sqsubset\bmb$ и существует последовательность $\bma=\bma_0\sqsubset\bma_1\sqsubset\ldots\sqsubset\bma_{p-1}\sqsubset\bma_p=\bmb$ такая, что для любых  $j=0, \ldots, p$ множества $F_{\bma_j}\neq\0$  и множества $G_{\bma_{j-1}\bma_j}\neq\0$ для $j=1, \ldots, p$; либо
\item[2.] $\bma=\bmb$ и множество $G_{\bma\bmb}\neq\0$.
\end{itemize}
Если в $\Ga_\Sa$ есть направленный путь от $F_\bma$ до $F_\bmb$, то мы пишем $\bmb\succcurlyeq\bma$.
Если при этом $\bma\neq\bmb$, то $\bmb\succ\bma$.
\end{definition}

%Если $\bmb\succcurlyeq\bma$ или $\bma\succcurlyeq\bmb$, то мы говорим, что $\bma$ и $\bmb$ являются $\Ga${\em-сравнимы}.
%
%Мы обозначим через $\Ga_\bma$ подграф в $\Ga$, все вершины которого являются $F_\bmb$ такими, что $\bmb\succcurlyeq\bma$. 
%Мы говорим, что $\bmb$ является {\em максимальным} для $\Ga_\bma$, если $\Ga_\bmb$ является единственной вершиной $F_\bmb$.
%Мы говорим, что $\bmb$ является {\em минимальным} для $\Ga_\Sa$, если нет $\bma$ такого, что $\bma\prec\bmb$.
%
%Стоит отметить, что, согласно теореме \ref{IFC}, граф $\Ga_\bma$ показывает множество всех уравнений, которые полностью определяют каждое из множеств $F_\bmb$, для которых $\bmb\succcurlyeq\bma$.\\


%\subsection{Мощность пересечений копий фрактального квадрата}
%
%Рассмотрим фрактальный квадрат $K=\dfrac{K+D}{n}$ и пару таких копий $\dfrac{K+d_1}{n}$ и $\dfrac{K+d_2}{n}$, что $d_2=d_1+\bma$, где $\bma\in A\mmm\{0\}$.
%Назовём такую пару копий {\em копиями с $\bma$-соседством}.
%Тогда $\dfrac{K+d_1}{n}\cap\dfrac{K+d_2}{n}=\dfrac{d_1+F_\bma}{n}$, значит мощность пересечения копий с $\bma$-соседством совпадает с мощностью множества $F_\bma$.
%Уравнение \eqref{sideint} в Теореме \ref{IFC} позволяет нам оценить мощность множества $F_\bma$.
%
%\begin{theorem}\label{fin_int}
%Пусть $K=\dfrac{K+D}{n}$ --- фрактальный квадрат. Рассмотрим $F_\bma, \bma\in A\mmm\{0\}$.
%\begin{itemize}[nolistsep]
% \item[(i)] Если $\#G_{\bma\bma}> 1$, то множество $F_\bma$ несчётно.
% \item[(ii)] Если $\#G_{\bma\bma}=1$ и существует $\bmb\sqsupset\bma$ такое, что  $F_\bmb$ непусто и $\#G_{\bma\bmb}\geq1$, то $F_\bma$ бесконечное счётное.
% \item[(iii)] Множество $F_\bma$ конечно в следующих случаях:
% \begin{itemize}[nolistsep]
% \item[\textbf{(a)}] $\#G_{\bma\bma}=1$ и $\#F_\bmb\cdot\#G_{\bma\bmb}=0$ для каждого $\bmb\sqsupset\bma$;
% \item[\textbf{(b)}] $\#G_{\bma\bma}=0$ и существует $\bmb\sqsupset\bma$ такое, что $\#F_\bmb\cdot\#G_{\bma\bmb}\geq1$.
% \end{itemize}
%\end{itemize} 
%\end{theorem}
%
%\begin{proof}
%При $\bma\in A\mmm\{0\}$ множество $F_\bma$ удовлетворяет уравнению
%$F_\bma=\bigcup\limits_{\bmb\sqsupseteq\bma} \dfrac{F_\bmb+G_{\bma\bmb}}{n}$, 
%где $G_{\bma\bmb}=D_\bma\cap(D_{-\bma}+n\bma-\bmb)$.\\
%Если $\bma\in\{(1,0),\ (-1,0),\ (0,1),\ (0,-1)\}$, то $\#\{\bmb:\bmb\sqsupset\bma\}=2$, при этом $\bmb\in\{(1,1),\ (-1,1),\ (1,-1),\ (-1,-1)\}$.\\
%Если $\bma\in\{(1,1),\ (-1,1),\ (1,-1),\ (-1,-1)\}$, то $\#\{\bmb:\bmb\sqsupset\bma\}=0$.\\
%
%Для $\bma\in\{(1,1),\ (-1,1),\ (1,-1),\ (-1,-1)\}$, то $\#G_{\bma\bmb}\leq1$ множество $F_\bma$ не более чем одноточечно.\\
%
%Далее предполагаем, что $\bma\in\{(1,0), (-1,0), (0,1), (0,-1)\}$.\\
%
%(i) Если $\#G_{\bma\bma}> 1$, то $F_\bma$ содержит в себе фрактальный квадрат с множеством единиц $G_{\bma\bma}$, который имеет мощность континуума.\\
%
%(ii) Рассмотрим случай, когда $\#G_{\bma\bma}=1$ и существует $\bmb\sqsupset\bma$ такое, что  $F_\bmb$ непусто и $\#G_{\bma\bmb}\geq1$. 
%Такое непустое $F_\bmb$ единственно, поскольку иначе $\#G_{\bma\bma}>1$.
%Множество $\dfrac{F_\bmb+G_{\bma\bmb}}{n}$ конечно, поэтому $F_\bma=\bigcup\limits_{n=1}^\8 S_\bma^n$.\\
%
%(iii.{\bf a}) Пусть $G_{\bma\bma}=\{d_i\}$ и $\#F_\bmb\cdot\#G_{\bma\bmb}=0$ для каждого $\bmb\sqsupset\bma$.
%В этом случае из равенства $F_\bma=\dfrac{F_\bma+\{d_i\}}{n}$следует $F_\bma=\left\{\dfrac{d_i}{n-1}\right\}$.\\
%
%(iii.{\bf b}) Наконец, рассмотрим случай, когда $G_{\bma\bma}=\0$ и существует $\bmb\sqsupset\bma$ такое, что $\#F_\bmb\cdot\#G_{\bma\bmb}\geq1$.
%При $G_{\bma\bma}=\0$ может существовать не более одного $\bmb\sqsupset\bma$ такого, что $F_\bmb\neq\0$, в противном случае $\#G_{\bma\bma}\geq2$.
%Так как $\#F_\bmb=1$, множество $F_\bma=\dfrac{F_\bmb+G_{\bma\bmb}}{n}$ конечно, а говоря точнее, $\#F_\bma=\#G_{\bma\bmb}$.
%\end{proof}
%
%Далее докажем, что фрактальный квадрат, являющийся дендритом, обладает свойством одноточечного пересечения.
%Поэтому укажем условия, при которых $F_\bma$ одноточечно:
%
%\begin{corollary}\label{onepoint} 
%Множество $F_\bma$ одноточечно, если \\
%\textbf{(a)} $\#G_{\bma\bma}=1$ и $\#F_\bmb\cdot\#G_{\bma\bmb}=0$ для каждого $\bmb\sqsupset\bma$; или\\
%\textbf{(b)} $\#G_{\bma\bma}=0$ и существует $\bmb\sqsupset\bma$ такое, что $\#F_\bmb\cdot\#G_{\bma\bmb}=1$.
%\hfill\qed
%\end{corollary}

\subsection{Мощность множества $F_\al$}

Рассмотрим фрактальные $k$-кубы $K^1=\dfrac{K^1+D^1}{n}$ и  $K^2=\dfrac{K^2+D^2}{n}$, и пару таких копий $\dfrac{K^1+d_1}{n}\in K^1$ и $\dfrac{K^2+d_2}{n}\in K^2$, что $d_2=d_1+\bma$, где $\bma\in A$.
Назовём такую пару копий {\em копиями с $\bma$-соседством}.
Тогда $\dfrac{K^1+d_1}{n}\cap\dfrac{K^1+d_2}{n}=\dfrac{d_1+F_\bma}{n}$, значит мощность пересечения копий с $\bma$-соседством совпадает с мощностью множества $F_\bma$.
Уравнение \eqref{sideint} в Теореме \ref{IFC} позволяет нам оценить мощность множества $F_\bma$.

\begin{theorem}
Пусть $K^1=\dfrac{K^1+D^1}{n}$ и $K^2=\dfrac{K^2+D^2}{n}$ --- фрактальные $k$-кубы. 
Рассмотрим $F_\bma=K^1\cap(K^2+\bma), \bma\in A$.
\begin{enumerate}[nolistsep]
\item Множество $F_\bma$ несчетно, если существует $\bmb\succcurlyeq\bma$ такое, что $\#G_\bmb>1$;

\item Множество $F_\bma$ счетно, если для любого $\bmb\succcurlyeq\bma$ верно неравенство $\#G_\bmb\leq1$;

\item Множество $F_\bma$ конечно, если для всех максимальных вершин $\bmb$ в $\Gamma_\bma$, верно $\#G_\bmb=1$ и $G_{\bmb}=\0$ для всех остальных вершин в $\Gamma_\bma$.
В этом случае $\#F_\bma$ равно сумме всех композиций $\prod \limits_{j=1}^{p-1}\# G_{\bma_j\bma_{j+1}}$, взятых по всем цепочкам $\bma=\bma_1\prec\ldots\prec\bma_p=\bmb$, где $\bmb$ является максимальным в $\Gamma_\bma$;

\item Множество $F_\bma$ является одноточечным, если все $\bma_i\succcurlyeq\bma$ образуют цепочку $\bma=\bma_1\prec\ldots\prec\bma_p$, в которой для всех $j\le p-1$, $\# G_{\bma_j\bma_{j+1}}=1$, $G_{\bma_j}=\0$ и $\#G_{\bma_p}=1$;
\end{enumerate}
\end{theorem}


\begin{proof}

\end{proof}


\begin{example}\label{ex:3_1}
[Пересечение двух фрактальных квадратов, состоящее из $24$ точек, и его стркутурный граф $\Ga_\Sa$.]

Рассмотрим пересечение фрактальных квадратов $K_1$ и $K_2$ порядка $6$ с множествами единиц $D_1$ и $D_2$.
На рисунке \ref{fig:FSI_6x6_DSK} слева мы представляем $D_1$ и $D_2$ множеством красных квадратов $T_1(P)=\dfrac{D_1+P}{6}$ и синих квадратов $T_2(P)=\dfrac{D_2+P}{6}$.
На этом же рисунке справа показаны фрактальные квадраты $K_1$ и $K_2$. \\

Большинство из множеств $G_\bma$, а именно, 
$G_0$, $G_{(1,0)}$, $G_{(-1,0)}$, $G_{(0,1)}$, $G_{(0,-1)}$, $G_{(1,1)}$, $G_{(1,-1)}$ и $G_{(-1,1)}$ являются пустыми, и только $G_{(-1,-1)}=\{(0,0)\}$. 
Следовательно, $F_{(1,1)}=F_{(1,-1)}=F_{(-1,1)}=\0$ и $F_{(-1,-1)}=\{(0,0)\}$.

\begin{figure}[H]
    \includegraphics[width=0.45\textwidth]{FSI_6x6_DS.pdf}
    \hfill
    \includegraphics[width=0.45\textwidth]{FSI_6x6_K.png}
    \caption{Множества $\dfrac{D_1+P}{6}$ и $\dfrac{D_2+P}{6}$ (слева) и фрактальные квадраты $K_1$ и $K_2$ (справа).}
    \label{fig:FSI_6x6_DSK}
\end{figure}

Множество $F_{(1,0)}$ пусто, поскольку $G_{(1,0)}$, $F_{(1,1)}$ и $F_{(1,-1)}$ пусты. 
По той же причине $F_{(0,1)}=\0$.

Тут $G_{(-1,0)(-1,-1)}=\{(0,2), (0,4)\}$ и $G_{0(-1,0)}=\{(1,2), (1,4),(2,3), (3,2), (3,4), (4,3)\}$.
Множества $G_{(0,-1)(-1,-1)}$ и $G_{0(0,-1)}$ получены их предыдущих двух.

Таким образом, после удаления пустых вершин и ребер
граф $\Ga_\Sa$ содержит четыре вершины  $F_0$, $F_{(-1,0)}$, $F_{(0,-1)}$, $F_{(-1,-1)}$ и четыре ребра, которым соответствуют $G_{(-1,0)(-1,-1)}$, $ G_{(0,-1)(-1,-1)}$, $G_{0(-1,0)}$ и $G_{0(0,-1)}$.

Вычисление с использованием формулы \eqref{perall} показывает, что
$\#F_{(-1,0)}=\#F_{(0,-1)}=2$ и $\#F_0=2 \#G_{0(-1,0)}+2\#G_{0(0,-1)}=24.$

\begin{figure}[H]
    \centering
    \includegraphics[width=.45\textwidth]{FSI_6x6_SG.pdf}
    \caption{Упрощенный структурный граф $\Ga_\Sa$ для примера \ref{ex:3_1}}
\end{figure} 
\end{example} 




\section{Фрактальный куб с одноточечным пересечением, являющийся дендритом}

\subsection{Свойство одноточечного пересечения}

Далее будут получены условия, при которых фрактальный $k$-куб с одноточечным пересечением является дендритом, поэтому укажем условия, при которых $F_\bma$ одноточечно:

\begin{corollary}\label{onepoint} 
Множество $F_\bma$ одноточечно, если \\
\textbf{(a)} $\#G_{\bma\bma}=1$ и $\#F_\bmb\cdot\#G_{\bma\bmb}=0$ для каждого $\bmb\sqsupset\bma$; или\\
\textbf{(b)} $\#G_{\bma\bma}=0$ и существует $\bmb\sqsupset\bma$ такое, что $\#F_\bmb\cdot\#G_{\bma\bmb}=1$.
\hfill\qed
\end{corollary}

\begin{corollary}
Фрактальный куб $K$ обладает свойством одноточечного пересечения, если структурный граф $\Gamma_\Sa$ представляет собой объединение цепочек $0\prec\bma_{i1}\prec\ldots\prec\bma_{ip_i}$, для которых все $\bma_{ij}$ различны и таковы, что для всех $i$ верно $\#G_{\bma_{ip_i}}=1$ и для всех $i,j$ таких, что $j\le p_i-1$, верно $\# G_{\bma_{ij}\bma_{i,j+1}}=1$ и $G_{\bma_{ij}}=\0$.
\end{corollary}

Рассмотрим фрактальные $k$-кубы с одноточечным пересечением и проверим, является ли рассматриваемое множество дендритом.

\begin{theorem}
Пусть $K=\dfrac{K+D}{n}$ фрактальный куб с одноточечным пересечением. 

Если существуют $\bma,\bmb\succ0$ такие, что $F_\bma=F_\bmb\neq\0$,  то для каждой тройки $d, d+\bma, d+\bmb\in D$ копии $K_{(d)},$ $K_{(d+\bma)}$ и $K_{(d+\bmb)}$ пересекаются по общей точке $x=\dfrac{F_\bma+d}{n}=\dfrac{F_\bmb+d}{n}$.
\end{theorem}

\begin{proof}
It is clear that $K_{(d)}\cap K_{(d+\bma)}=\dfrac{F_\bma+d}{n}$ и $K_{(d)}\cap K_{(d+\bmb)}=\dfrac{F_\bmb+d}{n}$.
The equality $F_\bma=F_\bmb,$ implies 
$$K_{(d)}\cap K_{(d+\bma)}=K_{(d)}\cap K_{(d+\bmb)}=\dfrac{F_\bmb+d}{n}=\dfrac{F_\bma+d}{n}=\{x\},$$
therefore $K_{(d)}\cap K_{(d+\bma)}\cap K_{(d+\bmb)}=\{x\}.$

Therefore, the white vertices corresponding to $K_{(d)}, K_{(d+\bma)}, K_{(d+\bmb)}$ are connected to the same single black vertex $p$ corresponding to the point $x$.

\begin{figure}[H]
    \centering
    \includegraphics[width=.75\textwidth]{FSD_ord3_SSB.pdf}
    \caption{A triple of copies with unique intersection point and a triple of copies forming a cycle.}
\end{figure}

On the other hand, if $d, d-\bma, d-\bmb\in D$ then the intersection points are different. 
$$  K_{(d)}\cap K_{(d-\bma)}=\dfrac{F_\bma+(d-\bma)}{n};\;
    K_{(d)}\cap K_{(d-\bmb)}=\dfrac{F_\bma+(d-\bmb)}{n};$$
$$  K_{(d-\bma)}\cap K_{(d-\bmb)}=\dfrac{F_{\bma-\bmb}+(d-\bma)}{n}=\dfrac{F_{\bmb-\bma}+(d-\bmb)}{n}$$

Therefore, the white vertices corresponding to $K_{(d)}, K_{(d+\bma)}, K_{(d+\bmb)}$ form a cycle consisting of 3 white and 3 black points in the graph $\Gamma$.
\end{proof}

\begin{corollary}\label{mpoint}
Let $ B=\{d_1,...,d_m\}$ be a subset of $D$ which satisfies the condition:
For any $d_i,d_j,d_k\in B$, $$(d_j-d_i),(d_k-d_i)\in A\setminus0\text{ and } F_{d_j-d_i}=F_{d_k-d_i}\neq\0.$$
Then there is a point $x\in K$ such that for any $d_i,d_j\in B$, $K_{(d_i)}\cap K_{(d_j)}=\{x\}$, and $x$ corresponds to a black vertex of order $m$ in $\Gamma$.
\qed
\end{corollary}

\begin{figure}[H]
    \centering
\includegraphics[width=.9\textwidth]{FQ_den.pdf}
    \caption{A fractal cube dendrite and its intersection graph.\cite{BMM2022}}
    \label{fig:enter-label}
\end{figure}

\subsection{Алгоритм проверки фрактального куба на свойство дендритности}

Пусть $K$ --- фрактальный $k$-куб.
Чтобы проверить $K$ на наличие дендритности, нужно выполнить следующие шаги:

 \begin{enumerate}

    \item Найдём все множества $G_\bma, G_{\bma\bmb}$ для системы $\Sigma=\Sigma(K,K)$ и запишем систему $\Sa$.
    Согласно Определению \ref{strg} и Лемме \ref{red}, исключим все исчезающие вершины и ребра и построим граф $\Ga_\Sa$.
   
    \item Используя Следствие \ref{SIPQ}, проверим выполнение свойства одноточечного пересечения для $K$.
    Если это не удается, то $K$ не является дендритом.
    
    \item Построим двудольный граф пересечений для фрактального куба $K$, соединив рёбрами пересекающиеся копии с точкой пересечения этих копий. 
    Нужно учесть случай с кратными точками, упомянутыми в следствии \ref{mpoint}.
    Теперь если получившийся двудольный граф пересечений будет деревом, то $K$ --- дендрит.    
\end{enumerate}


\begin{example}[Фрактальный куб с одноточечным пересечением]
Возьмем фрактальный куб $K=\dfrac{K+D}{4}$ с множеством единиц 
\begin{equation*}
\begin{split}
D=\{
    &(0,0,0), (1,1,1), (2,2,2), (3,3,3), (2,1,1), (1,2,1), (1,1,2),\\ 
    & (1,2,2),(2,2,1), (2,1,2), (0,0,2), (0,2,1), (3,3,1), (3,1,1),\\ 
    & (2,0,0), (1,2,0),(1,3,3), (1,1,3)\}     
\end{split}
\end{equation*}
 
\begin{figure}[H]
    \centering
    \includegraphics[width=0.45\textwidth]{fqP1a.png}
    \hfill
    \includegraphics[width=0.45\textwidth]{fqK1a.png}
    \caption{Фрактальный куб с одноточечным пересечением}
    \label{fig:fq}
\end{figure}

\begin{figure}[H]
    \centering
    \includegraphics[width=\textwidth]{SG_for_FQ.pdf}
    \caption{Структурный граф}
    \label{fig:fq_sg}
\end{figure}
\end{example}