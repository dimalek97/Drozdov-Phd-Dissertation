% \newpage
\chapter*{Заключение}                       % Заголовок
\addcontentsline{toc}{chapter}{Заключение}  % Добавляем его в оглавление

%% Согласно ГОСТ Р 7.0.11-2011:
%% 5.3.3 В заключении диссертации излагают итоги выполненного исследования, рекомендации, перспективы дальнейшей разработки темы.
%% 9.2.3 В заключении автореферата диссертации излагают итоги данного исследования, рекомендации и перспективы дальнейшей разработки темы.
%% Поэтому имеет смысл сделать эту часть общей и загрузить из одного файла в автореферат и в диссертацию:

Основные результаты работы заключаются в следующем.

\begin{enumerate}
\item Найдено необходимое условие (условие совпадения параметров) того, что аттрактор обобщённой полигональной системы является дендритом (теорема \ref{PMT}). 

\item Доказано, что при достаточно малом $\delta=\delta(\eS)>0$ аттрактор любой (удовлетворяющей условию совпадения параметров) $\delta$-деформации $\eS'$  полигональной системы $\eS$ является дендритом, изоморфным аттрактору системы $\eS$ (теорема \ref{mainthm}).

\item Оценена мощность пересечения копий фрактальных $k$-кубов. Такое пересечение может быть одноточечным, конечным, счётным или несчётным.

\item Получен алгоритм, позволяющий определить фрактальные $k$-кубы с одноточечным пересечением, являющиеся дендритмами.

\item Было доказано, что фрактальные квадраты, являющиеся дендритами, обладют свойством одноточечного пересечения (Теорема \ref{thm:den_necessary_sufficient}).
Из этого следует, что фрактальный квадрат является дендритом тогда и только тогда, когда его двудольный граф пересечений является деревом (следствие \ref{cor:fsden}).

\item Было доказано, что у фрактальных квадратов, являющихся денритами, всего семь возможных топологических типов главного дерева (теорема \ref{thm:7trees}).
Это позволяет классифицировать фрактальные квадраты, являющиеся дендритами, по типу их главного дерева.
Для каждого класса были построены примеры фрактальных квадратов (со всеми возможными типами самоподобной границы) и их главных деревьев.
\end{enumerate}

Все результаты работы, полученные в научно-квалификационной работе, являются новыми.
% и вносят вклад в исследования фрактальных квадратов и самоподобных дендритов.
\\

\textbf{Благодарности.}
Я выражаю благодарность и большую признательность своему научному руководителю Тетенову Андрею Викторовичу за неизменную поддержку, помощь, обсуждение результатов и научное руководство.
Я благодарен Дмитрию Юрьевичу Мехонцеву за консультации по работе в IFStile, а также Медных Александру Дмитриевичу и Асееву Владиславу Васильевичу за их поддержку и советы.