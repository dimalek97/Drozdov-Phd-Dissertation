%\newpage
%
%
%\bibliographystyle {gost2008s}
%\bibliography {dissbib}
%\printbibliography
% \nocite {*} % добавляет непроцитированные источники из bib-файла


\addcontentsline{toc}{chapter}{Список использованных источников и литературы} % Добавляем раздел в содержание

%\bibliographystyle {gost2008s}
%\bibliography {dissbib}
%\nocite {*}
%\nocite{*}
% \printbibliography[
%                     title = {Список использованных источников и литературы}]

\begin{thebibliography}{99}

\bibitem{AST2024}
{\bf Allabergenova K., Samuel M., Tetenov A.}, 
Intersections of the pieces of self-similar dendrites in the plane //
Chaos, Solitons \& Fractals. 2024. Т. 182. С. 114805.

\bibitem{SSS7}
{\bf Bandt C., Graf S.},
Self-Similar Sets 7. A Characterization of Self-Similar Fractals with Positive Hausdorff Measure // 
Proceedings of the American Mathematical Society. 1992. Т. 114. № 4. С. 995--1001.

\bibitem{SSS2}
{\bf Bandt C., Keller K.},
Self‐Similar Sets 2. A Simple Approach to the Topological Structure of Fractals // 
Mathematische Nachrichten. 1991. Т. 154. № 1. С. 27--39.

\bibitem{BR}  {\bf Bandt C. , Rao H.},
Topology and separation of self-similar fractals in the plane // 
Nonlinearity. 2007. Т. 20. № 6. С. 1463--1474.

\bibitem{SSS6}
{\bf Bandt C., Stahnke J.},
Self-similar sets 6. Interior distance on deterministic fractals //
preprint, 1990

\bibitem{SF}
{\bf Barnsley~M.F., Hutchinson~J.E., Stenflo~\"O.},
$V$-variable fractals: Fractals with partial self similarity //
Advances in Mathematics. 2008. Т. 218. № 6. С. 2051--2088.

\bibitem{Bedford1984}
{\bf Bedford T.},
Crinkly curves, Markov partitions and dimension: Phd Thesis / 
University of Warwick, 1984.

\bibitem{BM}
{\bf Bonk M., Merenkov S.}, 
Quasisymmetric rigidity of square Sierpiński carpets // 
Annals of Mathematics. 2013. Т. 177. № 2. С. 591--643.

\bibitem{Ces}
{\bf Cesaro E.}, 
Remarques sur la courbe de von Koch // 
Atti della R. Accad. della Scienze fisiche e matem. Napoli. 1905. T. 12. № 15. С. 1--12.

\bibitem{Char1998}
{\bf Charatonik J. J., Charatonik W. J.},
Dendrites // 
Aportaciones Mat. Comun. 1998. Т. 22. С. 227--253.

\bibitem{CS1}
{\bf Cristea L. L., Steinsky B.},
Curves of infinite length in $4\times4$-labyrinth fractals // 
Geometriae Dedicata. 2008. Т. 141. № 1. С. 1--17.


\bibitem{CS2}
{\bf Cristea L. L., Steinsky B.},
Curves of infinite length in labyrinth fractals // 
Proceedings of the Edinburgh Mathematical Society. 2011. Т. 54. № 2. С. 329--344.


\bibitem{CS3}
{\bf Cristea L. L., Steinsky B.},
Mixed labyrinth fractals // 
Topology and its Applications. 2017. Т. 229. С. 112--125.

\bibitem{EKM2009}
{\bf Elekes M., Keleti T., Máthé A.}, 
Self-similar and self-affine sets: measure of the intersection of two copies // Ergodic Theory and Dynamical Systems. 2009. Т. 30. № 2. С. 399--440.

\bibitem{Falconer2004}
{\bf Falconer K. J.},
Fractal geometry: mathematical foundations and applications / 
K. J. Falconer. --- 3rd ed. --- New York: J.~Wiley and Sons., 2014. --- 398 p.

\bibitem{Fraser_2021}
{\bf Fraser J. M.},
Fractal Geometry of Bedford-McMullen Carpets // 
Thermodynamic Formalism / под ред. M. Pollicott, S. Vaienti. : Springer International Publishing, 2021. С. 495--516.

\bibitem{Hata1985}
{\bf Hata M.},
On the structure of self-similar sets // 
Japan Journal of Applied Mathematics. 1985. Т. 2. № 2. С. 381--414.

\bibitem{Hut1981}
{\bf Hutchinson J.},
Fractals and Self-Similarity // 
Indiana University Mathematics Journal. 1981. Т. 30. № 5. С. 713--747.

\bibitem{Jana2017}
{\bf Jana A., García R. E.}, Lithium dendrite growth mechanisms in liquid electrolytes // Nano Energy. 2017. Т. 41. С. 552--565.

\bibitem{TF}
{\bf Kamalutdinov K. G., Tetenov A. V.},
Twofold Cantor sets in $\rr$ // 
Sibirskie Elektronnye Matematicheskie Izvestiya. 2018. Т. 15. С. 801--814.

\bibitem{KenyonPeres1996}
{\bf Kenyon R., Peres Y.}, 
Measures of full dimension on affine-invariant sets // 
Ergodic Theory and Dynamical Systems. 1996. Т. 16. № 2. С. 307--323.

\bibitem{Kig}
{\bf Kigami~J.}, 
Analysis on fractals / Cambridge Tracts in Mathematics 143 : 
Cambridge University Press. 2001. 251 С.

\bibitem{Kig95}  
{\bf Kigami J.}, 
Harmonic Calculus on Limits of Networks and Its Application to Dendrites // 
Journal of Functional Analysis. 1995. Т. 128. № 1. С. 48--86.

\bibitem{Koch}
{\bf H.~von~Koch},  
Sur une courbe continue sans tangente, obtenue par
une construction geometrique elementaire // 
Archiv for Matemat., Astron. och Fys., 1904, Т.~1, С.~681--702.

\bibitem{Kur1} 
{\bf Kuratowski, K.}, 
Topology : vol. 1 / 
K. Kuratowski. --- Academic Press. 1966. --- 580 p.

\bibitem{Kur2} 
{\bf Kuratowski, K.},
Topology : vol. 2 / 
K. Kuratowski. --- Academic Press. 1968. --- 608 p.

\bibitem{LLR2013}
{\bf Lau K.-S., Luo J. J., Rao H.},
Topological structure of fractal squares // 
Mathematical Proceedings of the Cambridge Philosophical Society. 2013. Т. 155. № 1. С. 73--86.

\bibitem{Levy1939}
{\bf L{\'e}vy, P.}, 
Les courbes planes ou gauches et les surface compos{\'e}es de parties semblables au tout // 
J. I’Ecole Poly. 1939. Т. 144. С. 227--292.

\bibitem{LL}
{\bf Luo~J.J., Liu~J.-C.},
On the classification of fractal squares
// Fractals. 2016. Т. 24. № 01. Art. №~1650008.

\bibitem{Man75} 
{\bf Mandelbrot B.},
Les objets fractals: forme hasard et dimension // 
Paris. Flammarion. 1975.

\bibitem{MW1988}
{\bf Mauldin R. D., Williams S. C.},
Hausdorff dimension in graph directed constructions // 
Transactions of the American Mathematical Society. 1988. Т. 309. № 2. С. 811--829.

\bibitem{McMullen1984}
{\bf McMullen C.},
The Hausdorff dimension of general Sierpiński carpets // 
Nagoya Mathematical Journal. 1984. Т. 96. С. 1–9.

\bibitem{McWorter1987}
{\bf McWorter Jr. W. A., Tazelaar J. M.},
Creating fractals // 
Byte. 1987. Т. 12. № 9. С. 123--132.

\bibitem{IFStile} 
{\bf Mekhontsev D.}, 
IFStile (2016-2024) / D. Mekhontsev // [Программное обеспечение] : 
\href{https://ifstile.com/}{URL:https://ifstile.com/}

\bibitem{Moran1996}
{\bf Moran M.},
Hausdorff measure of infinitely generated self-similar sets // 
Monatshefte für Mathematik. 1996. Т. 122. № 4. С. 387--399.

\bibitem{Moran1946}
{\bf Moran P. A. P.},
Additive functions of intervals and Hausdorff measure // 
Mathematical Proceedings of the Cambridge Philosophical Society. 1946. Т. 42. № 1. С. 15–23.

\bibitem{Olsen1998}
{\bf Olsen L.}, 
Self-affine multifractal Sierpinski sponges in $\mathbb{R}^d$ // 
Pacific Journal of Mathematics. 1998. Т. 183. № 1. С. 143--199.

\bibitem{Peres1994}
{\bf Peres Y.},
The self-affine carpets of McMullen and Bedford have infinite Hausdorff measure // 
Mathematical Proceedings of the Cambridge Philosophical Society. 1994. Т. 116. № 3. С. 513--526.

\bibitem{Pot2017}
{\bf Potapov A., Potapov V.},
Fractal radioelement’s, devices and systems for radar and future telecommunications: Antennas, capacitor, memristor, smart 2d frequency-selective surfaces, labyrinths and other fractal metamaterials // Journal of International Scientific Publications: Materials, Methods \& Technologies. 2017. Т. 11. С. 492--512.

\bibitem{RW} 
{\bf Ruan~H.J., Wang~Y.},
Topological invariants and Lipschitz equivalence of fractal squares //
J. Math. Anal. Appl.  2017. Т.~451. С.~327--344.

\bibitem{STV2017}
{\bf Samuel M., Tetenov A. V., Vaulin D. A.},
Self-similar dendrites generated by polygonal systems in the plane //
Siberian Electronic Mathematical Reports. 2017. Т. 14. С. 737--751.

\bibitem{Schief1994} 
{\bf Schief~A.},  Separation properties for self-similar sets // 
Proc. Amer. Math. Soc. 1994. Т. 112. №.~1. С.~111--115.

\bibitem{strich1999}
{\bf Strichartz R.},
Analysis on fractals // 
Notices AMS. 1999. Т. 46. № 10. С. 1199--1208.

\bibitem{Strichartz1999}
{\bf Strichartz R.},
Isoperimetric Estimates on Sierpinski Gasket Type Fractals // 
Transactions of the American Mathematical Society. 1999. Т. 351. № 5. С. 1705--1752.

\bibitem{FIP}
{\bf Tetenov A.},
Finiteness properties for self-similar continua // 
arXiv:2003.04202v2 [math.MG]. 2021.

\bibitem{Xiao2021}
{\bf Xiao J.-C.},
Fractal squares with finitely many connected components // 
Nonlinearity. 2021. Т. 34. № 4. С. 1817--1836.

\bibitem{Zerner1996} 
{\bf Zerner M. P. W.},
Weak separation properties for self-similar sets // 
Proc. Amer. Math. Soc.  -- 1996. -- vol.~124, no.~11. -- pp.~3529--3539.

\bibitem{ATK}  
{\bf Асеев~В.~В.} 
О самоподобных жордановых кривых на плоскости / 
В.~В.~Асеев, А.~В~Тетенов., А.~С.~Кравченко //
Сиб. мат. журн., 2003, Т. 44, № 3, С.~481 -- 492.

\bibitem{2} 
{\bf Асеев~В.~В.},
О жордановых самоподобных дугах, допускающих структурную параметризацию /
В.~В.~Асеев, А.~В.~Тетенов //
Сиб. матем. журн. (2005), том 46, № 4, С.~733 -- 748.

\bibitem{Zaitseva2022}{\bf Зайцева Т. И., Протасов В. Ю.},
Самоподобные 2-аттракторы и тайлы // Математический сборник. 2022. Т. 213. № 6. С. 71--110.

\bibitem{Pot2005}{\bf Потапов А. А.}, 
Фракталы в радиофизике и радиолокации: Топология выборки. 
2-е изд., перераб. и доп. --- М.: Университетская книга, 2005. --- 848 c.

\bibitem{Pot2023}{\bf Потапов А.А., Кузнецов В.А.},
Способ комплексирования многочастотных радиолокационных изображений на основе фрактального подхода. --- В книге: Радиолокация. Теория и практика / Под ред. А.Б. Бляхмана. --- Москва: ЮНИТИ-ДАНА, 2023. --- С. 108--115. 

\bibitem{Pot2024}{\bf Потапов А.А.}, 
Расширение исследований по дальнейшей разработке и совершенствованию высокоэффективных текстурно-фрактальных методов для радаров с синтезированной апертурой // 
Сб. тр. VII междунар. науч.-техн. форума “Современные технологии в науке и образовании - СТНО-2024”: В 10-и тт. / Под общ. ред. О. В. Миловзорова. --- Рязань: Рязан. гос. радиотехн. ун-т, 2024. Т. 1. С. 6--25.

\bibitem{TSV2017} 
{\bf Тетенов~А.~В.},  
О дендритах, заданных системами полиэдров и их точках ветвления /
А.~В.~Тетенов, М.~Самуэль, Д.~А.~Ваулин // 
Труды ИММ УРО РАН, 2017, т.23, № 4, С.~281 -- 291.

\bibitem{Tet06}
{\bf Тетенов~А.~В.},
Самоподобные жордановы дуги и граф-ориентированные системы подобий /
А.~В.~Тетенов //
Сиб. мат. журн. 2006. Т. 47, № 5, С.~1147 -– 1159.

\bibitem{Tet07}
{\bf Тетенов А.~В.}, 
Структурные теоремы в теории самоподобных фракталов : диссертация на соискание уч. степени доктора физико-математических наук / А.В. Тетенов -- Горно-Алтайск, 2010. -- 216~с. 

%\bibitem{FPS}
%{\bf Tetenov~A.}
%Finiteness properties for self-similar sets /
%A.~Tetenov //
%[Электронный ресурс] : URL: https://arxiv.org/abs/2003.04202v1

%\bibitem{deltadef}
%{\bf Drozdov D.}
%On $\delta$-deformations of polygonal dendrites /
%D.~Drozdov, M.~Samuel, A.~Tetenov //
%[Электронный ресурс] : URL: https://arxiv.org/abs/1812.00439v1
%
%\bibitem{TSV0}
%{\bf Samuel~M.}
% Self-similar dendrites generated by polygonal systems in the plane /
% M.~Samuel, A.~Tetenov, D.~Vaulin // 
%Sib. El. Math. Rep., № 14 (2017), C. 737 -- 751. 

% ссылки из адамовской статьи

%\bibitem{ATK} Aseev, V. V.; Tetenov, A. V.; Kravchenko, A. S. Self-similar Jordan curves on
%the plane.  Siberian Math. J. 44 (2003), no. 3, 379--386 MR1984698


%\bibitem {C} Croydon~D., Random fractal dendrites, Ph.D. thesis, St. Cross College, University of Oxford, Trinity, 2006.

%\bibitem{TMV}
%Tetenov A.~V., Mekhontsev D., Vauilin.~D: On weak separation property for plane dendrites. 	(to appear)

%\bibitem{TSV0}
%Samuel, M.; Tetenov, A.; Vaulin, D. Self-similar dendrites generated by polygonal systems in the plane. Sib. El. Math. Rep. 14 (2017), 737--751. MR3693741

%\bibitem{TSV}
%Tetenov A.~V., Samuel.~M, Mekhontsev D.:
%On Dendrites Generated by Symmetric 
%Polygonal Systems: The Case of Regular 
%Polygons. in:  Trends in Mathematics, Advances in Algebra and Analysis: International Conference on Advances in Mathematical Sciences, Vellore, India, December 2017 - Volume I,  Springer, 2019, pp.17- 25.

%\bibitem{TSM}
%Tetenov A.~V., Samuel.~M, Vauilin.~D:
%On dendrites generated by polyhedral systems and their ramification points.  Proc.  Krasovskii Inst. Math. Mech. UB RAS \textbf{23}(4), 281---291 (2017) DOI: 10.21538/0134-4889-2017-23-4-281-291


%\bibitem{MS}Samuel M.  On some problems in fractal geometry // Bharata Mata College Koch (Cochin Ernakulam), 2017

% ссылки про квадраты и кубы

%\bibitem{bib:Olsen1998}    
%    {\sc Olsen L.,}
%    {\em Self-affine multifractal Sierpinski sponges in $\rr^d$,}
%    {Pac. J. Math. \textbf{116} (1998), 143--199}
%
%\bibitem{bib:EKM2010}    
%    {\sc Elekes M., Keleti T., M\'ath\'e A.,}
%    {\em Self-similar and self-affine sets: measure of the intersection of two copies,}
%    {Ergod. Th. \& Dynam. Sys. \textbf{30} (2010), 399--440}
    
    
\bigskip \bigskip

\noindent {\bf \Large  Работы автора по теме диссертации}

\bigskip \bigskip

\bibitem{DST2021}
{\bf Drozdov D., Samuel M., Tetenov A.},
On deformation of polygonal dendrites preserving the intersection graph //
The Art of Discrete and Applied Mathematics. 2021. Т. 4. № 2. С. 1--21.

\bibitem{DST2022}
{\bf Drozdov D., Samuel M., Tetenov A.}, 
On $\delta$-deformations of Polygonal Dendrites // 
Topological Dynamics and Topological Data Analysis. : Springer Singapore, 2021. С. 147--164.

\bibitem{DT2024fqd}
{\bf Drozdov D. A., Tetenov A. V.}, On the dendrite property of fractal cubes // Advances in the Theory of Nonlinear Analysis and Its Application. 2024. Т. 8. № 1. С. 73--80.

\bibitem{TD2022fs}
{\bf Drozdov D., Tetenov A.}, 
On fractal squares possessing finite intersection property // 
Bulletin of National University of Uzbekistan: Mathematics and Natural Sciences. 2022. Т. 5. № 3. С. 164--181.

\bibitem{TD2023fs}
{\bf Drozdov D., Tetenov A.}, 
On the classification of fractal square dendrites // 
Advances in the Theory of Nonlinear Analysis and Its Application. 2023. Т. 7. № 3. С. 19--96.

\bibitem{VDT2020}{\bf Ваулин Д. А., Дроздов Д. А., Тетенов А. В.},
О связных компонентах фрактальных кубов // 
Труды Института математики и механики УрО РАН. 2020. Т. 26. № 2. С. 98--107.


\end{thebibliography}