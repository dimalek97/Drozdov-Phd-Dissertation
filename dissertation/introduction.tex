% \newpage
\chapter*{Введение}                         % Заголовок
\addcontentsline{toc}{chapter}{Введение}    % Добавляем его в оглавление

% \newcommand{\actuality}{}
% \newcommand{\progress}{}
% \newcommand{\aim}{{\textbf\aimTXT}}
% \newcommand{\tasks}{\textbf{\tasksTXT}}
% \newcommand{\novelty}{\textbf{\noveltyTXT}}
% \newcommand{\influence}{\textbf{\influenceTXT}}
% \newcommand{\methods}{\textbf{\methodsTXT}}
% \newcommand{\defpositions}{\textbf{\defpositionsTXT}}
% \newcommand{\reliability}{\textbf{\reliabilityTXT}}
% \newcommand{\probation}{\textbf{\probationTXT}}
% \newcommand{\contribution}{\textbf{\contributionTXT}}
% \newcommand{\publications}{\textbf{\publicationsTXT}}

% \input{common/characteristic} % Характеристика работы по структуре во введении и в автореферате не отличается (ГОСТ Р 7.0.11, пункты 5.3.1 и 9.2.1), потому её загружаем из одного и того же внешнего файла, предварительно задав форму выделения некоторым параметрам

\begin{center}
\textbf{История вопроса и основные направления.}
\end{center}
% история вопроса
{\bf Тут добавить историю фракталов до 1970 года}


На практике строго определить фрактал не так просто.
К.~Фальконер \cite{Fal} даёт несколько необязательных признаков, котрорым могут удовлетворять фракталы.
Так для того чтобы можно было назвать объект $A$ фракталом, он должен характеризоваться какими-либо из следующих свойств:

\begin{enumerate}
\item $A$ имеет тонкую структуру, т. е. содержит сложные структурные элементы на любых масштабах;
\item $A$ слишком неоднородно, чтобы описываться на традиционном геометрическом языке;
\item $A$ самоподобно в том или ином смысле, т. е. имеет повторяющуюся структуру в разных масштабах. Возможно, самоподобие приблизительное или статистическое.
\item Каким-то образом определенная <<фрактальная>> размерность множества $A$ превышает его топологическую размерность и зачастую является дробным числом;
\item $A$ можно построить через рекурсивные или итеративные схемы (что позволяет моделировать фракталы на компьютерах).
\end{enumerate}


Одним из крупнейших разделов фрактальной геоиетрии является теория самоподобных множеств.
На протяжении всей работы мы будем рассматривать именно самоподобные множества.
Хотя понятие самоподобия впервые появилось у П. Леви в 1939 году \cite{Levy1939} при описании его знаменитой кривой Леви, но наиболее существенный вклад в теорию самоподобных множеств внёс Дж. Хатчинсон в 1981 г. \cite{Hut}.
Он дал строгое определение самоподобного множества, состоящего из уменьшеных образов самого себя, и описал четкий математический подход к исследованию таких множеств. 
Работа Хатчинсона послужила толчком для множества дальнейших исследований, многие из которых расширяют и обобщают класс самоподобных множеств.

Прежде всего стоит отметить вклад Р. Молдина и С. Вильямса \cite{MW1981}, которые разработали концепцию граф-ориентированных систем подобий, аттракторром которых является уже система компактов, каждый из которых может состаять не только из своих копий, но и из копий других компактов системы.
% Данная концепция перекликается с идеей самоподобных множеств с конденсацией, введённых \red{???}.

В 1996 году М. Моран \cite{Moran1996} определил бесконечно порождённые самоподобные множества и рассмотрел их свойства.
Эти множества имеют свои особенности при вычислении размерности и в целом значительно расширяют класс стандартных самоподобных множеств.

Одним из важнейших вопросов теории самоподобных множеств является вопрос о вычислении их хаусдорфовой размерности.
Вычисление размерности самоподобных множеств напрямую связано со структурой пересечений их копий, что в свою очередь тесно связано с условиями отделимости порождающих эти фракталы систем сжимающих подобий, такими как условие открытого множества (OSC) и слабое условие отделимости (WSP).
Множества, не удовлетворяющие никаким стандартным условиям отделимости, могут быть весьма сложными.

%Прежде всего стоит отметить вклад \red{кого???}, описавших строгое условие отделимости (SSC).
%Мы говорим, что самоподобное множество удовлетворяет строгому условию отделимости, если его копии попарно друг с другом не пересекаются.
%Однако это условие слишком сильное и для большинства множеств оно не подходит.

П.~Моран в 1946 г. \cite{Moran} ввел условие открытого множества (OSC) для самоподобных множеств на прямой, а Дж.~Хатчинсон \cite{Hut} обобщил введенное Мораном условие открытого множества на системы сжимающих подобий в $\rr^n$ для любого натурального $n$.
Мы говорим, что система сжимающих подобий $\eS=\{S_1, \ldots, S_m\}$ удовлетворяет условию открытого множества, если существует открытое множество $O$ такое, что множества $\{O_i=S_i(O) | S_i\in\eS\}$ содержатся в $O$ и попарно друг с другом не пересекаются.
Размерность Хаусдорфа самоподобных множеств, удовлетворяющих OSC или SSC, равна размерности подобия, которая легко вычисляется.
Так для системы $\eS=\{S_1,\ldots,S_m\}$ сжимающих подобий в $\rr^n$ с коэффициентами подобия $r_1, \ldots, r_m$ выполнено условие открытого множества (или строгое условие отделимости), то хаусдорова размерность  аттрактора этой системы равна его размерности подобия $s$, которая является решением следующего уравнения: $$r_1^s+\ldots+r_m^s=1.$$ 

Условие открытого множества тоже подходит далеко не для всех примеров.
Есть примеры самоподобных множеств с копиями, пересекающимися по множеству со столь сложной структурой, что вопрос существования подходящего открытого множества не является очевидным.
Иногда подходящим открытым множеством является внутренность самого фрактала, что не является удобным для проверки. 
Порой подходящее открытое множество может иметь очень сложную стуктуру, например состоять из бесконечного числа непересекающихся компонент.

К.~Бандт и З.~Граф в 1992 году \cite{SSS7} искали алгебраический аналог для OSC и ввели алгебраическое условие, основанное на ассоциированном семействе подобий. 
Они показали, что оно эквивалентно условию положительности хаусдорфовой меры аттрактора в размерности $s$, где $s$ --- размерность подобия для данной системы подобий.
Вышеупомянутое алгебраическое условие заключается в том, что копии самоподобного множества попарно пересекаются по множеству нулевой меры Хаусдорфа в размерности $s$.

Условие открытого множества можно усилить следующим требованием: открытое множество $O$ и аттрактор $K$ системы $\eS$ имеют непустое пересечение. 
Так получается сильное условие открытого множества (SOSC).
Однако А.~Шиф \cite{Schi} в 1994 г. показал, что все три условия: SOSC, OSC и условие положительности меры Хаусдорфа в размерности подобия --- эквивалентны.

Условие открытого множетсва и слабое условие отделимости позволяют вычислять размерность Хаусдорфа с помощью размерности подобия, поскольку копии самоподобного множества пересекаются не слишком сильно или вовсе не пересекаются.
Но даже для тех случаев, когда копии фрактала пересекаются по множеству ненулевой меры, есть возможность вычислить размерность, если оно удовлетворяет  слабому условию отделимости (WSP), определённому М.~Цернером \cite{Zerner}.
Наиболее хорошо оно себя показывает в случаях, когда копии самоподобного множества пересекаются по своим подкопиям.

Тем не менее, для систем, не удовлетворяющих ни OSC, ни WSP, вычисление размерности Хаусдорфа их аттракторов становится действительно сложной проблемой.\\

{\bf Тут написать про дендриты и полигональные системы.}\\
Далее поговорим про самоподобные дендриты, которые представляют интерес как для теории самоподобных множеств в целом, так и для текущей диссертации в частности.
Дендритом называют локально связный континуум, не содержащий простых замкнутых дуг.
Изучение дендритов занимает значительное место в общей топологии \cite{Kur1, whyburn1948antop}, а в своей работе \cite{Char} Я. Харатоник и В. Харатоник дают исчерпывающий обзор, охватывающий более чем 75 лет исследований в этой области.
В то же время в теории самоподобных множеств предпринимаются отдельные попытки выработать некоторые подходы к самоподобным дендритам.
Так в 1985 году М. Хата \cite{Hata1985}  показал, что самоподобный дендрит имеет бесконечное множество концевых точек.
В 1990 году К. Бандт показал в \cite{SSS6}, что жордановы дуги, соединяющие пары точек посткритически конечного самоподобного множества, являются самоподобными, а множество возможных значений размерностей таких дуг конечено, применив эти результаты к дендритам.
Он также рассмотрел факторизацию индексного пространства, приводящую к появлению дендритов в \cite{SSS2}.
Дж. Кигами в своей работе \cite{Kig95} применил методы гармонического анализа на фракталах к дендритам. 
%Д.Кройдон в своей диссертации [6] получил оценки теплового ядра для континуального случайного дерева и для некоторого семейства случайных дендритов p.c.f. на плоскости.
%Д.Думитру и А.Михаил [7] предприняли попытку получить достаточное условие для того, чтобы автомодельное множество было дендритом в терминах последовательностей графов пересечений для уточнения системы S.

Тем не менее, долгое время отсутствовали удобные геометрические методы построения подходящих систем сжимающих подобий, пока А. В. Тетенов, М. Самуэль и Д. А. Ваулин в статье \cite{TSV2017} не описали методы задания и геометрические свойства самоподобных дендритов в $\rr^d$ --- вопросы, до 2017 года еще не достаточно разработанные в теории самоподобных фракталов. 
Для этого строился и исследовался класс $P$-полиэдральных дендритов в $\rr^d$. 
Такие дендриты $K$ определяются как аттракторы систем $\eS = {S_1,\ldots, S_m}$ сжимающих подобий в$\rr^d$, переводящих заданный полиэдр $P \IN \rr^d$ в полиэдры $P_i \IN P$, попарные пересечения которых либо пусты, либо одноточечны и являются общими вершинами этих полиэдров, а гиперграф попарных пересечений полиэдров $P_i$ ацикличен.
Этими же авторами в ттом же году в работе \cite{STV2017} были более подробно изучены стягиваемые $P$-полигональные ситстемы --- двумерный частный случай $P$-полиэдральных систем.
Была рассмотрена возможность гомеоморфизма между аттракторами двух разных полигональных систем.
Тем не менее, оставался вопрос о возможности получения более широкого класса дендритов путём ослабления условий, задающих стягиваемые $P$-полигональные ситстемы. 
Получение такого обобщения является одной из целей данной работы.

Говоря о полигональных системах нельзя не упомянуть о полигаскетах, описанных Робертом Стритчартсом в работах \cite{strich1999, Strichartz1999}, которые хоть и не являются дендритами, но для их построения использовались схожие геометические методы.
Для полигаскетов им также была описаны кратчайше дуги, соединяющие пару точек полигаскета и имеющих минимальную размерность и меру.
Идеи кратчайших дуг позднее будут применены в вышеупомянутых работах \cite{TSV2017, STV2017} А. В. Тетенова, М. Самуэль и Д. А. Ваулина для построения главных дуг и главного дерева самоподобного дендрита, являющегося аттрактором полигональных систем.

Подходы к проверке дендритности самоподобного множества во многих работах связан с проверкой структуры попарных пересечения копий этого аттрактора.
Начать можно с результатов М. Хаты \cite{Hata1985}, который ещё в 1985 году определил для самоподобного множества его граф пересечений, в котором вершинам соответствовали копии самоподобного множества и эти вершины соединялись ребром, если соотвествующие копии имеют непустое пересечение.
Используя этот граф, Хата доказал критерий связности: самоподобное множество связно тогда и только тогда, когда его граф пересечений связен.
В дальнейшем К. Бандт и К. Келлер в работе \cite{SSS2} показали, что если у самоподобного множества копии пересекаются не более чем по одной точке и его граф пересечений есть дерево, то это множество является дендритом.

Результат Бандта и Келлера не учитывал дендриты, в которых по одной точке пересекались более двух копий. 
Эту проблему решил А.В. Тетенов в своём препринте \cite{FIP}, определив двудольный граф пересечений, в котором одной доле соответствуют копии аттрактора, а другой доле --- точки попарных пересечений этих копий. Ребром в графе могут соединятся только точки разных долей, если соответствующая точка пересечения лежит в соответствующей копии.
Таким образом был получен критерий дендритности для самоподобных множеств с одноточечным пересечением.

{\bf Главное дерево и кратчайшие дуги}\\

Фрактальные квадраты, в свою очередь, являются давно известным классом самоподобных множеств.
Фрактальные квадраты имеюю немало более общих классов, например фракталные $k$-кубы, ковры Бедфорда-МакМаллена, ковры Баранского и губки Серпинского.

{\bf Тут написать про фрактальные кважраты, кубы и губки.}\\
Ковры бедфорда макмаллена и губки серпинского. ковры баранского\\
проблемы размерности и меры ковров и губок (Юваль, Перес). Обзор Фрейзера о размерности\\
Фрактальный квадрат, его топологические свойства (LLR)\\
Фрактальные лабиринты и фрактальные треугольники Лоретты\\
Фрактальные квадраты с конечным числом компонент, связь с графом пересечений\\
актуальность темы результатов о перекрытиях.\\

\begin{center}
\textbf{Выносимые на защиту положения.}
\end{center}

\begin{enumerate}
\item Найдено необходимое условие (теорема о совпадении параметров) того, что аттрактор обобщённой полигональной системы является дендритом.
Доказано существование такого $\delta$, что при соблюдении условия совпадении параметров аттрактор любой $\delta$-деформации $\eS'$ стягиваемой $P$-полигональной системы $\eS$ будет дендритом, причём гомеоморфным аттрактору системы $\eS$.

\item Для пары фрактальных $k$-кубов $K_1$ и $K_2$ порядка $n$ получена система множеств $\Sigma=\Sigma(K_1,K_2)=\{F_\bma=K_1\cap(K_2+\bma)\ :\ \bma\in\{-1,0,1\}^k\}$ попарных пересечений их противоположных граней, где $F_0=K_1\cap K_2$. 
Оказалось, что $\Sigma$ является аттрактором граф-ориентированной системы подобий.
Были изучены свойства меры и размерности этих пересечений.

\item Был найден критерий дендритности фрактальных квадратов. Так же был получен метод проверки фрактальных $k$-кубов со свойством одноточечного пересечения на свойство дендритности.

\item Было показано, что у фрактальных квадратов, являющихся денритами, всего семь возможных топологических шаблонов главного дерева.
\end{enumerate}

\begin{center}
    \textbf{Содержание диссертации.}
\end{center}

Перейдём к описанию структуры работы и точным формулировкам основных результатов.
Диссертация выполнена в издательской системе \LaTeX, содержит \red{??} страниц и состоит из введения, трёх глав, заключения и списка литературы.
Каждая глава разбита на параграфы.
Список литературы приведён в алфавитном порядке.

\textbf{Первая глава} посвящена обобщению класса стягиваемых полигональных систем,

Рассмотрим самоподобные дендриты, являющиеся аттракторами стягиваемой $P$-полигональной системы.

\restate{dfn:sss}

\restate{dfn:den}

Тетенов А. В., Самюэль М. и Ваулин Д. А. в своих работах \cite{TSV2017, STV2017} ввели геометрически задаваемую систему $\eS$ сжимающих подобий, задаваемый с помощью многоугольника $P$:

\restate{dfn:cps}

Аттрактор $K=K(\eS)$ такой системы обладает многими интересными свойствами.
Прежде всего нам интересно то, что $\eS$ удовлетворяет условию открытого множества (в качестве открытого множества можем взять $\dot P$), копии его аттрактора $K$ пересекаются друг с другом не более чем по одной точке, а сам аттрактор является дендритом:

\restate{thm:cpsden}

Результаты этой главы относятся именно к обобщению таких систем, для этого мы ослабим требования, накладываемые на стягиваемые полигональные системы:

\restate{dfn:gps}

Прежде всего интерес представляют те обобщённые полигональные системы, которые являются $\delta$-деформацией (т.е. являются обобщением) стягиваемой полигональной системы:

\restate{dfn:deform}

Стоит отметить, что обобщённая полигональная система вовсе не обязательно является деформацией какой-то стягиваемой полигональной системы.

Однако аттрактор $K$ обобщённой полигональной системы $\eS$ уже не обязательно будет дендритом, поэтому требуется для $\eS$ дополнительно проверить выполнение условия, задаваемого равенством \eqref{icnd}:

\restate{thm:pcint}

В таком случае аттрактор $K$ стягиваемой полигональной системы $\eS$ и аттрактор $K'$ её $\delta$-деформации $\eS'$ будут гомеоморфны:

\restate{thm:attrmap}

%индексная диаграма
%
%циклическая вершина и вершина, подчинённая циклической
%
%параметр
%
%теорема о совпадении параметров


Технически, теорема о совпадении параметров показывает необходимое условие того, что аттрактор $\da$-деформации будет дендритом. 
Однако если стягиваемую полигональную систему деформировать слишком сильно, то даже при соблюдении условия совпадения параметров аттрактор $\da$-деформации может и не быть дендритом. 
Значит нам нужны ограничения при деформациях. 
Поэтому далее мы оценим $\da$ в доказательстве теоремы о малых деформациях. 
Эта оценка и будет представлять из себя достаточное условие того, что аттрактор $\da$-деформации будет дендритом.


Параметры $\rho_0, \rho_1, \rho_2, \alpha_0$


\begin{center}
\textbf{Апробация результатов.}
\end{center}


\begin{center}
\textbf{Благодарности}
\end{center}



% \textbf{Структура работы.}
% \textbf{Объем и структура работы.} Диссертация состоит из~введения,
% \formbytotal{totalchapter}{глав}{ы}{}{},
% заключения и
% \formbytotal{totalappendix}{приложен}{ия}{ий}{}.
% %% на случай ошибок оставляю исходный кусок на месте, закомментированным
% %Полный объём диссертации составляет  \ref*{TotPages}~страницу
% %с~\totalfigures{}~рисунками и~\totaltables{}~таблицами. Список литературы
% %содержит \total{citenum}~наименований.
% %
% Полный объём диссертации составляет
% \formbytotal{TotPages}{страниц}{у}{ы}{}, включая
% \formbytotal{totalcount@figure}{рисун}{ок}{ка}{ков} и
% \formbytotal{totalcount@table}{таблиц}{у}{ы}{}.
% Список литературы содержит
% \formbytotal{citenum}{наименован}{ие}{ия}{ий}.

