% \newpage
\chapter*{Введение}                         % Заголовок
\addcontentsline{toc}{chapter}{Введение}    % Добавляем его в оглавление

% \newcommand{\actuality}{}
% \newcommand{\progress}{}
% \newcommand{\aim}{{\textbf\aimTXT}}
% \newcommand{\tasks}{\textbf{\tasksTXT}}
% \newcommand{\novelty}{\textbf{\noveltyTXT}}
% \newcommand{\influence}{\textbf{\influenceTXT}}
% \newcommand{\methods}{\textbf{\methodsTXT}}
% \newcommand{\defpositions}{\textbf{\defpositionsTXT}}
% \newcommand{\reliability}{\textbf{\reliabilityTXT}}
% \newcommand{\probation}{\textbf{\probationTXT}}
% \newcommand{\contribution}{\textbf{\contributionTXT}}
% \newcommand{\publications}{\textbf{\publicationsTXT}}

% \input{common/characteristic} % Характеристика работы по структуре во введении и в автореферате не отличается (ГОСТ Р 7.0.11, пункты 5.3.1 и 9.2.1), потому её загружаем из одного и того же внешнего файла, предварительно задав форму выделения некоторым параметрам

\begin{center}
\textbf{История вопроса и основные направления.}
\end{center}
***

\begin{center}
\textbf{Цели и задачи.}
\end{center}
***

\begin{center}
\textbf{Выносимые на защиту положения.}
\end{center}
***

\begin{center}
    \textbf{Содержание диссертации.}
\end{center}

Перейдём к описанию структуры работы и точным формулировкам основных результатов.
Диссертация выполнена в издательской системе \LaTeX, содержит \red{??} страниц и состоит из введения, трёх глав, заключения и списка литературы.
Каждая глава разбита на параграфы.
Список литературы приведён в алфавитном порядке.

\textbf{Первая глава} состоит из четырёх параграфов.
\red{В этой главе рассматриваются...}

В первом параграфе рассматриваются самоподобные дендриты, являющиеся аттракторами стягиваемой $P$-полигональной системы.

\restate{dfn:sss}

\restate{dfn:den}

Тетенов А. В., Самюэль М. и Ваулин Д. А. в своей работе \cite{TSV2017} ввели удобно задаваемую систему $\eS$ сжимающих подобий, задаваемый с помощью многоугольника $P$:

\restate{dfn:cps}

Аттрактор $K=K(\eS)$ такой системы обладает многими полезными свойствами.
Прежде всего нам интересно то, что $\eS$ удовлетворяет условию открытого множества (в качестве открытого множества можем взять $\dot P$), копии его аттрактора $K$ пересекаются друг с другом не более чем по одной точке, а сам аттрактор является дендритом:

\restate{thm:cpsden}

Начиная со второго параграфа мною рассматривается обобщение класса таких систем:

\restate{dfn:gps}

Для меня интерес представляют те обобщённые полигональные системы, которые являются $\delta$-деформацией (т.е. являются обобщением) стягиваемой полигональной системы:

\restate{dfn:deform}

Однако аттрактор $K$ обобщённой полигональной системы $\eS$ уже не обязательно будет дендритом, поэтому требуется для $\eS$ дополнительно проверить выполнение условия, задаваемого равенством \eqref{icnd}:

\restate{thm:pcint}

В таком случае аттрактор $K$ стягиваемой полигональной системы $\eS$ и аттрактор $K'$ её $\delta$-деформации $\eS'$ будут гомеоморфны:

\restate{thm:attrmap}

%В третьем параграфе 

индексная диаграма

циклическая вершина и вершина, подчинённая циклической

параметр

теорема о совпадении параметров


Технически, теорема о совпадении параметров показывает необходимое условие того, что аттрактор $\da$-деформации будет дендритом. 
Однако если стягиваемую полигональную систему деформировать слишком сильно, то даже при соблюдении условия совпадения параметров аттрактор $\da$-деформации может и не быть дендритом. 
Значит нам нужны ограничения при деформациях. 
Поэтому далее мы оценим $\da$ в доказательстве теоремы о малых деформациях. 
Эта оценка и будет представлять из себя достаточное условие того, что аттрактор $\da$-деформации будет дендритом.


Параметры $\rho_0, \rho_1, \rho_2, \alpha_0$


\begin{center}
\textbf{Апробация результатов.}
\end{center}


\begin{center}
\textbf{Благодарности}
\end{center}



% \textbf{Структура работы.}
% \textbf{Объем и структура работы.} Диссертация состоит из~введения,
% \formbytotal{totalchapter}{глав}{ы}{}{},
% заключения и
% \formbytotal{totalappendix}{приложен}{ия}{ий}{}.
% %% на случай ошибок оставляю исходный кусок на месте, закомментированным
% %Полный объём диссертации составляет  \ref*{TotPages}~страницу
% %с~\totalfigures{}~рисунками и~\totaltables{}~таблицами. Список литературы
% %содержит \total{citenum}~наименований.
% %
% Полный объём диссертации составляет
% \formbytotal{TotPages}{страниц}{у}{ы}{}, включая
% \formbytotal{totalcount@figure}{рисун}{ок}{ка}{ков} и
% \formbytotal{totalcount@table}{таблиц}{у}{ы}{}.
% Список литературы содержит
% \formbytotal{citenum}{наименован}{ие}{ия}{ий}.

