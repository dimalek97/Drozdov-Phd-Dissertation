\documentclass[14pt, a4paper]{extarticle}
%%% Кодировка и локализация %%%
\usepackage[utf8]{inputenc} % кодовая страница документа
\usepackage[T2A]{fontenc} % внутренняя кодировка  TeX
\usepackage[english,russian]{babel} % локализация и переносы
\usepackage{cmap} % русский поиск в pdf
\usepackage[top=20mm, left=30mm, right=10mm, bottom=20mm]{geometry}

\usepackage{amsmath} % удобная вёрстка многострочных формул, масштабирующийся текст в формулах, формулы в рамках и др.
\usepackage{amssymb} % математические символы
\usepackage{amsthm} % окружения «теорема», «лемма» и т. п.
\usepackage{amsfonts} % Ажурный \mathbb{} и готический \mathfrak{} шрифты
\usepackage{mathrsfs} % шрифт Эйлера \mathscr{}
\usepackage{euscript} % Шрифт Евклида \EuScript{}
% каллиграфический шрифт \mathcal{} не требует пакета
\usepackage{indentfirst} % русский стиль: отступ первого абзаца раздела
\usepackage{enumitem}

\title{Заключение от организации}
\author{Dmitry Drozdov}
\date{14.08.2024}

\begin{document}
\noindent\hspace{7cm}<<УТВЕРЖДАЮ>>\par
\noindent\hspace{7cm}И. о. директора Института математики\par
\noindent\hspace{7cm}им. С. Л. Соболева СО РАН\par
\noindent\hspace{7cm}д. ф.-м. н.,член-корр. РАН\par
\vspace{1cm}
\noindent\hspace{7cm}Миронов Андрей Евгеньевич\par
\noindent\hspace{7cm}<<\underline{\hspace{1cm}}>>\underline{\hspace{3cm}} 2024 г.\\

\vspace{1cm}

\begin{center}\bf
ЗАКЛЮЧЕНИЕ\\
Федерального государственного бюджетного учреждения науки\\
Института математики им. С. Л. Соболева\\
Сибирского отделения Российской академии наук  
\end{center}

Диссертация {<<\bf Анализ на самоподобных множествах с конечным пересечением>>} выполнена в лаборатории теории функций Федерального государственного бюджетного учреждения науки <<Институт математики им. С. Л. Соболева СО РАН>>.

В период подготовки диссертации соискатель Дроздов Дмитрий Алексеевич обучался в аспирантуре Федерального государственного бюджетного учреждения науки <<Институт математики им. С. Л. Соболева СО РАН>> и работал в  Федеральном государственном автономном образовательном учреждении высшего образования <<Новосибирский государственный университет>> на кафедре теории функций ММФ в должности ассистента и в международном математическом центре <<Математический центр в Академгородке>> в лаборатории <<Прикладная абстрактная алгебра: алгебраические методы в топологии, комбинаторике и теории сложности вычислений>> в должности инженера-исследователя.

В 2020 году окончил Федеральное государственное бюджетное образовательное учреждение высшего образование <<Горно-Алтайский государственный университет>>, присуждена квалификация магистра по направлению подготовки 01.04.01 <<Математика>>.

Справка о сдаче кандидатских экзаменов № 2024/10 выдана 07.05.2024 г. 

Научный руководитель --- доктор физико-математических наук, доцент Тетенов Андрей Викторович, основное место работы: Федеральное государственное бюджетное учреждение науки <<Институт математики им. С. Л. Соболева Сибирского отделения Российской академии наук>>, ведущий научный сотрудник лаборатории теории функций.

Тема диссертации утверждена учёным советом ФГБУН <<Институт математики им. С. Л. Соболева Сибирского отделения Российской академии наук>> (протокол от 20 августа 2024 г. № 9).

По итогам обсуждения принято следующее заключение:

Диссертационная работа <<Анализ на самоподобных множествах с конечным пересечением>> Дроздова Дмитрия Алексеевича является научно-ис\-сле\-до\-ва\-тель\-ской работой, посвящённой изучению самоподобных множеств на плоскости и в пространстве, самоподобных дендритов и фрактальных $k$-кубов.
Результаты неоднократно докладывались на российских и международных конференциях и научно-исследовательских семинарах.\\

{\bf 1. Оценка выполненной соискателем работы.}
Следующие достижения, полученные в диссертации Дроздовым Д.А., являются наиболее важными:

\begin{enumerate}[nolistsep]
\item Найдено необходимое условие, при котором аттрактор обобщённой полигональной системы является дендритом.
\item Доказано, что при достаточно малом $\delta=\delta(\mathcal{S})>0$ аттрактор любой (удовлетворяющей условию совпадения параметров) $\delta$-деформации $\mathcal{S}'$ полигональной системы $\mathcal{S}$ является дендритом, изоморфным аттрактору системы $\mathcal{S}$.
\item Получена формула, выражающая пересечение двух фрактальных $k$-кубов в терминах их множеств единиц.
Найдены условия, при которых такое пересечение будет пустым, конечным, счётным и несчётным.
Для конечного пересечения получена оценка мощности.
\item Разработан алгоритм, позволяющий проверить, является ли фрактальный $k$-куб дендритом с одноточечным пересечением.
\item Доказано, что нетривиальные односвязные фрактальные квадраты являются дендритами со свойством одноточечного пересечения.
\item Доказано, что нетривиальные односвязные фрактальные квадраты допускают ровно семь возможных топологических типов главного дерева.
\end{enumerate}\quad

\quad

{\bf 2. Личное участие автора в получении результатов, изложенных в диссертации.}
Все основные результаты, изложенные в диссертации получены совместно с соавторами (результаты второй главы получены с Тетеновым А.В. и Самуэль М., результаты третьей и четвёртой глав получены с Тетеновым А.В.), вклад соавторов в эти результаты является равным и неделимым.
Научному руководителю Тетенову А.В. принадлежит формулировка задач и общее руководство работы.\\

{\bf 3. Полнота изложения материалов диссертации в работах, опубликованных автором.}
По теме диссертации опубликовано 6 статей в научных журналах, 5 из них (1-3, 5, 6) включены в перечень ВАК.\\

\begin{enumerate}[nolistsep]
\item {\bf Drozdov D., Samuel M., Tetenov A.}, On deformation of polygonal dendrites preserving the intersection graph // The Art of Discrete and Applied Mathematics. 2021. Т. 4. № 2. С. 1--21.
\item {\bf Drozdov D., Samuel M., Tetenov A.}, On $\delta$-deformations of Polygonal Dendrites // Topological Dynamics and Topological Data Analysis. : Springer Singapore, 2021. С. 147--164.
\item
{\bf Drozdov D. A., Tetenov A. V.}, On the dendrite property of fractal cubes // Advances in the Theory of Nonlinear Analysis and Its Application. 2024. Т. 8. № 1. С. 73--80.
\item {\bf Drozdov D., Tetenov A.}, On fractal squares possessing finite intersection property // Bulletin of National University of Uzbekistan: Mathematics and Natural Sciences. 2022. Т. 5. № 3. С. 164--181.
\item {\bf Drozdov D., Tetenov A.}, On the classification of fractal square dendrites // Advances in the Theory of Nonlinear Analysis and Its Application. 2023. Т. 7. № 3. С. 19--96.
\item {\bf Ваулин Д. А., Дроздов Д. А., Тетенов А. В.}, О связных компонентах фрактальных кубов // Труды Института математики и механики УрО РАН. 2020. Т. 26. № 2. С. 98--107.
\end{enumerate}

Перечисленные работы вполне отражают содержание диссертационной работы Дроздова Д.А. 
Вклад авторов в совместные работы равноправен и неделим.\\

{\bf 4. Степень достоверности результатов.}
Все положения и выводы, выносимые на защиту, являются обоснованными.
Достоверность представленных результатов опирается на подробные исчерпывающие доказательства.\\

{\bf 5. Степень новизны.}
Все основные результаты, полученные в диссертации, являются новыми.\\

{\bf 6. Практическая значимость и ценнность результатов.}
Полученные результаты имеют теоретический характер и могут быть использованы для дальнейшего изучения самоподобных множеств, фрактальных кубов, ковров Бедфорда-МакМаллена и губок Серпинского.
Результаты работы могут быть использованы специалистами по комплексному, действительному и функциональному анализу, топологии и фрактальной геометрии.
%Результаты, полученные на основе полигональных дендритов и фрактальных квадратов, могут иметь практическое применение в радиофизике, во фрактальной реконструкции сигналов и радиолокационных изображений, в материаловедении и в других областях физики, химии и биологии.
\\

{\bf 7. Соответствие специальности.}
Диссертация <<Анализ на самоподобных множествах с конечным пересечением>> Дроздова Дмитрия Алексеевича полностью соответствует специальности 1.1.1 --- <<Вещественный, комплексный и функциональный анализ>>.\\


{\bf 8. Рекомендация к защите.}
Научная работа Дроздова Д.А. удовлетворяет всем требованиям ВАК, предъявляемым к кандидатским диссертациям.
Диссертация <<Анализ на самоподобных множествах с конечным пересечением>> Дроздова Дмитрия Алексеевича рекомендуется к защите на соискание учёной степени кандидата физико-математических наук по специальности 1.1.1 --- <<Вещественный, комплексный и функциональный анализ>>.
Заключение принято на заседании лаборатории теории функций с привлечением других сотрудников ИМ СО РАН после выступления Дроздова Д.А. 14 августа 2024 года на семинаре  «Геометрическая теория функций». 
На заседании присутствовало 14 человек:
д. ф.-м. н. Медных А.Д., 
д. ф.-м. н. Асеев В.В., 
д. ф.-м. н. Тетенов А.В., 
д. ф.-м. н. Подвигин И.В., 
д. ф.-м. н. Романов А.С., 
к. ф.-м. н. Абросимов Н.В., 
к. ф.-м. н. Выонг Х.Б., 
к. ф.-м. н. Пчелинцев В.А., 
к. ф.-м. н. Гичев В.М., 
к. ф.-м. н. Кононенко Л.И., 
к. ф.-м. н. Волокитин Е.П., 
к. ф.-м. н. Медных И.А.

Результаты голосования: <<за>> -- 14 чел., <<против>> -- 0 чел., <<воздержались>> -- 0 чел., протокол №<<\underline{\hspace{1cm}}>> от 14 августа 2024 г. 

\vspace{1cm}

\noindent Медных Александр Дмитриевич\\
\noindent д. ф.-м. н.,\\
\noindent главный научный сотрудник\\
\noindent лаборатории теории функций ИМ СО РАН\\

% надо заверить в канцелярии ИМ СО РАН

\end{document}
