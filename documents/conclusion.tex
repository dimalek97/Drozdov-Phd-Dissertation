\documentclass[14pt, a4paper]{extarticle}
%%% Кодировка и локализация %%%
\usepackage[utf8]{inputenc} % кодовая страница документа
\usepackage[T2A]{fontenc} % внутренняя кодировка  TeX
\usepackage[english,russian]{babel} % локализация и переносы
\usepackage{cmap} % русский поиск в pdf
\usepackage[top=20mm, left=30mm, right=10mm, bottom=20mm]{geometry}

\usepackage{amsmath} % удобная вёрстка многострочных формул, масштабирующийся текст в формулах, формулы в рамках и др.
\usepackage{amssymb} % математические символы
\usepackage{amsthm} % окружения «теорема», «лемма» и т. п.
\usepackage{amsfonts} % Ажурный \mathbb{} и готический \mathfrak{} шрифты
\usepackage{mathrsfs} % шрифт Эйлера \mathscr{}
\usepackage{euscript} % Шрифт Евклида \EuScript{}
% каллиграфический шрифт \mathcal{} не требует пакета
\usepackage{indentfirst} % русский стиль: отступ первого абзаца раздела
\usepackage{enumitem}

\title{Заключение от организации}
\author{Dmitry Drozdov}
\date{13.05.2025}


\begin{document}
\;\\
\vspace{1cm}

\noindent\hspace{7cm}<<УТВЕРЖДАЮ>>\par
\noindent\hspace{7cm}И. о. директора ИМ СО РАН\par
\noindent\hspace{7cm}д. ф.-м. н.,член-корр. РАН\par
\vspace{1cm}
\noindent\hspace{7cm}Миронов Андрей Евгеньевич\par
\noindent\hspace{7cm}<<\underline{\hspace{1cm}}>>\underline{\hspace{3cm}} 2025 г.\\

\vspace{3.5cm}

\begin{center}\bf
ЗАКЛЮЧЕНИЕ\\
Федерального государственного бюджетного учреждения науки\\
Института математики им. С. Л. Соболева\\
Сибирского отделения Российской академии наук (ИМ СО РАН) 
\end{center}



Диссертация {<<\bf Анализ на самоподобных множествах с конечным пересечением>>} выполнена в лаборатории геометрии, топологии и теории функций ИМ СО РАН (до 12.05.2025 г. --- в лаборатории теории функций).

В период подготовки диссертации соискатель Дроздов Дмитрий Алексеевич обучался в аспирантуре ИМ СО РАН и работал в  Федеральном государственном автономном образовательном учреждении высшего образования <<Новосибирский государственный университет>> на кафедре теории функций ММФ в должности ассистента и в ИМ СО РАН в лаборатории геометрии, топологии и теории функций (до 12.05.2025 г. --- в лаборатории теории функций) в должности инженера-исследователя.

В 2024 году окончил ИМ СО РАН, освоил программу подготовки научно-педагогических кадров в аспирантуре по направлению подготовки 01.06.01 <<Математика и механика>>, присвоена квалификация <<Исследователь. Пре\-по\-да\-ватель-исследователь>>.

Справка о сдаче кандидатских экзаменов № 2024/10 выдана 07.05.2024 г. 

Научный руководитель --- доктор физико-математических наук, доцент Тетенов Андрей Викторович, основное место работы: ИМ СО РАН, ведущий научный сотрудник лаборатории геометрии, топологии и теории функций.

Тема диссертации утверждена учёным советом ИМ СО РАН (протокол от 20 августа 2024 г. № 9).

По итогам обсуждения принято следующее заключение:

Диссертационная работа <<Анализ на самоподобных множествах с конечным пересечением>> Дроздова Дмитрия Алексеевича является научно-ис\-сле\-до\-ва\-тель\-ской работой, посвящённой изучению самоподобных множеств на плоскости и в пространстве, самоподобных дендритов и фрактальных $k$-кубов.
Результаты неоднократно докладывались на российских и международных конференциях и научно-исследовательских семинарах.\\

{\bf 1. Оценка выполненной соискателем работы.}
Следующие достижения, полученные в диссертации Дроздовым Д.А., являются наиболее важными:

\begin{enumerate}[nolistsep]
\item Найдено необходимое условие, при котором аттрактор обобщённой полигональной системы является дендритом.
\item Доказано, что при достаточно малом $\delta=\delta(\mathcal{S})>0$ аттрактор любой (удовлетворяющей условию совпадения параметров) $\delta$-деформации $\mathcal{S}'$ полигональной системы $\mathcal{S}$ является дендритом, изоморфным аттрактору системы $\mathcal{S}$.
\item Получена формула, выражающая пересечение двух фрактальных $k$-кубов в терминах их множеств единиц.
Найдены условия, при которых такое пересечение будет пустым, конечным, счётным и несчётным.
Для конечного пересечения получена оценка мощности.
\item Разработан алгоритм, позволяющий проверить, является ли фрактальный $k$-куб дендритом с одноточечным пересечением.
\item Доказано, что нетривиальные односвязные фрактальные квадраты являются дендритами со свойством одноточечного пересечения.
\item Доказано, что нетривиальные односвязные фрактальные квадраты допускают ровно семь возможных топологических типов главного дерева.
\item Доказано, что самоподобные $k$-леса могут быть реализованы на фрактальных квадратах.
\end{enumerate}\quad

\quad

{\bf 2. Личное участие автора в получении результатов, изложенных в диссертации.}
Научному руководителю Тетенову А.В. принадлежит формулировка задач и общее руководство работы.
Доказательства всех основных результатов, кроме теоремы о малых деформациях, и построение всех демонстрационных изображений были проделаны соискателем.
Теорема о малых деформациях и аппарат индексных диаграмм в диссертации (её главы 2) были разработаны и доказаны совместно с А.В. Тетеновым.\\

{\bf 3. Полнота изложения материалов диссертации в работах, опубликованных автором.}
Основные результаты диссертации опубликованы в пяти изданиях в журналах, входящих в официальный перечень ВАК и индексируемых в системах Web of Sience и Scopus.
Из них две совместных работы  с Тетеновым А.В. и Мауэль М.; две в совместных работы с Тетеновым А.В.; одна работа без соавторов.

%\begin{enumerate}[nolistsep]
%\item {\bf Drozdov D., Samuel M., Tetenov A.}, On deformation of polygonal dendrites preserving the intersection graph // The Art of Discrete and Applied Mathematics. 2021. Т. 4. № 2. С. 1--21.
%\item {\bf Drozdov D., Samuel M., Tetenov A.}, On $\delta$-deformations of Polygonal Dendrites // Topological Dynamics and Topological Data Analysis. : Springer Singapore, 2021. С. 147--164.
%\item
%{\bf Drozdov D. A., Tetenov A. V.}, On the dendrite property of fractal cubes // Advances in the Theory of Nonlinear Analysis and Its Application. 2024. Т. 8. № 1. С. 73--80.
%\item {\bf Drozdov D., Tetenov A.}, On the classification of fractal square dendrites // Advances in the Theory of Nonlinear Analysis and Its Application. 2023. Т. 7. № 3. С. 19--96.
%\item {\bf Дроздов Д. А.}, Самоподобные леса на фрактальных квадратах // Сибирские электронные математические известия. 2025. Т. 22, № 1. С. 385--394.
%\end{enumerate}

\begin{thebibliography}{9}
\bibitem{DST2021}
{\bf Drozdov D., Samuel M., Tetenov A.},
On deformation of polygonal dendrites preserving the intersection graph //
The Art of Discrete and Applied Mathematics. 2021. Т. 4. № 2. С. 1--21.

\bibitem{DST2022}
{\bf Drozdov D., Samuel M., Tetenov A.}, 
On $\delta$-deformations of Polygonal Dendrites // 
Topological Dynamics and Topological Data Analysis. : Springer Singapore, 2021. С. 147--164.

\bibitem{DT2024fqd}
{\bf Drozdov D. A., Tetenov A. V.}, On the dendrite property of fractal cubes // Advances in the Theory of Nonlinear Analysis and Its Application. 2024. Т. 8. № 1. С. 73--80.

\bibitem{TD2023fs}
{\bf Drozdov D., Tetenov A.}, 
On the classification of fractal square dendrites // 
Advances in the Theory of Nonlinear Analysis and Its Application. 2023. Т. 7. № 3. С. 19--96.

\bibitem{Drozdov2025} 
{\bf Дроздов Д. А.}, 
Самоподобные леса на фрактальных квадратах // Сибирские электронные математические известия. 2025. Т. 22, № 1. С. 385--394.

\end{thebibliography}

Перечисленные работы вполне отражают содержание диссертационной работы Дроздова Д.А. 
В совместных работах формулировки задач и общее руководство принадлежат А.В. Тетенову.
Доказательства большинства вошедших в текст диссертации (её глава 2) новых теорем из работ \cite{DST2021, DST2022} и построение всех демонстрационных изображений были проделаны автором.
Теорема о малых деформациях и аппарат индексных диаграмм из работ \cite{DST2021, DST2022} были разработаны и доказаны совместно с А.В. Тетеновым.

Доказательства всех вошедших в текст диссертации (её главы 3 и 4) новых теорем из работ \cite{DT2024fqd, TD2023fs} и построение всех демонстрационных изображений были проделаны автором.\\

{\bf 4. Степень достоверности результатов.}
Все положения и выводы, выносимые на защиту, являются обоснованными.
Достоверность представленных результатов опирается на подробные исчерпывающие доказательства.\\

{\bf 5. Степень новизны.}
Все основные результаты, полученные в диссертации, являются новыми.\\

{\bf 6. Практическая значимость и ценнность результатов.}
Полученные результаты имеют теоретический характер и могут быть использованы для дальнейшего изучения самоподобных множеств, фрактальных кубов, ковров Бедфорда-МакМаллена и губок Серпинского.
Результаты работы могут быть использованы специалистами по комплексному, действительному и функциональному анализу, топологии и фрактальной геометрии.
%Результаты, полученные на основе полигональных дендритов и фрактальных квадратов, могут иметь практическое применение в радиофизике, во фрактальной реконструкции сигналов и радиолокационных изображений, в материаловедении и в других областях физики, химии и биологии.
\\

{\bf 7. Соответствие специальности.}
Диссертация <<Анализ на самоподобных множествах с конечным пересечением>> Дроздова Дмитрия Алексеевича полностью соответствует специальности 1.1.1 --- <<Вещественный, комплексный и функциональный анализ>>.\\

{\bf 8. Диссертация Д.А. Дроздова не включает материалов, содержащих государственную или коммерческую тайну.}\\

{\bf 9. Рекомендация к защите.}
Научная работа Дроздова Д.А. удовлетворяет всем требованиям ВАК, предъявляемым к кандидатским диссертациям.
Диссертация <<Анализ на самоподобных множествах с конечным пересечением>> Дроздова Дмитрия Алексеевича рекомендуется к защите на соискание учёной степени кандидата физико-математических наук по специальности 1.1.1 --- <<Вещественный, комплексный и функциональный анализ>>.
Заключение принято на заседании лаборатории теории функций с привлечением других сотрудников ИМ СО РАН после выступления Дроздова Д.А. 14 августа 2024 года на семинаре  «Геометрическая теория функций». 
На заседании присутствовало 14 человек:
д. ф.-м. н. Медных А.Д., 
д. ф.-м. н. Асеев В.В., 
д. ф.-м. н. Тетенов А.В., 
д. ф.-м. н. Подвигин И.В., 
д. ф.-м. н. Романов А.С., 
к. ф.-м. н. Абросимов Н.В., 
к. ф.-м. н. Выонг Х.Б., 
к. ф.-м. н. Пчелинцев В.А., 
к. ф.-м. н. Гичев В.М., 
к. ф.-м. н. Кононенко Л.И., 
к. ф.-м. н. Волокитин Е.П., 
к. ф.-м. н. Медных И.А.

Результаты голосования: <<за>> -- 14 чел., <<против>> -- 0 чел., <<воздержались>> -- 0 чел., протокол №<<\underline{\hspace{1cm}}>> от 14 августа 2024 г. 

\vspace{1cm}

\noindent Медных Александр Дмитриевич\\
\noindent д. ф.-м. н.,\\
\noindent главный научный сотрудник\\
\noindent лаборатории геометрии, топологии\\
\noindent  и теории функций ИМ СО РАН\\

% надо заверить в канцелярии ИМ СО РАН

\end{document}
