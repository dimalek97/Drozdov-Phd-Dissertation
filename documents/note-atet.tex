\documentclass[14pt, a4paper]{extarticle}
%%% Кодировка и локализация %%%
\usepackage[utf8]{inputenc} % кодовая страница документа
\usepackage[T2A]{fontenc} % внутренняя кодировка  TeX
\usepackage[english,russian]{babel} % локализация и переносы
\usepackage{cmap} % русский поиск в pdf
\usepackage[top=20mm, left=30mm, right=10mm, bottom=20mm]{geometry}

\usepackage{amsmath} % удобная вёрстка многострочных формул, масштабирующийся текст в формулах, формулы в рамках и др.
\usepackage{amssymb} % математические символы
\usepackage{amsthm} % окружения «теорема», «лемма» и т. п.
\usepackage{amsfonts} % Ажурный \mathbb{} и готический \mathfrak{} шрифты
\usepackage{mathrsfs} % шрифт Эйлера \mathscr{}
\usepackage{euscript} % Шрифт Евклида \EuScript{}
% каллиграфический шрифт \mathcal{} не требует пакета
\usepackage{indentfirst} % русский стиль: отступ первого абзаца раздела
\usepackage{enumitem}

\title{Заключение от организации}
\author{Dmitry Drozdov}
\date{14.08.2024}

\begin{document}

\noindent\hspace{9cm}В диссертационный совет

\noindent\hspace{9cm}Д 21.1.074.01

\noindent\hspace{9cm}630090, г. Новосибирск, 

\noindent\hspace{9cm}просп. академика Коптюга, д. 4.



\begin{center}
Справка 
\end{center}

Я ознакомлен с диссертацией Дроздова Дмитрия Алексеевича <<Анализ на самоподобных множествах с конечным пересечением>> на соискание учёной степени кандидата физико-математических наук по специальности 1.1.1 --- <<Вещественный, комплексный и функциональный анализ>>.

Настоящим подтверждаю, что использованные в диссертации результаты статей
\begin{enumerate}[nolistsep]
\item {\bf Drozdov D., Samuel M., Tetenov A.}, On deformation of polygonal dendrites preserving the intersection graph // The Art of Discrete and Applied Mathematics. 2021. Т. 4. № 2. С. 1--21.
\item {\bf Drozdov D., Samuel M., Tetenov A.}, On $\delta$-deformations of Polygonal Dendrites // Topological Dynamics and Topological Data Analysis. : Springer Singapore, 2021. С. 147--164.
\item {\bf Drozdov D., Tetenov A.}, On fractal squares possessing finite intersection property // Bulletin of National University of Uzbekistan: Mathematics and Natural Sciences. 2022. Т. 5. № 3. С. 164--181.
\item {\bf Drozdov D., Tetenov A.}, On the classification of fractal square dendrites // Advances in the Theory of Nonlinear Analysis and Its Application. 2023. Т. 7. № 3. С. 19--96.
\item {\bf Drozdov D., Tetenov A.}, On the dendrite property of fractal cubes // Advances in the Theory of Nonlinear Analysis and Its Application. 2024. Т. 8. № 1. С. 73--80.
\item {\bf Ваулин Д. А., Дроздов Д. А., Тетенов А. В.}, О связных компонентах фрактальных кубов // Труды Института математики и механики УрО РАН. 2020. Т. 26. № 2. С. 98--107.
\end{enumerate}
получены Дроздовым Д.А., мной и соавторами (Ваулин Д.А., Самуэль М.) совместно и являются неделимыми.

\vspace{1cm}


\noindent д. ф.-м. н., доцент,\\
\noindent ведущий научный сотрудник\\
\noindent лаборатории теории функций ИМ СО РАН\\
\noindent Тетенов Андрей Викторович\hfill 12.09.2024

% надо заверить в канцелярии ИМ СО РАН

\end{document}
