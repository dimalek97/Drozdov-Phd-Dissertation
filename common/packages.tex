%%% Кодировка и локализация %%%
\usepackage[utf8]{inputenc} % кодовая страница документа
\usepackage[T2A]{fontenc} % внутренняя кодировка  TeX
\usepackage[english,russian]{babel} % локализация и переносы


%%% Гипперссылки и pdf %%%
\usepackage{cmap} % русский поиск в pdf
\usepackage[pagebackref]{hyperref} % гиперссылки
% для pagebackref после каждого bibitem нужна пустая строка или \par
\hypersetup{
    unicode, % корректная работа ссылок с кириллицей
    psdextra, % частично корректная работа ссылок с формулами (не все, ещё есть проблемы с math shift, subscript и superscript)
    pdftitle={Анализ на самоподобных множествах с конечным пересечением}, % название документа
    pdfsubject={Диссертация аспиранта}, % тема документа
    pdfauthor={Дроздов Дмитрий Алексеевич}, % авторы документа
    pdfkeywords={Фрактальный квадрат, фрактальный куб, самоподоное множество}, % список ключевых слов,
    colorlinks, % текст гиперссылки будет выделяться цветом, а рамки не будет
    linkcolor=black, % цвет текста ссылок на мишени внутри документа
    pagecolor=black, % цвет текста ссылок на страницы
    filecolor=cyan, % цвет текста ссылок на локальные PDF файлы
    citecolor=black, % цвет библиографических ссылок команды \cite
    urlcolor=black, % цвет текста ссылок на ресурсы с URL
    % draft, % отключить все гипертекстовые опции
}


%%% Пакеты для формул %%%
\usepackage{amsmath} % удобная вёрстка многострочных формул, масштабирующийся текст в формулах, формулы в рамках и др.
\usepackage{amssymb} % несколько сотен дополнительных математических символов
\usepackage{amsthm} % окружения «теорема», «лемма» и т. п.
\usepackage{amsfonts} % Ажурный \mathbb{} и готический \mathfrak{} шрифты
\usepackage{mathrsfs} % Mathematical Script letters (шрифт Эйлера) \mathscr{}
\usepackage{euscript} % Шрифт Евклида \EuScript{}
% каллиграфический шрифт \mathcal{} не требует пакета
\usepackage{enumitem}
\usepackage{bm}
\usepackage{multicol}
% \usepackage{cite}     % оптимизирует списки цитирования
\usepackage{lastpage}

%%% Графика, изображения и цвета %%%
\usepackage[usenames]{color} % позволяет задавать цвет текста и фона, как отдельного блока, так и всего документа
\usepackage{graphicx} % Работа с графикой \includegraphics{}
\usepackage{float} % Фиксация плавабщих боксов [H]
\graphicspath{
    {./images/}, 
    {./images/img1/}, 
    {./images/img2/}, 
    {./images/img3/},
    {./images/img4/}, 
    {./images/tikz1/}, 
    {./images/tikz2/}, 
    {./images/tikz3/},
    {./images/tree/}
    }
% Вставка tikz-картинок с помощью \includestandalone[width=\textwidth]{images/tikz1/name.tex}
% requires -shell-escape
% \usepackage[mode=buildnew]{standalone} 
% \usepackage{tikz}
%     \usetikzlibrary{decorations.markings}
%     \usetikzlibrary{arrows}
%     \usetikzlibrary{arrows.meta}
%     \usetikzlibrary{cd}
%     \usetikzlibrary{calc}


%\usepackage[parentracker=true,
%             backend=biber,
%             hyperref=false,
%             bibencoding=utf8,
%             style=numeric-comp,
%             language=auto,
%             autolang=other,
%             citestyle=gost-numeric,
%             defernumbers=true,
%             bibstyle=gost-numeric,
%             sorting=ntvy,
%             ]{biblatex}
%\addbibresource{dissbib.bib}


