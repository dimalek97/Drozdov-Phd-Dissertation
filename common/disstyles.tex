%%% Макет страницы %%%
% Выставляем значения полей (ГОСТ 7.0.11-2011, 5.3.7)
\usepackage[a4paper, top=2cm, bottom=2cm, left=2.5cm, right=1cm, nofoot, nomarginpar]{geometry}
\setlength{\topskip}{0pt}   %размер дополнительного верхнего поля
\setlength{\footskip}{1cm} % снимет warning, согласно https://tex.stackexchange.com/a/334346


%%% Интервалы %%%
%% По ГОСТ Р 7.0.11-2011, пункту 5.3.6 требуется полуторный интервал
\usepackage{setspace} % Реализация средствами класса (на основе setspace) ближе к типографской классике.
%% И правит сразу и в таблицах (если со звёздочкой)
%\DoubleSpacing*     % Двойной интервал
\onehalfspacing    % Полуторный интервал
%\setSpacing{1.42}   % Полуторный интервал, подобный Ворду (возможно, стоит включать вместе с предыдущей строкой)
\renewcommand{\baselinestretch}{1} % интервал между абзацами

\usepackage{indentfirst} % отделять первую строку раздела абзацным отступом
\renewcommand{\baselinestretch}{1} % интервал между абзацами
\setlength{\parskip}{0cm}
\setlength\parindent{5ex} % абзацный отступ равный по ширине пяти строчным буквам основного шрифта
%\parindent=1cm


%%% Выравнивание и переносы %%%
%% http://tex.stackexchange.com/questions/241343/what-is-the-meaning-of-fussy-sloppy-emergencystretch-tolerance-hbadness
%% http://www.latex-community.org/forum/viewtopic.php?p=70342#p70342
\tolerance 1414
\hbadness 1414
\emergencystretch 1.5em % В случае проблем регулировать в первую очередь
\hfuzz 0.3pt
\vfuzz \hfuzz
%\raggedbottom
%\sloppy                 % Избавляемся от переполнений
\clubpenalty=10000      % Запрещаем разрыв страницы после первой строки абзаца
\widowpenalty=10000     % Запрещаем разрыв страницы после последней строки абзаца
\brokenpenalty=4991     % Ограничение на разрыв страницы, если строка заканчивается переносом

\pagestyle{plain} % без колонтитулов, номера страниц внизу в центре



\usepackage{environ}

\makeatletter
\long\def\@secondofthree#1#2#3{#2}
\long\def\@thirdoffour#1#2#3#4{#3}

% #1 -- theorem type
% #2 -- name
\NewEnviron{restatethis}[2]{%
  \begin{#1}%
      \BODY
      % Do the write inside the body of the theorem environment because if I do it afterwards it messes
      % up the spacing for some strange reason...
      \protected@write\@auxout{}{%
        \string\@restatetheorem{#1}{#2}{\csname the#1\endcsname}{\detokenize\expandafter{\BODY}}%
      }%
  \end{#1}%
}

% Figure out which macro to redefine to change the numbering.
% Easy if thmtools is loaded, otherwise we have to extract the name from some macro.
% This will break if some jerk loads thmtools and then ntheorem.
\@ifpackageloaded{thmtools}{%
    % just use the name of the environment
    \def\restatethm@getthmcountercsname#1{\def\thethmcsname{#1}}%
}{%
    \@ifpackageloaded{ntheorem}{%
        % extract the right macro from \mkheader@thmtype -> ???? \@thm{environmentname}{numberwithname}{thmname}
        \def\restatethm@getthmcountercsname#1{%
            \def\thethmcsname{\expandafter\expandafter\expandafter\restatethm@ntheorem@getthmcountercsname@helper\csname mkheader@#1\endcsname}}%
        \def\restatethm@ntheorem@getthmcountercsname@helper#1\@thm#2#3#4{#3}
    }{%
        \@ifpackageloaded{amsthm}{%
            % extract the right macro from \thmtype -> \@thm{????}{numberwithname}{thmname} (it's the third thing)
            \def\restatethm@getthmcountercsname#1{\edef\thethmcsname{\expandafter\expandafter\expandafter\@thirdoffour\csname#1\endcsname}}%
        }{%
            % extract the right macro from \thmtype -> \@thm{numberwithname}{thmname} (it's the second thing)
            \def\restatethm@getthmcountercsname#1{\edef\thethmcsname{\expandafter\expandafter\expandafter\@secondofthree\csname#1\endcsname}}%
        }%
    }%
}

% #1 -- environment type
% #2 -- name for restating
% #3 -- number
% #4 -- content
\newcommand{\@restatetheorem}[4]{%
  \expandafter\gdef\csname restatethis@#2\endcsname{%
    \begingroup
    % First we have to set the numerical label to the right value.
    \restatethm@getthmcountercsname{#1}
    \expandafter\def\csname the\thethmcsname\endcsname{#3}%
    % Print the theorem
    \begin{#1}#4\end{#1}%
    \endgroup
  }%
}
\newcommand{\restate}[1]{\csname restatethis@#1\endcsname} % Just use the stored macro
\makeatother


\newtheorem{theorem}               {\bf Теорема}     [chapter]
\newtheorem{corollary}   [theorem] {\bf Следствие}
\newtheorem{lemma}       [theorem] {\bf Лемма}
\newtheorem{proposition} [theorem] {\bf Предложение}
\newtheorem{definition}  [theorem] {\bf Определение}
\newtheorem{remark}                {\bf Замечание}   [chapter]
\newtheorem{example}               {\bf Пример}      [chapter]

