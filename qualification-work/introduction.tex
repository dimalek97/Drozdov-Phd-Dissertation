% \newpage
\chapter*{Введение}                         % Заголовок
\addcontentsline{toc}{chapter}{Введение}    % Добавляем его в оглавление

% \newcommand{\actuality}{}
% \newcommand{\progress}{}
% \newcommand{\aim}{{\textbf\aimTXT}}
% \newcommand{\tasks}{\textbf{\tasksTXT}}
% \newcommand{\novelty}{\textbf{\noveltyTXT}}
% \newcommand{\influence}{\textbf{\influenceTXT}}
% \newcommand{\methods}{\textbf{\methodsTXT}}
% \newcommand{\defpositions}{\textbf{\defpositionsTXT}}
% \newcommand{\reliability}{\textbf{\reliabilityTXT}}
% \newcommand{\probation}{\textbf{\probationTXT}}
% \newcommand{\contribution}{\textbf{\contributionTXT}}
% \newcommand{\publications}{\textbf{\publicationsTXT}}

% \input{common/characteristic} % Характеристика работы по структуре во введении и в автореферате не отличается (ГОСТ Р 7.0.11, пункты 5.3.1 и 9.2.1), потому её загружаем из одного и того же внешнего файла, предварительно задав форму выделения некоторым параметрам

\begin{center}
\textbf{История вопроса и основные направления.}
\end{center}
% история вопроса
{\bf Тут добавить историю фракталов до 1970 года}


На практике строго определить фрактал не так просто.
К.~Фальконер \cite{Falconer2004} даёт несколько необязательных признаков, котрорым могут удовлетворять фракталы.
Так для того чтобы можно было назвать объект $A$ фракталом, он должен характеризоваться какими-либо из следующих свойств:

\begin{enumerate}
\item $A$ имеет тонкую структуру, т. е. содержит сложные структурные элементы на любых масштабах;
\item $A$ слишком неоднородно, чтобы описываться на традиционном геометрическом языке;
\item $A$ самоподобно в том или ином смысле, т. е. имеет повторяющуюся структуру в разных масштабах. Возможно, самоподобие приблизительное или статистическое.
\item Каким-то образом определенная <<фрактальная>> размерность множества $A$ превышает его топологическую размерность и зачастую является дробным числом;
\item $A$ можно построить через рекурсивные или итеративные схемы (что позволяет моделировать фракталы на компьютерах).
\end{enumerate}


Одним из крупнейших разделов фрактальной геоиетрии является теория самоподобных множеств.
На протяжении всей работы мы будем рассматривать именно самоподобные множества.
Хотя понятие самоподобия впервые появилось у П. Леви в 1939 году \cite{Levy1939} при описании его знаменитой кривой Леви, но наиболее существенный вклад в теорию самоподобных множеств внёс Дж. Хатчинсон в 1981 г. \cite{Hut1981}.
Он дал строгое определение самоподобного множества, состоящего из уменьшеных образов самого себя, и описал четкий математический подход к исследованию таких множеств. 
Работа Хатчинсона послужила толчком для множества дальнейших исследований, многие из которых расширяют и обобщают класс самоподобных множеств.

Прежде всего стоит отметить вклад Р. Молдина и С. Вильямса \cite{MW1988}, которые разработали концепцию граф-ориентированных систем подобий, аттракторром которых является уже система компактов, каждый из которых может состаять не только из своих копий, но и из копий других компактов системы.
% Данная концепция перекликается с идеей самоподобных множеств с конденсацией, введённых \red{???}.

В 1996 году М. Моран \cite{Moran1996} определил бесконечно порождённые самоподобные множества и рассмотрел их свойства.
Эти множества имеют свои особенности при вычислении размерности и в целом значительно расширяют класс стандартных самоподобных множеств.

Одним из важнейших вопросов теории самоподобных множеств является вопрос о вычислении их хаусдорфовой размерности.
Вычисление размерности самоподобных множеств напрямую связано со структурой пересечений их копий, что в свою очередь тесно связано с условиями отделимости порождающих эти фракталы систем сжимающих подобий, такими как условие открытого множества (OSC) и слабое условие отделимости (WSP).
Множества, не удовлетворяющие никаким стандартным условиям отделимости, могут быть весьма сложными.

%Прежде всего стоит отметить вклад \red{кого???}, описавших строгое условие отделимости (SSC).
%Мы говорим, что самоподобное множество удовлетворяет строгому условию отделимости, если его копии попарно друг с другом не пересекаются.
%Однако это условие слишком сильное и для большинства множеств оно не подходит.

П.~Моран в 1946 г. \cite{Moran1946} ввел условие открытого множества (OSC) для самоподобных множеств на прямой, а Дж.~Хатчинсон \cite{Hut1981} обобщил введенное Мораном условие открытого множества на системы сжимающих подобий в $\rr^n$ для любого натурального $n$.
Мы говорим, что система сжимающих подобий $\eS=\{S_1, \ldots, S_m\}$ удовлетворяет условию открытого множества, если существует открытое множество $O$ такое, что множества $\{O_i=S_i(O) | S_i\in\eS\}$ содержатся в $O$ и попарно друг с другом не пересекаются.
Размерность Хаусдорфа самоподобных множеств, удовлетворяющих OSC или SSC, равна размерности подобия, которая легко вычисляется.
Так для системы $\eS=\{S_1,\ldots,S_m\}$ сжимающих подобий в $\rr^n$ с коэффициентами подобия $r_1, \ldots, r_m$ выполнено условие открытого множества (или строгое условие отделимости), то хаусдорова размерность  аттрактора этой системы равна его размерности подобия $s$, которая является решением следующего уравнения: $$r_1^s+\ldots+r_m^s=1.$$ 

Условие открытого множества тоже подходит далеко не для всех примеров.
Есть примеры самоподобных множеств с копиями, пересекающимися по множеству со столь сложной структурой, что вопрос существования подходящего открытого множества не является очевидным.
Иногда подходящим открытым множеством является внутренность самого фрактала, что не является удобным для проверки. 
Порой подходящее открытое множество может иметь очень сложную стуктуру, например состоять из бесконечного числа непересекающихся компонент.

К.~Бандт и З.~Граф в 1992 году \cite{SSS7} искали алгебраический аналог для OSC и ввели алгебраическое условие, основанное на ассоциированном семействе подобий. 
Они показали, что оно эквивалентно условию положительности хаусдорфовой меры аттрактора в размерности $s$, где $s$ --- размерность подобия для данной системы подобий.
Вышеупомянутое алгебраическое условие заключается в том, что копии самоподобного множества попарно пересекаются по множеству нулевой меры Хаусдорфа в размерности $s$.

Условие открытого множества можно усилить следующим требованием: открытое множество $O$ и аттрактор $K$ системы $\eS$ имеют непустое пересечение. 
Так получается сильное условие открытого множества (SOSC).
Однако А.~Шиф \cite{Schief1994} в 1994 г. показал, что все три условия: SOSC, OSC и условие положительности меры Хаусдорфа в размерности подобия --- эквивалентны.

Условие открытого множетсва и слабое условие отделимости позволяют вычислять размерность Хаусдорфа с помощью размерности подобия, поскольку копии самоподобного множества пересекаются не слишком сильно или вовсе не пересекаются.
Но даже для тех случаев, когда копии фрактала пересекаются по множеству ненулевой меры, есть возможность вычислить размерность, если оно удовлетворяет  слабому условию отделимости (WSP), определённому М.~Цернером \cite{Zerner1996}.
Наиболее хорошо оно себя показывает в случаях, когда копии самоподобного множества пересекаются по своим подкопиям.

Тем не менее, для систем, не удовлетворяющих ни OSC, ни WSP, вычисление размерности Хаусдорфа их аттракторов становится действительно сложной проблемой.\\

Далее поговорим про самоподобные дендриты, которые представляют интерес как для теории самоподобных множеств в целом, так и для текущей диссертации в частности.
Дендритом называют локально связный континуум, не содержащий простых замкнутых дуг.
Изучение дендритов занимает значительное место в общей топологии \cite{Kur1, Kur2}, а Я. Харатоник и В. Харатоник дают в своей работе \cite{Char1998} исчерпывающий обзор, охватывающий более чем 75 лет исследований в этой области.
В то же время в теории самоподобных множеств предпринимаются отдельные попытки выработать некоторые подходы к самоподобным дендритам.
Так в 1985 году М. Хата \cite{Hata1985} помимо других базовых свойств дендритов показал, что самоподобный дендрит имеет бесконечное множество концевых точек.
В 1990 году К. Бандт показал в \cite{SSS6}, что жордановы дуги, соединяющие пары точек посткритически конечного самоподобного множества, являются самоподобными, а множество возможных значений размерностей таких дуг конечено, применив эти результаты к дендритам.
Он также рассмотрел факторизацию индексного пространства, приводящую к появлению дендритов в \cite{SSS2}.
Дж. Кигами в своей работе \cite{Kig95} применил методы гармонического анализа на фракталах к дендритам. 

Были построены и изучены многие отдельные представители самоподобных дендритов, например дерево Хаты \cite{Hata1985}, множество Вичека или пентадендрит \cite{McWorter1987}.
Тем не менее, долгое время отсутствовали удобные геометрические методы построения подходящих систем сжимающих подобий, пока А. В. Тетенов, М. Самуэль и Д. А. Ваулин в статьях \cite{TSV2017} не описали методы задания и геометрические свойства самоподобных дендритов в $\rr^d$ --- вопросы, до 2017 года еще не достаточно разработанные в теории самоподобных фракталов. 
Для этого строился и исследовался класс $P$-полиэдральных дендритов в $\rr^d$. 
Такие дендриты $K$ определяются как аттракторы систем $\eS = {S_1,\ldots, S_m}$ сжимающих подобий в$\rr^d$, переводящих заданный полиэдр $P \IN \rr^d$ в полиэдры $P_i \IN P$, попарные пересечения которых либо пусты, либо одноточечны и являются общими вершинами этих полиэдров, а гиперграф попарных пересечений полиэдров $P_i$ ацикличен.
Этими же авторами в том же году в работе \cite{STV2017} были более подробно изучены стягиваемые $P$-полигональные ситстемы --- двумерный частный случай $P$-полиэдральных систем.
Была рассмотрена возможность гомеоморфизма между аттракторами двух разных полигональных систем.
Тем не менее, оставался вопрос о возможности получения более широкого класса дендритов путём ослабления условий, задающих стягиваемые $P$-полигональные ситстемы. 
Получение такого обобщения является одной из целей данной работы.

Говоря о полигональных системах нельзя не упомянуть о полигаскетах, описанных Робертом Стритчартсом в работах \cite{strich1999, Strichartz1999}, которые хоть и не являются дендритами, но для их построения использовались схожие геометические методы.
Для полигаскетов им также была описаны кратчайше дуги, соединяющие пару точек полигаскета и имеющих минимальную размерность и меру.
Идеи кратчайших дуг позднее будут применены в вышеупомянутых работах \cite{TSV2017, STV2017} А. В. Тетенова, М. Самуэль и Д. А. Ваулина для построения главных дуг и главного дерева самоподобного дендрита, являющегося аттрактором полигональных систем.

Подходы к проверке дендритности самоподобного множества во многих работах связан с проверкой структуры попарных пересечения копий этого аттрактора.
Начать можно с результатов М. Хаты \cite{Hata1985}, который ещё в 1985 году определил для самоподобного множества его граф пересечений, в котором вершинам соответствовали копии самоподобного множества и эти вершины соединялись ребром, если соотвествующие копии имеют непустое пересечение.
Используя этот граф, Хата доказал критерий связности: самоподобное множество связно тогда и только тогда, когда его граф пересечений связен.

В дальнейшем К. Бандт и К. Келлер в работе \cite{SSS2} показали, что если у самоподобного множества копии пересекаются не более чем по одной точке и его граф пересечений есть дерево, то это множество является дендритом. Для дендритов, в которых по одной точке пересекались более двух копий, ими была сформулирована идея двудольного графа пересечений. 
Завершил доказательство этой идеи А. В. Тетенов в своём препринте \cite{FIP}, определив двудольный граф пересечений, в котором одной доле соответствуют копии аттрактора, а другой доле --- точки попарных пересечений этих копий. Ребром в графе могут соединятся только точки разных долей, если соответствующая точка пересечения лежит в соответствующей копии.
Таким образом был получено необходимое и достаточное того, что самоподобные множества с одноточечным пересечением являются дендритами.\\

Фрактальные квадраты, в свою очередь, являются давно известным классом самоподобных множеств.
Фрактальные квадраты имеют немало как более общих классов, например фракталные $k$-кубы, ковры Бедфорда-МакМаллена, ковры Баранского, губки Серпинского, так и смежные классы фрактальных треугольников и смешанных лабиринтных фракталов.

Некоторые примеры самоподобных фрактальных квадратов вроде множества Вичека и ковра Серпинского были построены и изучены довольно давно.
Тем не менее, с определённого момента многими людьми рассматривались обобщения фрактальных квадратов, причём как обобщения многомерные (трёхмерные фрактальные кубы), так и обобщения на более широкий класс сжимающих отображений (самоаффинные ковры Бедфорда-МакМаллена, ковры Баранского и губки Серпинского).

Так в 1984 году независимо друг от друга Т. Бедфорд \cite{Bedford1984} и К. МакМаллен \cite{McMullen1984} получили самоаффинное обобщение фрактальных квадратов и вывёли формулу для подсчёта размерности Хаусдорфа таких множеств.
Впоследствии эти множества стали называть коврами Бедфорда-МакМаллена.

Как оказалось, размерность и мера ковров Бедфорда-МакМаллена и их многомерных обобщений ведёт себя довольно непредсказуемо, особенно по сравнению со своими самоподобными частными случаями.
Так Ю. Перес \cite{Peres1994} в 1994 году показал, что мера Хаусдорфа у ковров Бедфорда-МакМаллена может быть не $\sigma$-конечной.
Позднее он в соавторстве с Р. Кеньёном \cite{KenyonPeres1996} вывел формулу размерности для губок Серпинского --- многомерных обобщений ковров Бедфорда-МакМаллена.
Самый подробный обзор и сравнение различных фрактальных размерностей (Хаусдорфа, клеточная, Асуада и др.) для губок Серпинского в 2021 году привёл Дж. Фрейзер \cite{Fraser_2021}.

Фрактальные квадраты оказались довольно богатыми на интересные задачи, которые многие учёные охотно решают.
Так в 2013 году К.-С. Лао, Дж. Дж. Луо и Х. Рао \cite{LLR2013} рассмотретли топологические свойства замощений плоскости фрактальными квадратами.
Л. Кристеа и Б. Штейнски выпустили цикл работ \cite{CS1,CS2,CS3}, посвящённый выделению и исследованию поддуг во фрактальных квадратах и смешанных лабиринтных фракталах.
Интересными являются и результаты Дж.-Ц. Сяо \cite{Xiao2021}, который строил и изучал несвязные фрактальные квадраты с конечным числом компонент, для чего разработал весьма интересную модификацию графа пересечений.


\begin{center}
\textbf{Выносимые на защиту положения.}
\end{center}

\begin{enumerate}
\item Найдено необходимое условие того, что аттрактор обобщённой полигональной системы является дендритом.

\item Доказано существование такого $\delta$, что при соблюдении условия совпадении параметров аттрактор любой $\delta$-деформации $\eS'$ стягиваемой $P$-полигональной системы $\eS$ будет дендритом, причём гомеоморфным аттрактору системы $\eS$.

%\item Для пары фрактальных $k$-кубов $K_1$ и $K_2$ порядка $n$ получена система множеств $\Sigma=\Sigma(K_1,K_2)=\{F_\bma=K_1\cap(K_2+\bma)\ :\ \bma\in\{-1,0,1\}^k\}$ попарных пересечений их противоположных граней, где $F_0=K_1\cap K_2$. 
%Оказалось, что $\Sigma$ является аттрактором граф-ориентированной системы подобий.
%На основе этой системы была доказана теорема о мощности множества $F_\bma$.
%Для несчётных множеств $F_\bma$ была получаена его размерность Хаусдорфа.
%Так же были получены условия, при которых множество $F_\bma$ имеет бесконечную счётную меру.

\item Было доказано, что фрактальные квадраты, являющиеся дендритами, бывают только со свойством одноточечного пересечения.

\item Было доказано, что у фрактальных квадратов, являющихся денритами, всего семь возможных топологических шаблонов главного дерева.
\end{enumerate}

\begin{center}
    \textbf{Содержание диссертации.}
\end{center}

Перейдём к описанию структуры работы и точным формулировкам основных результатов.
Диссертация выполнена в издательской системе \LaTeX, содержит \red{??} страниц и состоит из введения, трёх глав, заключения и списка литературы.
Каждая глава разбита на параграфы.
Список литературы приведён в алфавитном порядке.

\textbf{Первая глава} посвящена самоподобным дендритам, являющимися аттракторами стягиваемых полигональных систем.
В Главе я рассматриваю более общий класс полигональных систем и ищу условия, при которых их аттрактор будет дендритом. 

Рассмотрим самоподобные дендриты, являющиеся аттракторами стягиваемой $P$-полигональной системы.

\restate{dfn:sss}

\restate{dfn:den}

Тетенов А. В., Самюэль М. и Ваулин Д. А. в своих работах \cite{TSV2017, STV2017} ввели геометрически задаваемую систему $\eS$ сжимающих подобий, задаваемый с помощью многоугольника $P$:

\restate{dfn:cps}

Аттрактор $K=K(\eS)$ такой системы обладает многими интересными свойствами.
Прежде всего нам интересно то, что $\eS$ удовлетворяет условию открытого множества (в качестве открытого множества можем взять $\dot P$), копии его аттрактора $K$ пересекаются друг с другом не более чем по одной точке, а сам аттрактор является дендритом:

\restate{thm:cpsden}

Результаты этой главы относятся именно к обобщению таких систем, для этого мы ослабим требования, накладываемые на стягиваемые полигональные системы:

\restate{dfn:gps}

Прежде всего интерес представляют те обобщённые полигональные системы, которые являются $\delta$-деформацией стягиваемой полигональной системы:

\restate{dfn:deform}

%Стоит отметить, что обобщённая полигональная система вовсе не обязательно является деформацией какой-то стягиваемой полигональной системы.

Однако аттрактор $K$ обобщённой полигональной системы $\eS$ уже не обязательно будет дендритом, поэтому требуется для $\eS$ дополнительно проверить выполнение условия, задаваемого равенством \eqref{icnd}:

\restate{thm:pcint}

В таком случае аттрактор $K$ стягиваемой полигональной системы $\eS$ и аттрактор $K'$ её $\delta$-деформации $\eS'$ будут гомеоморфны:

\restate{thm:attrmap}

Значит нам нужно получить условия, при которых аттрактор обобщённой полигональной системы будет дендритом с одноточечным пересечением, то есть $S_i(K)\cap S_j(K)=P_i\cap P_j$.


Сперва пусть $\eS=\{S_1,\ldots,S_m\}$ -- обобщённая $P$-полигональная система, аттрактор $K$ которой --- дендрит.
Пусть также $A_i,A_j\in V_P$, тогда существует единственная дуга $\Gamma_{ij}\IN K$ с концами в этих точках.
Если $S_i(A_i)=A_i$, то существует дуга $\Gamma\IN\Gamma_{ij}$ с концом в $A_i$ такая, что $S_i(\Gamma)\IN\Gamma$.
Тогда мы говорим, что $\Gamma$ инвариантна относительно подобия $S_i$.

Теперь для любой точки $x=S_i(K)\cap S_j(K)=P_i\cap P_j$ существует пара таких дуг $\Gamma_i\IN K_i$ и $\Gamma_j\IN K_j$, что $\Gamma_i\cap\Gamma_i=x$.
Более того, для этих дуг существуют тображения $S_\bi, S_\bj$ такие, что $S_\bi(x)=S_\bj(x)=x$ и при этом $S_\bi(\Gamma_i)\IN\Gamma_i$ и $S_\bj(\Gamma_j)\IN\Gamma_j$, то есть дуги $\Gamma_i, \Gamma_j$ инвариантны относительно $S_\bi, S_\bj$ соответственно.

Необходимое условие того, что пара таких дуг будут пересекаться только по общему концу, даёт лемма о непересекающихся дугах \cite{ATK}:

{\bf Лемма.}
{\em Пусть $\Gamma_1$ и $\Gamma_2$ -- такие жордановы дуги с общим началом в $0$ и  концами $z_1$ и $z_2$ ($|z_1|=|z_2|=1$), что $\Gamma_1\cap\Gamma_2=\{0\}$.
Если сжимающие подобия $S_1(z)=\rho_1 e^{i\alpha}z$ и $S_2(z)=\rho_2 e^{i\beta}z$ таковы, что $S_1(\Gamma_1)\IN\Gamma_1$ и $S_2(\Gamma_2)\IN\Gamma_2$, 
а $\alpha:=\Delta\underset{{\Gamma_1\mmm S_1(\Gamma_1)}}{\mathrm {Arg}}(z)$ и $\beta:=\Delta\underset{{\Gamma_2\mmm S_2(\Gamma_2)}}{\mathrm {Arg}}(z)$, то
$$\lambda_1:=\dfrac{\alpha}{\ln\rho_1} = \lambda_2:=\dfrac{\beta}{\ln\rho_2}.$$}

Отношения $\lambda_1$ и $\lambda_1$ мы называем параметрами инвариантных дуг $\Gamma_1$ и $\Gamma_2$ относительно их общего конца.
Именно равенство этих параметров является необходимым условием одноточечного пересечения дуг, и мы можем его применить для формулировки необходимого условия того, что аттрактор обобщённой полигональной системы будет дендритом.

Мы будем говорить, что обобщённая полигональная система система $\eS$ удовлетворяет условию совпадения параметров, если для каждого $x=S_i(P)\cap S_j(P)$ (при $i\neq j$) все инвариантные дуги $\gamma_i\IN S_i(K)$ и $\gamma_j\IN S_j(K)$) с концом в $x$ имеют относительно $x$ одинаковые параметры.

Тогда первый основной результат этой Главы, {\bf Теорема \ref{PMT}} о совпадении параметров говорит следующее:\\
{\em
Пусть аттрактор $K$ обобщенной полигональной системы $\eS$ является дендритом. 
Тогда система $\eS$ удовлетворяет условию совпадения параметров.}

Эта теорема действительно даёт необходимое условие того, что аттрактор обобщённой полигональной системы, и, в частности, аттрактор $\da$-деформации, будет дендритом. 
Однако если стягиваемую полигональную систему деформировать слишком сильно, то даже при соблюдении условия совпадения параметров аттрактор $\da$-деформации может и не быть дендритом. 
Значит нам нужны ограничения при деформациях. 
Эти ограничения даёт наш второй основной результат в этой Главе:

{\bf Теорема \ref{mainthm}.} [Теорема о малых деформациях]
{\em Для каждой полигональной системы $\eS$ существует такое $\delta > 0$, что для всякой её $\delta$-деформации $\eS'$, удовлетворяющей условию совпадения параметров, аттрактор $K(\eS')$ является дендритом, гомеоморфным $K(\eS)$.}\\


Во {\bf второй главе} я рассматриваю фрактальные квдраты, являющиеся дендритами, и их главные деревья.


Критическое множество\\
Самоподобная граница\\
Главное дерево\\

Фрактальный квадрат, его грани\\
Одноточечное пересечение копий фрактального квадрата\\
Основной результат: Критерий дендритности фрактального квадрата\\

Самоподобная граница фрактального квадрата, являющегося дендритом: типы.\\
Порядок точки ветвления фрактального квадрата и его углов\\
порядок ветвления точек главного дерева для ФК типа D\\
Основной результат: теорема о классификации фрактальных квадратов, являющихся дендритами











\begin{center}
\textbf{Апробация результатов.}
\end{center}


\begin{center}
\textbf{Благодарности}
\end{center}



% \textbf{Структура работы.}
% \textbf{Объем и структура работы.} Диссертация состоит из~введения,
% \formbytotal{totalchapter}{глав}{ы}{}{},
% заключения и
% \formbytotal{totalappendix}{приложен}{ия}{ий}{}.
% %% на случай ошибок оставляю исходный кусок на месте, закомментированным
% %Полный объём диссертации составляет  \ref*{TotPages}~страницу
% %с~\totalfigures{}~рисунками и~\totaltables{}~таблицами. Список литературы
% %содержит \total{citenum}~наименований.
% %
% Полный объём диссертации составляет
% \formbytotal{TotPages}{страниц}{у}{ы}{}, включая
% \formbytotal{totalcount@figure}{рисун}{ок}{ка}{ков} и
% \formbytotal{totalcount@table}{таблиц}{у}{ы}{}.
% Список литературы содержит
% \formbytotal{citenum}{наименован}{ие}{ия}{ий}.

