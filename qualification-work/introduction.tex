% \newpage
\chapter*{Введение}                         % Заголовок
\addcontentsline{toc}{chapter}{Введение}    % Добавляем его в оглавление

% \newcommand{\actuality}{}
% \newcommand{\progress}{}
% \newcommand{\aim}{{\textbf\aimTXT}}
% \newcommand{\tasks}{\textbf{\tasksTXT}}
% \newcommand{\novelty}{\textbf{\noveltyTXT}}
% \newcommand{\influence}{\textbf{\influenceTXT}}
% \newcommand{\methods}{\textbf{\methodsTXT}}
% \newcommand{\defpositions}{\textbf{\defpositionsTXT}}
% \newcommand{\reliability}{\textbf{\reliabilityTXT}}
% \newcommand{\probation}{\textbf{\probationTXT}}
% \newcommand{\contribution}{\textbf{\contributionTXT}}
% \newcommand{\publications}{\textbf{\publicationsTXT}}

% \input{common/characteristic} % Характеристика работы по структуре во введении и в автореферате не отличается (ГОСТ Р 7.0.11, пункты 5.3.1 и 9.2.1), потому её загружаем из одного и того же внешнего файла, предварительно задав форму выделения некоторым параметрам

\begin{center}
\textbf{История вопроса и основные направления.}
\end{center}
% история вопроса
{\bf Самоподобное множество.}

{\bf Тут добавить историю фракталов до 1970 года}


На практике строго определить фрактал не так просто.
К.~Фальконер \cite{Falconer2004} даёт несколько необязательных признаков, котрорым могут удовлетворять фракталы.
Так для того чтобы можно было назвать объект $A$ фракталом, он должен характеризоваться какими-либо из следующих свойств:

\begin{enumerate}
\item $A$ имеет тонкую структуру, т. е. содержит сложные структурные элементы на любых масштабах;
\item $A$ слишком неоднородно, чтобы описываться на традиционном геометрическом языке;
\item $A$ самоподобно в том или ином смысле, т. е. имеет повторяющуюся структуру в разных масштабах. Возможно, самоподобие приблизительное или статистическое.
\item Каким-то образом определенная <<фрактальная>> размерность множества $A$ превышает его топологическую размерность и зачастую является дробным числом;
\item $A$ можно построить через рекурсивные или итеративные схемы (что позволяет моделировать фракталы на компьютерах).
\end{enumerate}


Важным разделом фрактальной геометрии является теория самоподобных множеств.
На протяжении всей работы мы будем рассматривать именно самоподобные множества.
Хотя понятие самоподобия упоминается довольно давно, например у П. Леви в 1939 году \cite{Levy1939} при описании его знаменитой кривой Леви, но начало теории самоподобных множеств положил Дж. Хатчинсон в 1981 г. \cite{Hut1981}.
Он дал строгое определение самоподобного множества, состоящего из уменьшеных образов самого себя, и описал четкий математический подход к исследованию таких множеств. 
Работа Хатчинсона послужила основой для множества дальнейших исследований.
%, многие из которых расширяют и обобщают класс самоподобных множеств.

Прежде всего стоит отметить вклад Р. Молдина и С. Вильямса \cite{MW1988}, которые разработали концепцию граф-ориентированных систем подобий, аттрактором которых является уже система компактов, каждый из которых может состаять не только из своих копий, но и из копий других компактов системы.
% Данная концепция перекликается с идеей самоподобных множеств с конденсацией, введённых \red{???}.

В 1996 году М. Моран \cite{Moran1996} определил бесконечно порождённые самоподобные множества и рассмотрел их свойства.
Эти множества имеют свои особенности при вычислении размерности.
% и в целом значительно расширяют класс стандартных самоподобных множеств.

В теории самоподобных множеств часто возникает вопрос о вычислении их фрактальной размерности, например размерности Хаусдорфа.
Вычисление размерности самоподобных множеств напрямую связано со структурой пересечений их копий.
Это, в свою очередь, тесно связано с условиями отделимости порождающих эти фракталы систем сжимающих подобий, такими как условие открытого множества (OSC) и слабое условие отделимости (WSP).
Множества, не удовлетворяющие никаким стандартным условиям отделимости, могут быть весьма сложными.

%Прежде всего стоит отметить вклад \red{кого???}, описавших строгое условие отделимости (SSC).
%Мы говорим, что самоподобное множество удовлетворяет строгому условию отделимости, если его копии попарно друг с другом не пересекаются.
%Однако это условие слишком сильное и для большинства множеств оно не подходит.

П.~Моран в 1946 г. \cite{Moran1946} ввел условие открытого множества (OSC) для самоподобных множеств на прямой, а Дж.~Хатчинсон \cite{Hut1981} применил введенное Мораном условие открытого множества к системам сжимающих подобий в $\rr^n$ для любого натурального $n$.

Мы говорим, что система сжимающих подобий $\eS=\{S_1, \ldots, S_m\}$ удовлетворяет условию открытого множества, если существует открытое множество $O$ такое, что множества $\{O_i=S_i(O) | S_i\in\eS\}$ содержатся в $O$ и попарно друг с другом не пересекаются.
%Так если система $\eS$ удовлетворяtn OSC, то размерность Хаусдорфа её аттрактора $K(\eS)$ равна размерности подобия, которая легко вычисляется.
Если система $\eS=\{S_1,\ldots,S_m\}$ сжимающих подобий в $\rr^n$ с коэффициентами подобия $r_1, \ldots, r_m$ удовлетворяет условию открытого множества, то хаусдорова размерность аттрактора этой системы равна его размерности подобия $s$, которая является решением следующего уравнения: $$r_1^s+\ldots+r_m^s=1.$$ 

Условие открытого множества справедлива далеко не для всех случаев.
Есть примеры самоподобных множеств с копиями, пересекающимися по множеству со столь сложной структурой, что вопрос существования подходящего открытого множества не является очевидным.
%Иногда подходящим открытым множеством является внутренность самого фрактала, что не является удобным для проверки. 
Порой открытое множество может иметь очень сложную стуктуру и состоять из бесконечного числа связных компонент.

К.~Бандт и З.~Граф в 1992 году \cite{SSS7} искали алгебраический аналог для OSC и ввели алгебраическое условие, основанное на ассоциированном семействе подобий $\mathcal{F}(\eS)$. 
Они показали, что это алгебраическое условие эквивалентно тому, что для аттрактора $K$ с размерностью подобия $s$ его $s$-мерная мера Хаусдорфа положительна:
$$\Id\notin\overline{\mathcal{F}(\eS)} \Leftrightarrow H^s(K)>0.$$
Это также означает, что копии такого самоподобного множества попарно пересекаются по множеству нулевой меры.

Условие открытого множества можно усилить следующим требованием: открытое множество $O$ и аттрактор $K$ системы $\eS$ имеют непустое пересечение. 
Так получается сильное условие открытого множества (SOSC).
В 1994 г. А.~Шиф \cite{Schief1994} показал, что все три условия: SOSC, OSC и условие положительности меры Хаусдорфа в размерности подобия --- эквивалентны.

Условие открытого множетсва позволяет вычислять размерность Хаусдорфа с помощью размерности подобия, поскольку копии самоподобного множества пересекаются не слишком сильно или вовсе не пересекаются.
Самоподобное множество, копии которого пересекаются либо по своим подкопиям, либо по множеству нулевой меры, удовлетворяет слабому условию отделимости (WSP), определённому М.~Цернером \cite{Zerner1996}.
Для самоподобных множеств, удовлетворяющих WSP, можно модифицировать формулу размерности подобия и с её помощью вычислить размерность Хаусдорфа.
Тем не менее, для систем сжимающих подобий, не удовлетворяющих WSP, вычисление размерности Хаусдорфа их аттракторов зачастую может быть действительно сложной проблемой.\\

{\bf Дендриты.}
Далее поговорим про самоподобные дендриты, которые представляют интерес для теории самоподобных множеств.
Дендритом называют локально связный континуум, не содержащий простых замкнутых дуг.
Дендриты в течении многих лет рассмативались в общей топологии \cite{Kur1, Kur2}, а Я. Харатоник и В. Харатоник дают в своей работе \cite{Char1998} исчерпывающий обзор, охватывающий более чем 75 лет исследований в этой области.

В теории самоподобных множеств с самого начала предпринимались попытки выработать некоторые подходы к самоподобным дендритам.
Так в 1985 году М. Хата \cite{Hata1985} помимо других базовых свойств дендритов показал, что самоподобный дендрит имеет бесконечное множество концевых точек.
В 1990 году К. Бандт показал в \cite{SSS6}, что жордановы дуги, соединяющие пары точек самоподобной границы в посткритически конечном самоподобном множестве, являются самоподобными, а множество возможных значений размерностей таких дуг конечено.
Применяя эти результаты к дендритам, мы получим для каждой пары точек самоподобной границы единственную соединияющцю их дугу.
Он также рассмотрел факторизацию индексного пространства, приводящую к появлению дендритов в \cite{SSS2}.
Дж. Кигами в своей работе \cite{Kig95} применил методы гармонического анализа на фракталах к дендритам. 

Были построены многие отдельные представители самоподобных дендритов, например дерево Хаты \cite{Hata1985}, множество Вичека или пентадендрит \cite{McWorter1987}.
Тем не менее, долгое время отсутствовали удобные геометрические методы, позволяющие строить системы сжимающих подобий, аттракторы которых являлись бы дендритами.
Так было, пока А. В. Тетенов, М. Самуэль и Д. А. Ваулин в статье \cite{TSV2017} не описали методы задания и геометрические свойства самоподобных дендритов в $\rr^d$ --- вопросы, до 2017 года еще не достаточно разработанные в теории самоподобных фракталов. 
Для этого строился и исследовался класс $P$-полиэдральных дендритов в $\rr^d$. 
Такие дендриты $K$ определяются как аттракторы систем $\eS = {S_1,\ldots, S_m}$ сжимающих подобий в $\rr^d$, переводящих заданный полиэдр $P \IN \rr^d$ в полиэдры $P_i \IN P$, попарные пересечения которых либо пусты, либо одноточечны и являются общими вершинами этих полиэдров, а граф попарных пересечений системы полиэдров $P_i$ ацикличен.
Этими же авторами в том же году в работе \cite{STV2017} были более подробно изучены стягиваемые $P$-полигональные системы --- двумерный частный случай $P$-полиэдральных систем.
Была показана возможность изоморфизма между аттракторами двух разных полигональных систем.
Тем не менее, оставался вопрос о возможности получения более широкого класса дендритов путём ослабления условий, задающих стягиваемые $P$-полигональные ситстемы. 
Исследование такого класса дендритов является одной из целей данной работы.

Говоря о полигональных системах нельзя не упомянуть о полигаскетах, описанных К. Бандтом и Дж. Штанком в работе \cite{SSS6} и Р. Стритчартсом в работах \cite{strich1999, Strichartz1999}, которые хоть и не являются дендритами, но для их построения использовались схожие геометрические методы.
Для полигаскетов им также были описаны кратчайшие дуги, соединяющие пару точек полигаскета и имеющих минимальную размерность и меру.
Кратчайшие дуги позднее будут применены в вышеупомянутых работах \cite{TSV2017, STV2017} А. В. Тетенова, М. Самуэль и Д. А. Ваулина для построения главных дуг и главного дерева самоподобного дендрита, являющегося аттрактором полигональных систем.

Проверка того, является ли самоподобное множество дендритом, связана со структурой попарных пересечения копий этого аттрактора.
Начать следует с результатов М. Хаты \cite{Hata1985}, который в 1985 году доказал для самоподобного множества критерий связности.
Мы будем использовать следующую эквивалентную интерпретацию этого критерия.
Пусть для самоподобного множества дан его граф пресечений, в котором вершинам графа соответствую копии самоподобного множества и эти вершины соединялись ребром, если соотвествующие копии имеют непустое пересечение.
Самоподобное множество связно тогда и только тогда, когда его граф пересечений связен.
Такое связное самоподобное множество также локально связно и линейно связно.

В дальнейшем К. Бандт и К. Келлер в работе \cite{SSS2} показали, что если у самоподобного множества копии пересекаются не более чем по одной точке и его граф пересечений есть дерево, то это множество является дендритом. Для дендритов, в которых по одной точке пересекались более двух копий, ими была сформулирована идея двудольного графа пересечений. 
Завершил доказательство этой идеи А. В. Тетенов в своём препринте \cite{FIP}, определив двудольный граф пересечений, в котором одной доле соответствуют копии аттрактора, а другой доле --- точки попарных пересечений этих копий. Ребром в графе могут соединятся только точки разных долей, если соответствующая точка пересечения лежит в соответствующей копии.
Таким образом был получено необходимое и достаточное того, что самоподобные множества с одноточечным пересечением являются дендритами.\\

{\bf Фрактальные квадраты}

Пусть нам дан единичный квадрат на лоскости.
Разобьём этот квадрат на $n^2$ маленьких квадратов с ребром $1/n$, и в этом множестве маленьких квадратов выберем какое-то непустое подмножество.
Аттрактор системы гомотетий, переводящих единичный квадрат в выбранные маленькие квадраты, мы и будем называть {\em фрактальным квадратом}.
Некоторые примеры фрактальных квадратов, такие как множество Вичека и ковёр Серпинского, были построены довольно давно.

Вообще фактальные квадраты являются самоподобным частным случаем самоаффинных {\em ковров Бедфорда-МакМаллена}, которые в свою очередь являются двумерным частным случаем {\em губок Серпинского}.
Самоподобный частный случай губок Серпинского называют фрактальным кубом.

%Фрактальные квадраты, в свою очередь, являются давно известным классом самоподобных множеств.
%Фрактальные квадраты имеют немало как более общих классов, например фракталные $k$-кубы, ковры Бедфорда-МакМаллена, ковры Баранского, губки Серпинского, так и смежные классы фрактальных треугольников и смешанных лабиринтных фракталов.


%Тем не менее, с определённого момента многими людьми рассматривались обобщения фрактальных квадратов, причём как обобщения многомерные (трёхмерные фрактальные кубы), так и обобщения на более широкий класс сжимающих отображений (самоаффинные ковры Бедфорда-МакМаллена, ковры Баранского и губки Серпинского).

Так в 1984 году независимо друг от друга Т. Бедфорд \cite{Bedford1984} и К. МакМаллен \cite{McMullen1984} определили и рассмотрели класс самоаффинных множеств, которые впоследствии стали называть коврами Бедфорда-МакМаллена.
Одним из результатов их исследований стала формула для подсчёта размерности Хаусдорфа таких множеств.

Как оказалось, размерность и мера ковров Бедфорда-МакМаллена и их многомерных аналогов ведёт себя довольно непредсказуемо, особенно по сравнению со своими самоподобными частными случаями.
Так Ю. Перес \cite{Peres1994} в 1994 году показал, что мера Хаусдорфа (в своей размерности) у ковров Бедфорда-МакМаллена может быть не $\sigma$-конечной.
Позднее Перес в соавторстве с Р. Кеньоном \cite{KenyonPeres1996} вывел формулу размерности Хаусдорфа для губок Серпинского.
Подробный обзор и сравнение различных фрактальных размерностей (Хаусдорфа, клеточная, Ассуада и др.) для губок Серпинского в 2021 году привёл Дж. Фрейзер \cite{Fraser_2021}.

На примере фрактальных квадратов формулируются и решаются многие интересные задачи. 
Так в 2013 году К.-С. Лао, Дж. Дж. Луо и Х. Рао \cite{LLR2013} рассмотрели топологические свойства замощений плоскости фрактальными квадратами.
Л. Кристеа и Б. Штейнски выпустили цикл работ \cite{CS1,CS2,CS3}, посвящённый выделению и исследованию поддуг во фрактальных квадратах и фрактальных треугольниках.
Далее, Дж.-Ц. Сяо \cite{Xiao2021} строил и изучал несвязные фрактальные квадраты с конечным числом компонент, для этого он особым образом модифицировал граф пересечений.


\begin{center}
\textbf{Выносимые на защиту положения.}
\end{center}

\begin{enumerate}
\item Найдено необходимое условие того, что аттрактор обобщённой полигональной системы является дендритом.

\item Доказано существование такого $\delta$, что при соблюдении условия совпадении параметров аттрактор любой $\delta$-деформации $\eS'$ стягиваемой $P$-полигональной системы $\eS$ будет дендритом, изоморфным аттрактору системы $\eS$.

%\item Для пары фрактальных $k$-кубов $K_1$ и $K_2$ порядка $n$ получена система множеств $\Sigma=\Sigma(K_1,K_2)=\{F_\bma=K_1\cap(K_2+\bma)\ :\ \bma\in\{-1,0,1\}^k\}$ попарных пересечений их противоположных граней, где $F_0=K_1\cap K_2$. 
%Оказалось, что $\Sigma$ является аттрактором граф-ориентированной системы подобий.
%На основе этой системы была доказана теорема о мощности множества $F_\bma$.
%Для несчётных множеств $F_\bma$ была получаена его размерность Хаусдорфа.
%Так же были получены условия, при которых множество $F_\bma$ имеет бесконечную счётную меру.

\item Было доказано, что фрактальные квадраты, являющиеся дендритами, обладют свойством одноточечного пересечения.

\item Было доказано, что у фрактальных квадратов, являющихся денритами, всего семь возможных топологических типов главного дерева.
\end{enumerate}

\begin{center}
    \textbf{Содержание диссертации.}
\end{center}

Перейдём к описанию структуры работы и точным формулировкам основных результатов.
Диссертация выполнена в издательской системе \LaTeX, содержит \red{??} страниц и состоит из введения, трёх глав, заключения и списка литературы.
Каждая глава разбита на параграфы.
Список литературы приведён в алфавитном порядке.


\textbf{Первая глава} посвящена базовым понятиям теории самоподобных множеств.
Сперва я определяю самоподобное множество и расматриваю его размерность и связность. 

%Рассмотрим самоподобные дендриты, являющиеся аттракторами стягиваемой $P$-полигональной системы.
\restate{dfn:sss}

{\em Критическое множество} аттрактора $K$ системы $\mathcal{S}$ --- это множество $C:=\{x:\; x\in S_i(K)\cap S_j(K),\; S_i, S_j\in\eS\}$ точек попарных пересечений копий $K$. 
Множество $\dd K$ всех $x\in K$ таких, что для некоторого $\bj\in I^*$ верно $S_\bj(x)\in C$, называется {\em самоподобной границей} множества $K$.
Говоря иначе, самоподобная граница ---



\restate{dfn:den}

Простой замкнутой дугой мы называем непрерывный иньективный образ окружности.
Дедриты обладают некоторыми примечательными свойствами:
\begin{enumerate}[nolistsep]
\item[1.] Любые две точки дендрита можно соединить единственной лежащей в этом дендрите жордановой дугой;
\item[2.] Любое связное подмножество дендрита само является дендритом;
\item[3.] Пересечение любых связных подмножеств дендрита связно;
%\item[4.] ...
\end{enumerate}

Если $K$ --- самоподобный дендрит, то любую пару точек $a_i,a_j\in\dd K$ из его самоподобной границы можно соединить единственной жордановой дугой $\gamma(a_i,a_j)\IN K$.
Такие дуги, соединяющие пары точек самоподобной границы дендриты, мы будем называть {\em главными дугами}.
Если самоподобная граница $\dd K$ самоподобного дендрита конечна, то множество всех главных дуг тоже будет конечно.
Объединение всех главных дуг будем называть {\em главным деревом}.

\noindent{\bf Определение \ref{dfn:MT}.}
{\em Пусть $K$ --- самоподобный дендрит c конечной самоподобной границей $\dd K$. 
Минимальный поддендрит $\hat\gamma\IN K$, содержащий $\dd K$, называется {\em главным деревом} дендрита $K$.}\\

Порядком ветвления $Ord(x,K)$ точки $x$ в дендрите $K$ мы называем число компонент дополнения этой точки.
Точку порядка ветвления 1 мы называем концом, точку с порядком ветвления 2 --- разбивающей, а точку с порядком ветвления не менее 3 --- точкой ветвления.
Заметим, что концами главного дерева $\hat\gamma$ самоподобного дендрита $K$ могут быть только точки из самоподобной границы $\dd K$.
При этом не все точки самоподобной границы $\dd K$ должны быть концами главного дерева $\hat\gamma$.\\

Если в самоподобном множестве его копии попарно пересекаются не более чем по одной точке, то мы говорим, что такое множество обладает свойством одноточечного пересечения.
Для нас интерес представляют как раз самоподобные дендриты с одноточечным пересечением.
Первым удобным класом таких множеств являются аттракторы стягиваемых полигональных систем.

Тетенов А. В., Самюэль М. и Ваулин Д. А. в своих работах \cite{TSV2017, STV2017} определили с помощью многоугольника $P$ систему $\eS$ сжимающих подобий следующим образом:

\restate{dfn:cps}

Полигональная система $\eS$ удовлетворяет условию открытого множества (в качестве открытого множества можем взять $\dot P$), копии аттрактора $K(\eS)$ пересекаются друг с другом не более чем по одной точке, а сам аттрактор является дендритом:

\restate{thm:cpsden}

Среди свойств таких $P$-полигональных дендритов стоит отдельно выделить следующие:
\begin{enumerate}
\item[1.] $P$-полигональный дендрит $K$ содержится в многоугольнике $P$;
\item[2.] Самоподобная граница $\dd K$ у $P$-полигонального дендрита $K$ совпадает с множеством вершин $\eV_P$;
\item[3.]
\end{enumerate}


Во \textbf{второй главе} я получаю более широкий класс полигональных дендритов. 
Для этого мы сперва ослабим требования, накладываемые на стягиваемые полигональные системы:

\restate{dfn:gps}

Прежде всего стоит рассмотреть те обобщённые полигональные системы, которые являются $\delta$-деформацией стягиваемой полигональной системы:

\restate{dfn:deform}

%Стоит отметить, что обобщённая полигональная система вовсе не обязательно является деформацией какой-то стягиваемой полигональной системы.

Однако аттрактор $K$ обобщённой полигональной системы $\eS$ уже не обязательно будет дендритом, поэтому требуется для $\eS$ дополнительно проверить выполнение условия, задаваемого равенством \eqref{icnd}:

\restate{thm:pcint}

В таком случае аттрактор $K$ стягиваемой полигональной системы $\eS$ и аттрактор $K'$ её $\delta$-деформации $\eS'$ будут изоморфны:

\restate{thm:attrmap}

Значит нам нужно получить условия, при которых аттрактор обобщённой полигональной системы будет дендритом с одноточечным пересечением, то есть $S_i(K)\cap S_j(K)=P_i\cap P_j$.


Сперва пусть $\eS=\{S_1,\ldots,S_m\}$ -- обобщённая $P$-полигональная система, аттрактор $K$ которой --- дендрит.
Пусть также $A_i,A_j\in V_P$, тогда существует единственная дуга $\Gamma_{ij}\IN K$ с концами в этих точках.
Если $S_i(A_i)=A_i$, то существует дуга $\Gamma\IN\Gamma_{ij}$ с концом в $A_i$ такая, что $S_i(\Gamma)\IN\Gamma$.
Тогда мы говорим, что $\Gamma$ инвариантна относительно подобия $S_i$.

Теперь для любой точки $x=S_i(K)\cap S_j(K)=P_i\cap P_j$ существует пара таких дуг $\Gamma_i\IN K_i$ и $\Gamma_j\IN K_j$, что $\Gamma_i\cap\Gamma_i=x$.
Более того, для этих дуг существуют тображения $S_\bi, S_\bj$ такие, что $S_\bi(x)=S_\bj(x)=x$ и при этом $S_\bi(\Gamma_i)\IN\Gamma_i$ и $S_\bj(\Gamma_j)\IN\Gamma_j$, то есть дуги $\Gamma_i, \Gamma_j$ инвариантны относительно $S_\bi, S_\bj$ соответственно.

Необходимое условие того, что пара таких дуг будут пересекаться только по общему концу, даёт лемма о непересекающихся дугах \cite{ATK}:

{\bf Лемма.}
{\em Пусть $\Gamma_1$ и $\Gamma_2$ -- такие жордановы дуги с общим началом в $0$ и  концами $z_1$ и $z_2$ ($|z_1|=|z_2|=1$), что $\Gamma_1\cap\Gamma_2=\{0\}$.
Если сжимающие подобия $S_1(z)=\rho_1 e^{i\alpha}z$ и $S_2(z)=\rho_2 e^{i\beta}z$ таковы, что $S_1(\Gamma_1)\IN\Gamma_1$ и $S_2(\Gamma_2)\IN\Gamma_2$, 
а $\alpha:=\Delta\underset{{\Gamma_1\mmm S_1(\Gamma_1)}}{\mathrm {Arg}}(z)$ и $\beta:=\Delta\underset{{\Gamma_2\mmm S_2(\Gamma_2)}}{\mathrm {Arg}}(z)$, то
$$\lambda_1:=\dfrac{\alpha}{\ln\rho_1} = \lambda_2:=\dfrac{\beta}{\ln\rho_2}.$$}

Отношения $\lambda_1$ и $\lambda_1$ мы называем параметрами инвариантных дуг $\Gamma_1$ и $\Gamma_2$ относительно их общего конца.
Именно равенство этих параметров является необходимым условием одноточечного пересечения дуг, и мы можем его применить для формулировки необходимого условия того, что аттрактор обобщённой полигональной системы будет дендритом.

Мы будем говорить, что обобщённая полигональная система система $\eS$ удовлетворяет условию совпадения параметров, если для каждого $x=S_i(P)\cap S_j(P)$ (при $i\neq j$) все инвариантные дуги $\gamma_i\IN S_i(K)$ и $\gamma_j\IN S_j(K)$) с концом в $x$ имеют относительно $x$ одинаковые параметры.

Тогда первый основной результат этой Главы можно сформулировать следующим образом:\\

\noindent{\bf Теорема \ref{PMT}} (о совпадении параметров). 
{\em
Пусть аттрактор $K$ обобщенной полигональной системы $\eS$ является дендритом. 
Тогда система $\eS$ удовлетворяет условию совпадения параметров.}

Эта теорема даёт необходимое условие того, что аттрактор обобщённой полигональной системы, в том числе $\da$-деформации, будет дендритом. 
Однако даже при соблюдении условия совпадения параметров аттрактор $\da$-деформации может и не быть дендритом, ведь для этого нам нужно, чтобы $\da$ было достаточно малым. 
Эту оценку для $\da$ даёт наш второй основной результат этой Главы:\\

\noindent{\bf Теорема \ref{mainthm}.} (о малых деформациях)
{\em Для каждой полигональной системы $\eS$ существует такое $\delta > 0$, что для всякой её $\delta$-деформации $\eS'$, удовлетворяющей условию совпадения параметров, аттрактор $K(\eS')$ является дендритом, гомеоморфным $K(\eS)$.}\\


Во {\bf третьей главе} я рассматриваю главные деревья фрактальных квадратов, являющихся дендритами.
В истории вопроса было дано описание фрактального квадрата, а более формально его можно определить так:\\

\noindent{\bf Определение \ref{dfn:FS}.}
{\em Пусть $D=\{d_1,\ldots,d_m\}\IN\{0,1,\ldots,n-1\}^2$, где $n\ge 2$, и $1<m<n^2$. 
{\em Фрактальным квадратом} порядка $n$ с {\em множеством единиц $D$} называют компактное множество $K\IN\rr^2$, удовлетворяющее уравнению
$$K=\dfrac{K+D}{n}=\bigcup_{d_i\in D}\dfrac{d_i+K}{n}.$$}

Если у фрактальных квадратов, являющихся дендритам, конечная самоподобная граница, то главные деревья таких дендритов имеют конечное число концов.
Тогда мы можем попробовать перечислить все топологические типы главных деревьев.
Это позволит разбить все фрактальные квадраты на конечное число классов согласно форме главного дерева.
Значит сперва нам нужно рассмотреть структуру самоподобной границы фрактальных квадратов, являющихся дендритами.

Поскольку 



%Фрактальный квадрат, его грани\\
%Одноточечное пересечение копий фрактального квадрата\\
%Основной результат: Критерий дендритности фрактального квадрата\\
%
%Самоподобная граница фрактального квадрата, являющегося дендритом: типы.\\
%Порядок точки ветвления фрактального квадрата и его углов\\
%порядок ветвления точек главного дерева для ФК типа D\\
%Основной результат: теорема о классификации фрактальных квадратов, являющихся дендритами











\begin{center}
\textbf{Апробация результатов.}
\end{center}


\begin{center}
\textbf{Благодарности}
\end{center}



% \textbf{Структура работы.}
% \textbf{Объем и структура работы.} Диссертация состоит из~введения,
% \formbytotal{totalchapter}{глав}{ы}{}{},
% заключения и
% \formbytotal{totalappendix}{приложен}{ия}{ий}{}.
% %% на случай ошибок оставляю исходный кусок на месте, закомментированным
% %Полный объём диссертации составляет  \ref*{TotPages}~страницу
% %с~\totalfigures{}~рисунками и~\totaltables{}~таблицами. Список литературы
% %содержит \total{citenum}~наименований.
% %
% Полный объём диссертации составляет
% \formbytotal{TotPages}{страниц}{у}{ы}{}, включая
% \formbytotal{totalcount@figure}{рисун}{ок}{ка}{ков} и
% \formbytotal{totalcount@table}{таблиц}{у}{ы}{}.
% Список литературы содержит
% \formbytotal{citenum}{наименован}{ие}{ия}{ий}.

