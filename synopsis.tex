%&preformat-synopsis
\RequirePackage[l2tabu,orthodox]{nag} % Раскомментировав, можно в логе получать рекомендации относительно правильного использования пакетов и предупреждения об устаревших и нерекомендуемых пакетах

% Откомментируйте, чтобы отключить генерацию закладок в pdf
% \PassOptionsToPackage{bookmarks=false}{hyperref}
\documentclass[a5paper,10pt,twoside,openany,article]{memoir} %,draft

\input{common/setup}          % общие настройки шаблона
%%% Кодировка и локализация %%%
\usepackage[utf8]{inputenc} % кодовая страница документа
\usepackage[T2A]{fontenc} % внутренняя кодировка  TeX
\usepackage[english,russian]{babel} % локализация и переносы


%%% Гипперссылки и pdf %%%
\usepackage{cmap} % русский поиск в pdf
%\usepackage{refcheck} % 
\usepackage[pagebackref]{hyperref} % гиперссылки
% для pagebackref после каждого bibitem нужна пустая строка или \par
\hypersetup{
    unicode, % корректная работа ссылок с кириллицей
    psdextra, % частично корректная работа ссылок с формулами (не все, ещё есть проблемы с math shift, subscript и superscript)
    pdftitle={Анализ на самоподобных множествах с конечным пересечением}, % название документа
    pdfsubject={Диссертация аспиранта}, % тема документа
    pdfauthor={Дроздов Дмитрий Алексеевич}, % авторы документа
    pdfkeywords={Фрактальный квадрат, фрактальный куб, самоподоное множество}, % список ключевых слов,
    colorlinks, % текст гиперссылки будет выделяться цветом, а рамки не будет
    linkcolor=black, % цвет текста ссылок на мишени внутри документа
    pagecolor=black, % цвет текста ссылок на страницы
    filecolor=cyan, % цвет текста ссылок на локальные PDF файлы
    citecolor=black, % цвет библиографических ссылок команды \cite
    urlcolor=black, % цвет текста ссылок на ресурсы с URL
    % draft, % отключить все гипертекстовые опции
}


%%% Пакеты для формул %%%
\usepackage{amsmath} % удобная вёрстка многострочных формул, масштабирующийся текст в формулах, формулы в рамках и др.
\usepackage{amssymb} % несколько сотен дополнительных математических символов
\usepackage{amsthm} % окружения «теорема», «лемма» и т. п.
\usepackage{amsfonts} % Ажурный \mathbb{} и готический \mathfrak{} шрифты
\usepackage{mathrsfs} % Mathematical Script letters (шрифт Эйлера) \mathscr{}
\usepackage{euscript} % Шрифт Евклида \EuScript{}
% каллиграфический шрифт \mathcal{} не требует пакета
\usepackage{enumitem}
\usepackage{bm}
\usepackage{multicol}
% \usepackage{cite}     % оптимизирует списки цитирования
\usepackage{lastpage}

%%% Графика, изображения и цвета %%%
\usepackage[usenames]{color} % позволяет задавать цвет текста и фона, как отдельного блока, так и всего документа
\usepackage{graphicx} % Работа с графикой \includegraphics{}
\usepackage{float} % Фиксация плавабщих боксов [H]
\graphicspath{
    {./images/}, 
    {./images/img1/}, 
    {./images/img2/}, 
    {./images/img3/},
    {./images/img4/}, 
    {./images/tikz1/}, 
    {./images/tikz2/}, 
    {./images/tikz3/},
    {./images/tree/}
    }
% Вставка tikz-картинок с помощью \includestandalone[width=\textwidth]{images/tikz1/name.tex}
% requires -shell-escape
% \usepackage[mode=buildnew]{standalone} 
% \usepackage{tikz}
%     \usetikzlibrary{decorations.markings}
%     \usetikzlibrary{arrows}
%     \usetikzlibrary{arrows.meta}
%     \usetikzlibrary{cd}
%     \usetikzlibrary{calc}


%\usepackage[parentracker=true,
%             backend=biber,
%             hyperref=false,
%             bibencoding=utf8,
%             style=numeric-comp,
%             language=auto,
%             autolang=other,
%             citestyle=gost-numeric,
%             defernumbers=true,
%             bibstyle=gost-numeric,
%             sorting=ntvy,
%             ]{biblatex}
%\addbibresource{dissbib.bib}


       % Пакеты общие для диссертации и автореферата
%%% Русская традиция начертания математических знаков
\renewcommand{\le}  {\ensuremath{\leqslant}}
\renewcommand{\leq} {\ensuremath{\leqslant}}
\renewcommand{\ge}  {\ensuremath{\geqslant}}
\renewcommand{\geq} {\ensuremath{\geqslant}}
\renewcommand{\emptyset}{\varnothing}

%%% Русская традиция начертания математических функций (на случай копирования из зарубежных источников)
\renewcommand{\tan}{\operatorname{tg}}
\renewcommand{\cot}{\operatorname{ctg}}
\renewcommand{\csc}{\operatorname{cosec}}

%%% Русская традиция начертания греческих букв (греческие буквы вертикальные, через пакет upgreek)
% \usepackage{upgreek} % прямые греческие ради русской традиции
% \renewcommand{\epsilon}{\ensuremath{\upvarepsilon}}   %  русская традиция записи
% \renewcommand{\phi}{\ensuremath{\upvarphi}}
% %\renewcommand{\kappa}{\ensuremath{\varkappa}}
% \renewcommand{\alpha}{\upalpha}
% \renewcommand{\beta}{\upbeta}
% \renewcommand{\gamma}{\upgamma}
% \renewcommand{\delta}{\updelta}
% \renewcommand{\varepsilon}{\upvarepsilon}
% \renewcommand{\zeta}{\upzeta}
% \renewcommand{\eta}{\upeta}
% \renewcommand{\theta}{\uptheta}
% \renewcommand{\vartheta}{\upvartheta}
% \renewcommand{\iota}{\upiota}
% \renewcommand{\kappa}{\upkappa}
% \renewcommand{\lambda}{\uplambda}
% \renewcommand{\mu}{\upmu}
% \renewcommand{\nu}{\upnu}
% \renewcommand{\xi}{\upxi}
% \renewcommand{\pi}{\uppi}
% \renewcommand{\varpi}{\upvarpi}
% \renewcommand{\rho}{\uprho}
% %\renewcommand{\varrho}{\upvarrho}
% \renewcommand{\sigma}{\upsigma}
% %\renewcommand{\varsigma}{\upvarsigma}
% \renewcommand{\tau}{\uptau}
% \renewcommand{\upsilon}{\upupsilon}
% \renewcommand{\varphi}{\upvarphi}
% \renewcommand{\chi}{\upchi}
% \renewcommand{\psi}{\uppsi}
% \renewcommand{\omega}{\upomega}



\newcommand {\rr}  {\mathbb{R}}
\newcommand {\nn}  {\mathbb{N}}
\newcommand {\zz}  {\mathbb{Z}}
\newcommand {\bbc} {\mathbb{C}}
% \newcommand {\rd}  {\mathbb{R}^d}
% \newcommand {\rpo} {\mathbb{R}_+^1}

\newcommand {\al} {\alpha}
\newcommand {\be} {\beta}
\newcommand {\da} {\delta}
\newcommand {\Da} {\Delta}
\newcommand {\Dl} {\Delta}
\newcommand {\ga} {\gamma}
\newcommand {\Ga} {\Gamma}
\newcommand {\la} {\lambda}
\newcommand {\La} {\Lambda}
\newcommand {\om} {\omega}
\newcommand {\Om} {\Omega}
\newcommand {\sa} {\sigma}
\newcommand {\Sa} {\Sigma}
\newcommand {\te} {\theta}
\newcommand {\fy} {\varphi}
\newcommand {\Fy} {\varPhi}
\newcommand {\ep} {\varepsilon}
\newcommand {\ro} {\varrho}

\newcommand{\bd}{{\bf{d}}}
\newcommand{\bj}{{\bf{j}}}
\newcommand{\bi}{{\bf{i}}}
\newcommand{\bk}{{\bf{k}}}
\newcommand{\bu}{{\bf{u}}}
\newcommand{\bx}{{\bf{x}}}
\newcommand{\bl}{{\bf{l}}}

\newcommand{\bma}{{\bm{\alpha}}}
\newcommand{\bmb}{{\bm{\beta}}}
\newcommand{\bmg}{{\bm{\gamma}}}
\newcommand{\bal}{{\bar{\alpha}}}
\newcommand{\bxi}{{\bar{\xi}}}
\newcommand{\bmx}{{\bm{\xi}}}
\newcommand{\bmy}{{\bm{\eta}}}

\newcommand {\ra}  {\rightarrow}
\newcommand {\IN}  {\subset}
\newcommand {\NI}  {\supset}
\newcommand {\mmm} {\setminus}
\newcommand {\8}   {\infty}
\newcommand {\io}  {I^\infty}
\newcommand {\ia}  {I^*}
\newcommand {\0}   {\varnothing}
\newcommand {\dd}  {\partial}
\newcommand {\pr}  {\mathrm{pr}}
% \renewcommand{\span}{\mathrm{span}}


\newcommand {\eA} {{\EuScript A}}
\newcommand {\eJ} {{\EuScript J}}
\newcommand {\eC} {{\EuScript C}}
\newcommand {\eU} {{\EuScript U}}
\newcommand {\eP} {{\EuScript P}}
\newcommand {\eS} {{\EuScript S}}
\newcommand {\eW} {{\EuScript W}}
\newcommand {\eZ} {{\EuScript Z}}
\newcommand {\eK} {{\EuScript K}}
\newcommand {\cK} {{\mathcal K}}
\newcommand {\hT} {{\hat T}}
\newcommand {\tT} {{\tilde T}}
\newcommand {\wP} {{\widetilde P}}
\newcommand {\eV} {{\mathcal V}}

\def \diam {\mathop{\rm diam} \nolimits}
\def \sup  {\mathop{\rm sup}  \nolimits}
\def \fix  {\mathop{\rm fix}  \nolimits}
\def \Lip  {\mathop{\rm Lip}  \nolimits}
\def \min  {\mathop{\rm min}  \nolimits}
\def \Ln   {\mathop{\rm Ln}   \nolimits}
\def \Id   {\mathop{\rm Id}   \nolimits}

\newcommand{\red}{\textcolor{red}}       % Новые переменные, которые могут использоваться во всём проекте
\input{Synopsis/setup}        % Упрощённые настройки шаблона
\input{Synopsis/synstyles}    % Стили для автореферата
\input{Synopsis/userstyles}   % Стили для специфических пользовательских задач


\begin{document}

\thispagestyle{empty}

\noindent\hfill\large{На правах рукописи}\\


\vspace{0pt plus1fill}
\begin{center}
\textbf {\large Дроздов Дмитрий Алексеевич}
\end{center}

\vspace{0pt plus3fill}
\begin{center}
\textbf {\Large Анализ на самоподобных множествах с конечным пересечением}

\vspace{0pt plus3fill}
{\large Специальность 01.01.01\ "---\\
<<Вещественный, комплексный и функциональный анализ>>}


\vspace{0pt plus1.5fill}
\Large{Автореферат}\par
\large{диссертации на соискание учёной степени\par кандидата физико-математических наук}
\end{center}

\vspace{0pt plus4fill}
{\centering Новосибирск~--- 2024\par}

% оборотная сторона обложки
\newpage
\thispagestyle{empty}
\noindent Работа выполнена в {институте математики имени С. Л. Соболева СО РАН}.

\vspace{0.008\paperheight plus1fill}
\noindent%
\begin{tabularx}{\textwidth}{@{}lX@{}}
    Научный руководитель:   & д-р физ.-мат. наук, доц.\par
                              \textbf{Тетенов Андрей Викторович}
                              \vspace{0.013\paperheight}\\
    Официальные оппоненты:  &
    \ifnumequal{\value{showopplead}}{0}{\vspace{13\onelineskip plus1fill}}{%
        \textbf{opponentOneFio,}\par
        opponentOneRegalia,\par
        opponentOneJobPlace,\par
        opponentOneJobPost\par
        \vspace{0.01\paperheight}
        \textbf{opponentTwoFio,}\par
        opponentTwoRegalia,\par
        opponentTwoJobPlace,\par
        opponentTwoJobPost
    }%
    \vspace{0.013\paperheight} \\
    \ifdefined\leadingOrganizationTitle
    Ведущая организация:    &
    \ifnumequal{\value{showopplead}}{0}{\vspace{6\onelineskip plus1fill}}{%
        \leadingOrganizationTitle
    }%
    \fi
\end{tabularx}
\vspace{0.008\paperheight plus1fill}

\noindent Защита состоится \defenseDate~на~заседании диссертационного совета \defenseCouncilNumber~при \defenseCouncilTitle~по адресу: \defenseCouncilAddress.

\vspace{0.008\paperheight plus1fill}
\noindent С диссертацией можно ознакомиться в библиотеке \synopsisLibrary.

\vspace{0.008\paperheight plus1fill}
\noindent Отзывы на автореферат в двух экземплярах, заверенные печатью учреждения, просьба направлять по адресу: \defenseCouncilAddress, ученому секретарю диссертационного совета~\defenseCouncilNumber.

\vspace{0.008\paperheight plus1fill}
\noindent{Автореферат разослан \synopsisDate.}

\noindent Телефон для справок: \defenseCouncilPhone.

\vspace{0.008\paperheight plus1fill}
\noindent%
\begin{tabularx}{\textwidth}{@{}%
>{\raggedright\arraybackslash}b{18em}@{}
>{\centering\arraybackslash}X
r
@{}}
    Ученый секретарь\par
    диссертационного совета\par
    \defenseCouncilNumber,\par
    \defenseSecretaryRegalia
    &
    \ifnumequal{\value{showsecrsign}}{0}{}{%
        \includegraphics[width=2cm]{secretary-signature.png}%
    }%
    &
    \defenseSecretaryFio
\end{tabularx}

\mainmatter*                  % Нумерация страниц не изменится, но начнётся с новой страницы
\input{Synopsis/content}      % Содержание автореферата

%%% Выходные сведения типографии
\newpage\thispagestyle{empty}

\vspace*{0pt plus1fill}

\small
\begin{center}
    \textit{\thesisAuthor}
    \par\medskip

    \thesisTitle
    \par\medskip

    Автореф. дис. на соискание ученой степени \thesisDegreeShort
    \par\bigskip

    Подписано в печать \blank[\widthof{999}].\blank[\widthof{999}].\blank[\widthof{99999}].
    Заказ № \blank[\widthof{999999999999}]

    Формат 60\(\times\)90/16. Усл. печ. л. 1. Тираж 100 экз.
    %Это не совсем формат А5, но наиболее близкий, подробнее: http://ru.wikipedia.org/w/index.php?oldid=78976454

    Типография \blank[0.5\linewidth]
\end{center}
\cleardoublepage

\end{document}