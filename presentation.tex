\documentclass[aspectratio=1610, 10pt, notheorems]{beamer}
\usetheme{Warsaw} 
\usecolortheme{seahorse}

%%% Кодировка и локализация %%%
\usepackage[utf8]{inputenc} % кодовая страница документа
\usepackage[T2A]{fontenc} % внутренняя кодировка  TeX
\usepackage[english,russian]{babel} % локализация и переносы

%%% Гипперссылки и pdf %%%
\usepackage{cmap} % русский поиск в pdf
\usepackage{hyperref} % гиперссылки

%%% Пакеты для формул %%%
\usepackage{amsmath} % удобная вёрстка многострочных формул, масштабирующийся текст в формулах, формулы в рамках и др.
\usepackage{amssymb} % несколько сотен дополнительных математических символов
\usepackage{amsthm} % окружения «теорема», «лемма» и т. п.
\usepackage{amsfonts} % Ажурный \mathbb{} и готический \mathfrak{} шрифты
\usepackage{mathrsfs} % Mathematical Script letters (шрифт Эйлера) \mathscr{}
\usepackage{euscript} % Шрифт Евклида \EuScript{}
% каллиграфический шрифт \mathcal{} не требует пакета
\usepackage{enumitem}
\usepackage{bm}
\usepackage{multicol}

%%% Графика, изображения и цвета %%%
\usepackage{graphicx} % Работа с графикой \includegraphics{}
\usepackage{float} % Фиксация плавабщих боксов [H]
\graphicspath{
    {./images/}, 
    {./images/img1/}, 
    {./images/img2/}, 
    {./images/img3/}, 
    {./images/tikz1/}, 
    {./images/tikz2/}, 
    {./images/tikz3/},
    {./images/tree/},
    {./images/presentation/},
    }
    
\newtheorem{theorem}     {Теорема}
\newtheorem{lemma}       {Лемма}
\newtheorem{proposition} {Предложение}
\newtheorem{corollary}   {Следствие}
\newtheorem{definition}  {Определение}
\newtheorem{example}     {Пример} 
\newtheorem{remark}      {Замечание}
   
%%% Настройки презентации %%%
\everymath{\color{blue}}    % цвет формул в текстмоде
\everydisplay{\color{blue}} % цвет выключных формул
\setbeamertemplate{navigation symbols}{} % Убрать панель навигации
% \setbeamertemplate{footline}[frame number] % Номера слайдов вместо нижней строки
\setbeamertemplate{page number in head/foot}[totalframenumber]
\setbeamertemplate{bibliography item}{\insertbiblabel}
\hypersetup{unicode=true} % разрешить использование символов Юникод в закладках. настройка для hiperref

%%% Русская традиция начертания математических знаков
\renewcommand{\le}  {\ensuremath{\leqslant}}
\renewcommand{\leq} {\ensuremath{\leqslant}}
\renewcommand{\ge}  {\ensuremath{\geqslant}}
\renewcommand{\geq} {\ensuremath{\geqslant}}
\renewcommand{\emptyset}{\varnothing}

%%% Русская традиция начертания математических функций (на случай копирования из зарубежных источников)
\renewcommand{\tan}{\operatorname{tg}}
\renewcommand{\cot}{\operatorname{ctg}}
\renewcommand{\csc}{\operatorname{cosec}}

%%% Русская традиция начертания греческих букв (греческие буквы вертикальные, через пакет upgreek)
% \usepackage{upgreek} % прямые греческие ради русской традиции
% \renewcommand{\epsilon}{\ensuremath{\upvarepsilon}}   %  русская традиция записи
% \renewcommand{\phi}{\ensuremath{\upvarphi}}
% %\renewcommand{\kappa}{\ensuremath{\varkappa}}
% \renewcommand{\alpha}{\upalpha}
% \renewcommand{\beta}{\upbeta}
% \renewcommand{\gamma}{\upgamma}
% \renewcommand{\delta}{\updelta}
% \renewcommand{\varepsilon}{\upvarepsilon}
% \renewcommand{\zeta}{\upzeta}
% \renewcommand{\eta}{\upeta}
% \renewcommand{\theta}{\uptheta}
% \renewcommand{\vartheta}{\upvartheta}
% \renewcommand{\iota}{\upiota}
% \renewcommand{\kappa}{\upkappa}
% \renewcommand{\lambda}{\uplambda}
% \renewcommand{\mu}{\upmu}
% \renewcommand{\nu}{\upnu}
% \renewcommand{\xi}{\upxi}
% \renewcommand{\pi}{\uppi}
% \renewcommand{\varpi}{\upvarpi}
% \renewcommand{\rho}{\uprho}
% %\renewcommand{\varrho}{\upvarrho}
% \renewcommand{\sigma}{\upsigma}
% %\renewcommand{\varsigma}{\upvarsigma}
% \renewcommand{\tau}{\uptau}
% \renewcommand{\upsilon}{\upupsilon}
% \renewcommand{\varphi}{\upvarphi}
% \renewcommand{\chi}{\upchi}
% \renewcommand{\psi}{\uppsi}
% \renewcommand{\omega}{\upomega}



\newcommand {\rr}  {\mathbb{R}}
\newcommand {\nn}  {\mathbb{N}}
\newcommand {\zz}  {\mathbb{Z}}
\newcommand {\bbc} {\mathbb{C}}
% \newcommand {\rd}  {\mathbb{R}^d}
% \newcommand {\rpo} {\mathbb{R}_+^1}

\newcommand {\al} {\alpha}
\newcommand {\be} {\beta}
\newcommand {\da} {\delta}
\newcommand {\Da} {\Delta}
\newcommand {\Dl} {\Delta}
\newcommand {\ga} {\gamma}
\newcommand {\Ga} {\Gamma}
\newcommand {\la} {\lambda}
\newcommand {\La} {\Lambda}
\newcommand {\om} {\omega}
\newcommand {\Om} {\Omega}
\newcommand {\sa} {\sigma}
\newcommand {\Sa} {\Sigma}
\newcommand {\te} {\theta}
\newcommand {\fy} {\varphi}
\newcommand {\Fy} {\varPhi}
\newcommand {\ep} {\varepsilon}
\newcommand {\ro} {\varrho}

\newcommand{\bd}{{\bf{d}}}
\newcommand{\bj}{{\bf{j}}}
\newcommand{\bi}{{\bf{i}}}
\newcommand{\bk}{{\bf{k}}}
\newcommand{\bu}{{\bf{u}}}
\newcommand{\bx}{{\bf{x}}}
\newcommand{\bl}{{\bf{l}}}

\newcommand{\bma}{{\bm{\alpha}}}
\newcommand{\bmb}{{\bm{\beta}}}
\newcommand{\bmg}{{\bm{\gamma}}}
\newcommand{\bal}{{\bar{\alpha}}}
\newcommand{\bxi}{{\bar{\xi}}}
\newcommand{\bmx}{{\bm{\xi}}}
\newcommand{\bmy}{{\bm{\eta}}}

\newcommand {\ra}  {\rightarrow}
\newcommand {\IN}  {\subset}
\newcommand {\NI}  {\supset}
\newcommand {\mmm} {\setminus}
\newcommand {\8}   {\infty}
\newcommand {\io}  {I^\infty}
\newcommand {\ia}  {I^*}
\newcommand {\0}   {\varnothing}
\newcommand {\dd}  {\partial}
\newcommand {\pr}  {\mathrm{pr}}
% \renewcommand{\span}{\mathrm{span}}


\newcommand {\eA} {{\EuScript A}}
\newcommand {\eJ} {{\EuScript J}}
\newcommand {\eC} {{\EuScript C}}
\newcommand {\eU} {{\EuScript U}}
\newcommand {\eP} {{\EuScript P}}
\newcommand {\eS} {{\EuScript S}}
\newcommand {\eW} {{\EuScript W}}
\newcommand {\eZ} {{\EuScript Z}}
\newcommand {\eK} {{\EuScript K}}
\newcommand {\cK} {{\mathcal K}}
\newcommand {\hT} {{\hat T}}
\newcommand {\tT} {{\tilde T}}
\newcommand {\wP} {{\widetilde P}}
\newcommand {\eV} {{\mathcal V}}

\def \diam {\mathop{\rm diam} \nolimits}
\def \sup  {\mathop{\rm sup}  \nolimits}
\def \fix  {\mathop{\rm fix}  \nolimits}
\def \Lip  {\mathop{\rm Lip}  \nolimits}
\def \min  {\mathop{\rm min}  \nolimits}
\def \Ln   {\mathop{\rm Ln}   \nolimits}
\def \Id   {\mathop{\rm Id}   \nolimits}

\newcommand{\red}{\textcolor{red}}     % Новые переменные и переименования

%%% Данные для титульного листа %%%
\title[Анализ на самоподобных множествах с конечным пересечением]
    {Анализ на самоподобных множествах с конечным пересечением}
\author[Дмитрий Дроздов]
    {Дмитрий Дроздов \\ {\tt d.drozdov1@g.nsu.ru}}
\institute[ИМ СО РАН]{Институт математики имени С. Л. Соболева СО РАН}
\date[20.02.2024]
    {Семинар <<Геометрия, топология и их приложения>>\\10.06.2024}


\begin{document}


\begin{frame}{}
    \titlepage
\end{frame}


\begin{frame}{Самоподобное множество}

\only<1>{
    \begin{definition}[Hutchinson J. E. (1981)]
        Пусть $\eS=\{S_1, \ldots, S_m\}$ --- система сжимающих подобий в $\mathbb{R}^n$.  
        Непустой компакт $K$, удовлетворяющий уравнению  $K=\bigcup\limits_{i=1}^m S_i(K)$, будем называть {\em аттрактором системы} $\eS$ или множеством, самоподобным (инвариантным) относительно системы $\eS$.
    \end{definition}
    \begin{block}{}
        Для системы $\eS=\{S_1,\ldots,S_r\}$ сжимающих подобий в $\rr^k$  равенством $T(A)=\bigcup\limits_{i=1}^r S_i(A)$ задётся соответствующий оператор Хатчинсона на $C(\rr^k)$.
    \end{block}
    Мы обозначим как $I=\{1, \ldots, m\}$ множество индексов системы $\eS$, а также обозначим как $I^{*}=\bigcup\limits_{n=1}^\infty I^n$ множество слов $\bm{i}=i_1\ldots i_n$ конечной длинны алфавита $I$, называемых {\em мультииндексами}.
    Мы будем использовать сокращение $S_{\bj} = S_{j_{1}j_{2}\ldots j_{n}}=S_{j_{1}}S_{j_{2}} \ldots S_{j_{n}}$ и обозначим $S_{\bj}(K)$ как $K_{\bj}$.
    Множество $K_\bj$,  где $\bj=j_{1}j_{2}\ldots j_{n}$,  будем называть {\em копией порядка (итерации) $n$} множества $K$.
    \footnotetext[1]{Hutchinson J. E., Fractals and self-similarity, Indiana Univ. Math. J. \textbf{30} (1981), 713--747.}}
\only<2>{
    \includegraphics[width=0.4\textwidth]{GosperCarpet.png}
    \hfill
    \includegraphics[width=0.55\textwidth]{LevyCurve.png}}
\end{frame}


\begin{frame}{Граф-ориентированная система}
\only<1>{
    \begin{definition}[Mauldin \& Williams (1988)]
        Пусть $\{\eS_{ij}\ |\ i,j=1,\ldots,k\}$ --- набор систем сжимающих отображений, причём для некоторых $i, j$ система $\eS_{ij}$ может быть пустой.
        Пусть этим системам соответствуют операторы Хатчинсона $\{T_{ij}(A)\}$.
        Тогда набор компактов
        $$\left\{K_i=\bigcup\limits_{j=1}^k T_{ij}(K_j)\ |\ i=1,\ldots,k\right\}$$ 
        будем называть аттрактором граф-ориентированной системы из $k$ компонент.
    \end{definition}
    \centering{
        \includegraphics[width=0.5\textwidth]{GDS.pdf}}
    \footnotetext[2]{Mauldin R. D., Williams S. C., Hausdorff dimension in graph directed constructions, Trans. Amer. Math. Soc. (2) \textbf{309} (1988), 811--829.}}
\only<2>{
    \includegraphics[width=\textwidth]{gds2.png}}
\end{frame}


\begin{frame}{Дендрит}

\end{frame}


\begin{frame}{Стягиваемые $P$-полигональные  системы}
Даны многоугольник $P\IN \mathbb R^2$, множество его вершин $V_P=\{A_1,...,A_{n_P}\}$  и такая система подобий $\eS = \{S_1, \ldots, S_m\}$ в $\mathbb R^2$, что:\\
\only<1>{\[ \includegraphics[width=.7\textwidth]{presentation/2_11.png}\]}
\only<2>{
    {\bf(D1)}  для любого $i\in I$, множество $P_i=S_i(P)\IN P$;\\
\[ \includegraphics[width=.7\textwidth]{2_22.png}\]}
\only<3>{
{\bf(D1)}  для любого $i\in I$, множество $P_i=S_i(P)\IN P$;\\
{\bf(D2)}  для любых $i\neq j,\ \   i, j \in I,$ $P_i \cap P_j=V_{P_i}\cap V_{P_j}$ ;\\
\[ \includegraphics[width=.7\textwidth]{2_23.png}\]}
\only<4>{
{\bf(D1)}  для любого $i\in I$, множество $P_i=S_i(P)\IN P$;\\
{\bf(D2)}  для любых $i\neq j,\ \   i, j \in I,$ $P_i \cap P_j=V_{P_i}\cap V_{P_j}$ ;\\
{\bf(D3)}  $V_P\IN \bigcup\limits_{i\in I}S_i(V_P)$;\\
\[ \includegraphics[width=.7\textwidth]{2_24.png}\]}
\only<5>{
{\bf(D1)}  для любого $i\in I$, множество $P_i=S_i(P)\IN P$;\\
{\bf(D2)}  для любых $i\neq j,\ \   i, j \in I,$ $P_i \cap P_j=V_{P_i}\cap V_{P_j}$ ;\\
{\bf(D3)}  $V_P\IN \bigcup\limits_{i\in I}S_i(V_P)$;\\
{\bf(D4)}  множество    ${\wP} = \bigcup \limits_{i = 1}^m P_i$ стягиваемо.
\begin{definition}
Система \ $\eS $, удовлетворяющая условиям {\bf (D1)-(D4)},
называется  стягиваемой $P$-полигональной  системой подобий.
\end{definition}
\begin{theorem} Аттрактор $K$  стягиваемой  $P$-полигональной  системы подобий $\eS$ является дендритом.
\end{theorem}}
\end{frame}


\begin{frame}{Обобщённая полигональная система}
Даны многоугольник $P\IN \mathbb R^2$, множество его вершин $V_P=\{A_1,...,A_{n_P}\}$  и такая система подобий $\eS = \{S_1, \ldots, S_m\}$ в $\mathbb R^d$, что:\\
{\bf(D1)}  для любого $i\in I$, множество $P_i=S_i(P)\IN P$;\\
{\bf(D2)}  для любых $i\neq j,\ \   i, j \in I,$ $P_i \cap P_j=V_{P_i}\cap V_{P_j}$ ;\\
{\bf(D3)}  $V_P\IN \bigcup\limits_{i\in I}S_i(V_P)$;\\
{\bf(D4)}  множество    ${\wP} = \bigcup \limits_{i = 1}^m P_i$ стягиваемо.
\begin{definition}	
Если опустить условие {\bf (D1)}, то система \ $\eS $, удовлетворяющая условиям {\bf (D2)-(D4)},
называется  обобщённой $P$-полигональной  системой подобий.
\end{definition}
\end{frame}


\begin{frame}{$\delta$-деформация}

\end{frame}


\begin{frame}{}

\end{frame}


\begin{frame}{}

\end{frame}


\begin{frame}{}

\end{frame}


\begin{frame}{}

\end{frame}


\begin{frame}{}

\end{frame}


\begin{frame}{}

\end{frame}


\begin{frame}{Самоподобная граница}
\begin{definition}
{\em Критическое множество} аттрактора $K$ системы $\eS$ --- это множество $C:=\{x:\; x\in S_i(K)\cap S_j(K),\; S_i, S_j\in\eS\}$ точек попарных пересечений копий аттрактора $K$.
\end{definition}
\begin{definition}
{\em Самоподобной границей} аттрактора $K$ называется множество $\dd K$ всех $x\in K$ таких, что для некоторого $\bj\in I^*$ верно $S_\bj(x)\in C$.
\end{definition}
\end{frame}


\begin{frame}{Простой граф пересечений}
\begin{definition}
Обозначим как $\tilde{\Gamma}(\eS)$ граф, вершинам которого соответствуют копии $\{K_i:\; i\in I\}$, а пара вершин $K_i,K_j$ соединена ребром, если $K_i\cap K_j\neq\0$.
Назовём такой граф {\em простым графом пересечений} для $K(\eS)$ 
\end{definition}

\begin{theorem}[Критерий связности, Hata M. (1985)]
Аттрактор $K$ системы $\eS$ связен тогда и только тогда, когда его простой граф пересечений $\tilde{\Gamma}(\eS)$ связен.
\end{theorem}
\footnotetext[5]{Hata Masayoshi, On the structure of self-similar sets, Japan Journal of Applied Mathematics 2 (1985), 381--414.}
% \footnotetext[2]{Bandt C. and Keller K., Self-Similar Sets 2. A Simple Approach to the Topological Structure of Fractals, Mathematische Nachrichten 154, 1(1991), 27--39}
\end{frame}


\begin{frame}{Двудольный граф пересечений}
\only<1>{
\begin{definition}
Пусть множество $K=K(\eS)$ --- самоподобный континуум, обладающий свойством одноточечного пересечения.
{\em Граф пересечений} $\Gamma(\eS)$ системы $\eS$ --- это двудольный граф с долями $\eK=\{K_i:\; i\in I\}$ и $\eP=\{p:\;p\in K_i\cap K_j,\; i,j\in I, i\neq j\}$, и с множеством рёбер $E=\{(K_i,p):p\in K_i\}$.
\end{definition}
Мы называем $K_i\in \eK$ {\em белыми вершинами}, а $p\in \eP$   {\em чёрными вершинами} графа $\Ga$.\\

Набор компактов $\mathcal{A}=\{A_i,i\in I\}$ в $\mathbb{R}^n$ называют {\em системой со свойством одноточечного пересечения}, если для любых $i\neq j\in I$, пересечение $P_{ij}=A_i\cap A_j$ не более чем одноточечно.


\begin{definition}
{\em Дендрит} --- это локально связный континуум, не содержащий простых замкнутых дуг
\end{definition}
\begin{theorem}[Tetenov A. V. (2021)]
Пусть $K=K(\eS)$ --- самоподбный континуум со свойством одноточечного пересечения.
Если граф пересечений $\hat\Gamma(\eS)$ системы $\eS$ является деревом, то её аттрактор $K$ --- дендрит.
\end{theorem}
\footnotetext[6]{Tetenov A., Finiteness properties for self-similar continua, arXiv:2003.04202 (2021)}}
\end{frame}

%\begin{frame}{Двудольный граф пересечений для фрактального квадрата дендрита}
%\includegraphics[width=0.3\textwidth]{P1.png}
%\hfill
%\includegraphics[width=0.3\textwidth]{K.png}
%\hfill
%\includegraphics[width=0.3\textwidth]{BIG.pdf}
%\end{frame}


\begin{frame}{Грани единичного квадрата}
\begin{columns}
\column{0.5\textwidth}
\includegraphics[width=\textwidth]{faces.pdf}
\column{0.5\textwidth}
\begin{definition}
Пусть $\bma\in A=\{-1,0,1\}^2$, тогда {\em гранью $P_\bma$ единичного квадрата $P$} назовём множество $P_\bma:=P\cap(P+\bma)$.\\
$P\cap(P+\bma)=P_\bma=P_{-\bma}+\bma$
\end{definition}
% \begin{definition}
% Для разных $\bmb,\bma\in A$ мы будем говорить что $\bmb=(\beta_1,\ldots,\beta_k)$ больше $\bma=(\alpha_1,\ldots,\alpha_k)$ и обозначать это как $\bmb\sqsupset\bma$, если для каждой координаты $\alpha_i=\pm1$ выполняется $\alpha_i=\beta_i$.
% \end{definition}
\end{columns}
% \footnotetext[4]{M. Elekes, T. Keleti, A.M\'ath\'e, Self-similar and self-affine sets: measure of the intersection of two copies, Ergod. Th. \& Dynam. Sys. \textbf{30} (2010), 399--440}
\end{frame}


\begin{frame}{}

\end{frame}


\begin{frame}{}

\end{frame}


\begin{frame}{}

\end{frame}


\begin{frame}{}

\end{frame}


\begin{frame}{}

\end{frame}


\begin{frame}{}

\end{frame}


\begin{frame}{}

\end{frame}


\begin{frame}{}

\end{frame}


\begin{frame}{}

\end{frame}


\begin{frame}{}

\end{frame}


\begin{frame}{}

\end{frame}


\begin{frame}{}

\end{frame}


\end{document}